%%%%%%%%%%%%%%%%%%%%%%%%%%%%%%%%%%%%%%%%%%%%%%%%%%%%%%%%%%%%
\documentclass[a4paper,11pt,oneside]{article}
\usepackage[a4paper,vmargin={1.5cm,1.5cm},width=16cm]{geometry}
\usepackage{microtype}
\usepackage[style=verbose-inote,doi=false,sortcites=true,block=space,backend=bibtex]{biblatex}
\usepackage[utf8]{inputenc}
\usepackage{textcomp}
\usepackage[spanish]{babel}
\usepackage{microtype}
\usepackage{lmodern}
\usepackage{graphicx}
\usepackage{fancyhdr}
\usepackage{booktabs}
\usepackage{eurosym}
\usepackage{mathptmx}
\usepackage[T1]{fontenc}
\usepackage{hyperref}



%%%%%%%%%%%%%%%%%%%%%%%%%%%%%%%%%%%%%%%%%%%%%%%%%%%%%%%%%%%%
%% HEADERS 
%\setlength{\headheight}{1cm}
%\setlength{\headsep}{0.5cm}
%\pagestyle{fancyplain}
%\fancyhf{}
%\lhead{\fancyplain{}{\sc Short CV (CVA)}}
%\rhead{\fancyplain{}{\sc J.J. G\'omez-Cadenas}}
%\cfoot{\thepage}
%\renewcommand{\headrulewidth}{0pt} % remove lines
%\renewcommand{\footrulewidth}{0pt}

%%% HEADER
\setlength{\headheight}{1cm}
%\setlength{\headwidth}{20cm}
\setlength{\headsep}{0.5cm}
\pagestyle{fancyplain}
\fancyheadoffset[HR,HL]{2cm}
\fancyhf{}
\lhead{\raisebox{-0.4\height}{\includegraphics[height=0.9cm,keepaspectratio=true]{img/miniLogo}}}
\rhead{\fancyplain{}{\fontsize{10}{12} \selectfont \textbf{\underline{CURRíCULUM ABREVIADO (CVA)}}}}
%\cfoot{\thepage\, / parte A}
\renewcommand{\headrulewidth}{0pt} % remove lines
\renewcommand{\footrulewidth}{0pt}

%%%%%%%%%%%%%%%%%%%%%%%%%%%%%%%%%%%%%%%%%%%%%%%%%%%%%%%%%%%%
%% Hack to make math formulas bold in section titles
\makeatletter
\DeclareRobustCommand*{\bfseries}{%
  \not@math@alphabet\bfseries\mathbf
  \fontseries\bfdefault\selectfont
  \boldmath
}
\makeatother


%%%%%%%%%%%%%%%%%%%%%%%%%%%%%%%%%%%%%%%%%%%%%%%%%%%%%%%%%%%%
\def\thesection{\bf \textsf{\alph{section}}}
%\def\thesubsection{\alph{subsection}}
\def\thesubsubsection{\thesubsection.\arabic{subsubsection}}

\bibliography{biblio}
\begin{document}
\input{src/Commands.tex}
%
%\begin{center}
%{\Large \textsf{Currículum Abreviado (Short CV)--- CVA}} \\ \vspace{0.3cm}
%{\Large  \textsf{Juan José Gómez Cadenas}} \\ 
%{\Large \textsf{25-Sept-2014}} 
%\end{center}

%%%%%%%%%%%%%%%%%%%%%%%%%%%%%%%%%%%%%%%%%%%%%%%%%%%%%%%%%%%%
%% HEADING AN\begin{document}
\begin{table}[h!]
\begin{flushright}
\begin{tabular}{|l|c|}
\hline
Date CVA & 25/09/2014\\
\hline
\end{tabular}
\label{tab:DATE}
\end{flushright}
\end{table} 
\section{Personal data}

\begin{table}[h!]
\begin{center}
%\caption{Personal data}
\begin{tabular}{| l | l | l | l |}
\hline
\multicolumn{2}{|l|}{Name and surname } & \multicolumn{2}{|l|}{ Juan José Gómez Cadenas}\\
%Name & Juan José Gómez Cadenas &  &   \\
\hline
DNI & 40.915.139.W & Age & 54 \\
\hline
Researcher ID &  L-2003-2014 &  Orcid ID &  http://orcid.org/0000-0002-8224-7714 \\
\hline
\end{tabular}
\label{tab:personal}
\end{center}
\end{table} 

\subsection{Academic data}
\begin{table}[h!]
\begin{center}
%\caption{Professional data}
\begin{tabular}{| l | l | l | l |}
\hline
\multicolumn{2}{|l|}{Institution} & \multicolumn{2}{|l|}{ Consejo Superior de Investigaciones Científicas  (CSIC)}\\
\hline
\multicolumn{2}{|l|}{Department} & \multicolumn{2}{|l|}{ Instituto de Física Corpuscular (IFIC) }\\
\hline
\multicolumn{2}{|l|}{Adress} & \multicolumn{2}{|l|}{ Catedrático José Beltrán, 2  }\\
\hline
Phone & 654213611 & e-mail & gomez@mail.cern.ch\\
\hline
\multicolumn{2}{|l|}{UNESCO code} & \multicolumn{2}{|l|}{ 220700  }\\
\hline
\multicolumn{2}{|l|}{keywords} & \multicolumn{2}{|l|}{ Majorana neutrinos, double beta decay, xenon, neutrino oscillations}\\
\hline
\end{tabular}
\label{tab:profesional}
\end{center}
\end{table} 

\subsection{Professional data}
\begin{table}[h!]
\begin{center}
%\caption{Professional data}
\begin{tabular}{| l | l | l | }
\hline
Degree & University & Year\\
\hline
Degree (equivalent to Master) in Physics & University of Valencia & 1983 \\
Ph.D. in Physics & University of Valencia & 1987 \\
\hline
\end{tabular}
\label{tab:academic}
\end{center}
\end{table} 

\subsection{Indicators of scientific production}

\begin{table}[h!]
\begin{center}
\caption{Citations according to SPIRES data base}
\begin{tabular}{| l | l | l | }
\hline
Cittation summary results & Citable papers & Citable and published\\
\hline
Total number of papers analysed &	253	&215\\
Total number of citations &	18,243	& 17,257 \\
Average citations per paper &	72.1 &	80.3 \\
Renowned papers (500+)	 & 4	& 4 \\
Famous papers (250-499)	 & 5	& 5 \\
Very well-known papers (100-249) &	21 &	17\\
Well-known papers (50-99) &	50 &	48\\
Known papers (10-49) &	119	& 115\\
h index &63 &	61\\
\hline
\end{tabular}
\label{tab:spires}
\end{center}
\end{table} 

\begin{itemize}
\item {\bf Number of research 6-years steps (sexenios) }: 4 out of 4. The last approved period was 2004-2009. 
\item {\bf Ph.D. thesis advised}: The researcher has supervised a total of 11 Ph.D. students: De Fez (94), Lozano (95), Hernando (98), Cervera (02), Vidal (03), Burguet (08), Tornero (08), Novella (09), Catala (14) Martin-Albo (14) and Monrabal (14), six of whom have presented their thesis in the last 10 years (3 of them in 2014): De Fez, Lozano, Hernando and Cervera have obtained academic positions (Emirates, Granada, U. of Santiago and Valencia). Novella has been a Marie Curie fellow and has obtained a R\&C in 2014. Burguet works in a software company developing Internet products and Tornero has a permanent position in medical physics (radiotherapy). Catala and Vidal have continued their career as teachers. Martin-Albo and Monrabal will take post-doc position in the US in 2015 (Texas U. and Fermilab). 
\item{\bf Citations:} The data, according to the SPIRES database\footnote{inspirehep.net/search?ln=en\&ln=en\&p=find+a+gomez+cadenas++or+gomez+y+cadenas+or+gomez-cadenas+or+cadenas\&of=hcs\&action\_search=Search\&sf=earliestdate\&so=d\&rm=\&rg=25\&sc=0} is shown in Table \ref{tab:spires}:
\end{itemize}

\section{Summary of CV}
After his Ph.D., the researcher was a Fulbright Fellow at the Stanford Linear Accelerator Center (SLAC), USA, from 1987 to 1988. From 1988 to 1990 he was a postdoctoral associate at the Santa Cruz Institute for Particle Physics (SCIPP), USA. From 1990 to 1992 he was a CERN fellow and from 1992 to 1994 he was CERN Research Staff. From 1994 to 1996 he was an assistant Professor at the University of Massachusetts (Amherst). From 1996 to 1998 he was CERN Research Staff. In 1998 he joined the faculty at the Department of Atomic and Nuclear Physics at the University of Valencia, first as an associated professor then as a full professor. In 2006 he joined the Spanish Council for Research (CSIC) as a full professor. He has been visiting professor at the University of Harvard, at the University of Geneve and at the Japanese Laboratory for High Energy Physics, KEK. 

The researcher has made outstanding contributions to particle-physics experiments, mostly related to the physics of leptons. These include:
%
{\bf (1) SLAC (1988--1990)}, where he played a leading role in the data analysis of the Mark-II experiment, carrying the first search for lepton-flavor violating events involving $\tau$ leptons, and in the physics studies for the design of the Tau-Charm factory\footnote{Phys.\ Rev.\ Lett.\ {\bf66},1991, 1007--1010, Phys.\ Rev.\ D {\bf39} (1989) 1370; Phys.\ Rev.\ D {\bf41} (1990) 2179; Phys.\ Rev.\ D {\bf42} (1990) 3093--3099}. 
{\bf (2) DELPHI (1990--1994)}, where he was the $\tau$-physics analysis convener, and led, among others, the first analysis on $\tau$ branching ratios and polarisation. He was also one of the leaders in the construction of the upgrade of the DELPHI microvertex detector\footnote{Z.\ Phys.\ C {\bf55} (1992) 555--568; Nucl.\ Instrum.\ Meth.\ A {\bf368} (1996) 314--332}. 
{\bf (3) NOMAD (1994--1998)}.  He was one of the leaders of the experiment, both in the oscillation analysis [Phys.\ Lett.\ B {\bf431} (1998) 219--236] and as chief proponent and spokesperson of the NOMAD-STAR detector [], the first silicon detector used in a neutrino experiment, which allowed the tagging of charm mesons\footnote{Phys.\ Lett.\ B {\bf431} (1998) 219--236;Nucl.\ Instrum.\ Meth.\ A {\bf506} (2003) 217--237}.
{\bf (3) Future neutrino facilities (1998--2003)}, a field where he has made 
high impact contributions, including the studies of the physics potential of the so-called Neutrino Factory and Beta-Beam. He was one of the principal authors of several seminal papers\footnote{Nucl.\ Phys.\ B {\bf579} (2000) 17--55, 571 citations; Nucl.\ Phys.\ B {\bf608} (2011) 301--318, 338 citations.} that demonstrated the feasibility of measuring leptonic CP violation in future experiments. 
{\bf K2K and T2K (2003--2009)}.  He led the Spanish effort to join K2K, the experiment that observed for the first time neutrino oscillations in a man-made neutrino beam.  The researcher also lead the initial contribution of his group to T2K.
{\bf NEXT (2009--present)}. He proposed the NEXT experiment (jointly with Dr. Nygren), and formed the NEXT collaboration, of which he is the spokesperson. NEXT is a leading experiment to search for neutrino less double beta decays, whose discovery would imply that the neutrino is its own antiparticle. 

\section{Most relevant merits}
\subsection{Publications}

\subsubsection*{T2K, K2K and HARP publications}
\begin{itemize}
\item (1) {\bf T2K Collaboration} (K.~Abe {\it et al.}), {\it Indication of Electron Neutrino Appearance from an Accelerator-produced Off-axis Muon Neutrino Beam}, Phys.\ Rev.\ Lett.\  {\bf 107}, 041801 (2011). ({\it 855 citations}). (2) {\bf K2K Collaboration} (M.~H.~Ahn {\it et al.}), {\it Measurement of Neutrino Oscillation by the K2K Experiment}, Phys.\ Rev.\ D {\bf 74}, 072003 (2006). ({\it 637 citations}). (3) {\bf HARP Collaboration} (M.G.~Catanesi \textit{et al.}), \textit{Measurement of the production cross-section of positive pions in the collision of 8.9-GeV/c protons on beryllium}, Eur.\ Phys.\ J.\ C {\bf52} (2007) 29--53. ({\it 96 citations}).
\end{itemize}
The first two articles are renowned papers. In the first one,the T2K collaboration (400 authors, signature by alphabetic order) sees strong hints of muon neutrino to electron neutrino oscillation. In the second one the K2K collaboration (250 authors, signature by alphabetic order) measures neutrino oscillations  through the disappearance of $\nu_\mu$. The researcher was the leader of the IFIC T2K group until 2009, and made a major contribution to the K2K paper, through the measurement of the far-near ratio, an essential ingredient to reduce the systematic error of the measurement. The far-near ratio was computed using data obtained by the HARP collaboration (105 authors, signature by alphabetic order). The researched was the analysis convener of HARP and played a major role in the design, operation and data analysis of the experiment.
 	
\subsubsection*{Future neutrino facilities}
%
\begin{itemize}
\item (1) A. Cervera, A. Donini, M.B. Gavela, {\bf J.J. Gomez Cadenas}, P. Hernandez, Olga Mena, S. Rigolin {\it Golden measurements at a neutrino factory}, 
Nucl.Phys. B579 (2000).DOI: 10.1016/S0550-3213(00)00221-2. {\it 571 citations}
(2) R. Bayes, A. Laing, F.J.P. Soler, A. Cervera Villanueva, {\bf J.J. Gomez Cadenas}, P. Hernandez, J. Martin-Albo, J. Burguet-Castell, {\it 
The Golden Channel at a Neutrino Factory revisited: improved sensitivities from a Magnetised Iron Neutrino Detector}. Phys.Rev. D86 (2012) 093015. DOI: 10.1103/PhysRevD.86.093015. 
({\it 14 citations}).
\end{itemize}
Two papers are selected to represent the extensive work that the researcher has carried out in the field of future neutrino facilities, where he continues active. The first is a renowned paper where the golden signature of muons in a neutrino factory is proposed and the concept of the magnetic detector to detect them is described. This paper was published in 2000 and r revised in a  paper published in 2012, confirming the very good physics potential found in 2000 with a much more sophisticated simulation and analysis.  

\subsubsection*{Neutrinoless double beta decay and the NEXT experiment}
\begin{itemize}
\item	(1) {\bf NEXT collaboration} (F. Granena et al.), \textit{NEXT, a HPGXe TPC for neutrinoless double beta decay searches}. Jul 2009.e-Print: arXiv:0907.4054 ({\it 48 citations}) 
(2){\bf NEXT Collaboration} (V.~\'Alvarez {\it et al.}), \textit{NEXT-100 Technical Design Report (TDR): Executive Summary}, JINST {\bf 7}, T06001 (2012). ({\it 38 citations})
\end{itemize}

The first paper is the Letter of Intent (LOI) presented in 2009 to the Canfranc Scientific Committee. This LOI defined the NEXT experiment (and created, in practice the collaboration), based in ideas by Dr. D. Nygren (inventor of the TPC) and the researcher. The second paper is the Technical Design Report, which was written 3 years later, and specified the technical solutions adopted by the NEXT experiment. NEXT is a collaboration of about 60 authors. The researcher is the spokesperson of the collaboration. 

\begin{itemize}
\item (1) {NEXT Collaboration} (V.~\'Alvarez {\it et al.}), \textit{	
Operation and first results of the NEXT-DEMO prototype using a silicon photomultiplier tracking array}, 
JINST 8 (2013) P09011. DOI: 10.1088/1748-0221/8/09/P09011. 
e-Print: arXiv:1306.0471 ({\it 9 citations}).
(2) {\bf NEXT Collaboration} (V.~\'Alvarez {\it et al.}), \textit{Initial results of NEXT-DEMO, a large-scale prototype of the NEXT-100 detector},  JINST 8 (2013) P04002. 
DOI: 10.1088/1748-0221/8/04/P04002. 
e-Print: arXiv:1211.4838.  ({\it 12 citations}).
\end{itemize}
This two papers demonstrate the excellent energy resolution and the topological signature of the NEXT experiment, using the  NEXT-DEMO prototype. The papers have been determinant for the positive NSAC evaluation of NEXT, and the concession of an Advanced Grant of the ERC to the researcher. 
%Although the number of citations is modest (they are in the lower part of the ``known papers'' rank, defined by Spires) one has to take into account  the fact that these are instrumental  papers in a specialised field, where the number of citation is typically small. 

\begin{itemize}
\item (1) {\bf J.J.~G\'omez-Cadenas}, J.~Mart\'in-Albo, M.~Sorel, P.~Ferrario, F.~Monrabal, J.~Mu\~noz-Vidal, P.~Novella, A.~Poves, \textit{Sense and sensitivity of double beta decay experiments}, JCAP {\bf 1106} (2011) 007. ({\it 35 citations})  
(2) {\bf J.J.~G\'omez-Cadenas}, J.~Mart\'in-Albo, M.~Mezzetto, F.~Monrabal, M.~Sorel, {\em The search for neutrinoless double beta decay}, Riv.\ Nuovo Cim.\ {\bf35} (2012) 29--98, {\tt arXiv:1109.5515 [hep-ex]}. (94 citations)
\end{itemize}
These are two known review papers written by the researcher and collaborators (including several of his students), reviewing the field of neutrino less double beta decay.  

%\subsection*{Invited Talks}
%\begin{enumerate}
%%% 1
%\item \textit{Ettore Majorana through the looking glass (searching for neutrinoless double beta decay)}, Harvard Monday Colloquium, 22 October 2012; Fermilab Colloquium, 24 October, 2012; University of Wisconsin Madison seminar, 26 October 2012, CERN colloquium, January, 2013.
%%% 2
%\item \textit{NEXT, high-pressure xenon gas experiments for ultimate sensitivity to Majorana neutrinos}, invited talk at the 14th International Workshop on Radiation Imaging Detectors (iWoRID 2012), Figueira da Foz, Coimbra (Portugal), 1-5 July, 2012. 
%%% 3
%\item \textit{Xenon for DM and DBD searches}, invited talk at IDPASC Dark Matter Workshop, \'Evora (Portugal), 2011.
%%% 4
%\item \textit{Status of the NEXT experiment}, at DBD'11: International Workshop on Double Beta Decay and Neutrinos, Osaka (Japan), 2011.
%%% 5
%\item \textit{How to probe anti-neutrino = neutrino and the absolute neutrino mass scale}, International Neutrino Summer School, Geneva (Switzerland), 2011.
%%% 6
%\item \textit{Sense and sensitivity in \bbonu\ experiments} at XIV International Workshop on Neutrino Telescopes, Venice (Italy), 2010.
%%% 7
%\item {\it Ettore Majorana meets his shadow (searching for neutrino less double beta decay)}, Wednesday colloquium, Weizmann institute, 24 November 2010.
%%% 8
%\item \textit{The physics case of the Neutrino Factory}, at 23rd International Conference on Neutrino Physics and Astrophysics (Neutrino 2008), Christchurch (New Zealand).
%%% 9
%\item {\it Lectures on Neutrino Physics}, CERN Summer Student Programme, Geneva (Switzerland), 2004--2009.
%%% 10
%\item \textit{Measuring leptonic CP violation in future neutrino facilities} at the school \textit{CP violation: From quarks to leptons}, Varenna (Italy), 2005.
%\end{enumerate}
%
%\subsubsection*{Organization of international conferences}
%\begin{enumerate}
%%% 1
%%% 1
%\item \textit{Neutrino 2014}, Boston, 2014. \emph{Member of the International Advisory Committee.}
%%% 1
%\item \textit{Weak Interactions and Neutrinos, WIN 2013}, Natal, Brazil, September 2013. \emph{Member of the International Advisory Committee.}
%
%\item \textit{1st Workshop on Xenon-based Detector}, LBNL, Berkeley, Nov 2009. \emph{Member of the International Advisory Committee.}
%%% 2
%\item \textit{10th International Workshop on Neutrino Factories, Super-beams and Beta-beams (NuFact 08)}, IFIC, Valencia, July 2008. \emph{Co-chair.}
%%% 3
%\item \textit{International Workshop on the Golden Channel at a Neutrino Factory}, IFIC, Valencia, June 2007. \emph{Chair}.
%\end{enumerate}


\subsection{Participation in research grants}
\begin{itemize}

\item {\bf AdG/ERC number 339787}: This is an advanced grant of the ERC, granted to the researcher in 2013, with starting date in February 2014. Its goal is to co-fund the construction of the NEXT experiment (it fully finances the first stage of the experiment, the NEW detector). 
% 
\item {\bf FIS2012-37947-C04-01}: This is a MINECO coordinated project involving IFIC (Valencia),  UPV (Valencia) and UAM (Madrid) to fund the participation of our groups in NEXT. The researcher is the coordinator of the project, which involves about 0.5 M\euro.
%%
\item {\bf Project Consolider-Ingenio 2010 CUP} (CSD2008-0037), 2009--2013: This project was awarded by the Spanish MICINN to kick-start the NEXT experiment. The funding was 5 M\euro. It involved 7 institutes and about 40 physicists. 
%%
\item {\bf FPA2009-2011 13697-C04-04}: This is a MICINN coordinated project to fund the participation of several spanish groups in NEXT and T2K. The researcher was the coordinator of the project, which involved about 1 M\euro.
%%
\item {\bf LAGUNA-LBNO}: This is a European grant whose ultimate goal is the design of a large underground neutrino facility in Europe. The funding of the project is 60 k\euro. The researcher is the IP of the project. 
\end{itemize}
%
\subsection{Patents}
The NEXT project may originate a number of patents, related wit the new technologies currently being developed by the collaboration.

\subsection{International committees}
%
The researcher served for three years in the LHC Committee (LHCC, 2001-2004). He has been involved in the organisation of the \textit{Neutrino Factory Conference series (NuFact)} almost since its beginning. He has been nominated a member of the International Advisory Panel of the two most important conferences of his field, Weak Interactions and Neutrinos (WIN2013) and Neutrino (Neutrino2014). 

\subsection{International schools}
The researcher teaches regularly in numerous international schools, notably at the CERN Summer Student Programme (2004--2009), the 
Neutrino Factory Schools editions in KEK (2007), Benasque (2008), Fermilab (2009) and Geneva (2011), and at the ISAPP schools at
Gran Sasso (2012) and Canfranc (2013). In 2014 he has taught at the prestigious International Neutrino Summer School 2014 in St. Andrews and at the Gran Sasso Summer Institute. He was the organiser and chair of the Neutrino Summer School in Benasque (2008).

\subsection{Invited talks}
The researcher has presented numerous invited talks during his career. The most recent are: 
(1) {\em The NEXT experiment}, in ``The art of Experiment'', LBNL, California, May 3, 2014\footnote{http://nygrensymposium2014.lbl.gov}. 
(2) \textit{Ettore Majorana through the looking glass (searching for neutrinoless double beta decay)}, Harvard Monday Colloquium, 22 October 2012; Fermilab Colloquium, 24 October, 2012; University of Wisconsin Madison seminar, 26 October 2012, CERN colloquium, January, 2013.

\subsection{Outreach}
(1) He is the director of the science section of the cultural magazine {\em JotDown}, where he
also keeps a widely read blog\footnote{http://www.jotdown.es/category/ciencias/fasterthanlight/}.
(2) He published outreach articles, often interviewing scientific personalities such as Francis Halzen, PI of IceCube\footnote{http://www.jotdown.es/2014/05/francis-halzen-i-always-advise-to-my-students-dont-read-too-many-books-do-things/}, Dave Nygren (inventor of the TPC) and Alessandro Bettini (director of the Canfranc underground lab)\footnote{http://www.jotdown.es/2012/09/david-nygren-y-alessandro-bettini-the-physics-as-fountain-of-eternal-youth/}, among others. (3) He is the director of the scientific section of the JDB editorial brand. Among others, he has translated to spanish the book of G.F. Giudice ``Odyssey in the Zeptospace''. (4) He writes in numerous spanish cultural media, including the prestigious ``Review of books'', where he comments on scientific books and articles\footnote{http://www.revistadelibros.com/autores/428/juan-jose-gomez-cadenas}. (5) He gives numerous public lectures to the general public\footnote{
http://www.jotdown.es/2014/07/paisaje-sin-neutrinos/}.  




\end{document}