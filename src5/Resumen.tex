%%

NEXT (Neutrino Experiment with a Xenon TPC) es un experimento para buscar desintegraciones doble beta sin neutrinos (\bbonu), cuya detección demostraría unívocamente que el neutrino es una partícula de Majorana (es decir su propia antipartícula) y supondría un descubrimiento con profundas consecuencias en física de partículas y cosmología. 

La primera fase de NEXT ha sido completada con éxito. Durante esta etapa el grupo del IFIC ha construido el prototipo NEXT-DEMO que opera en nuestro laboratorio desde 2011 y ha sido esencial para  demostrar las características principales de la tecnología, a saber: excelente resolución de energía y caracterización de la señal mediante la reconstrucción de las trayectorias de los dos electrones emitidos en la desintegración \bbonu. 

Este  proyecto requiere financiación para demostrar una nueva idea que aumentaría dramáticamente las posibilidades de realizar un descubrimiento y situaría a NEXT como el líder mundial del campo. Concretamente se propone añadir un campo magnético al experimento, capaz de aportar una caracterización adicional a la señal, ya que las trayectorias de los dos electrones emitidos en el proceso \bbonu\ se curvarían siguiendo una doble hélice. Para demostrar experimentalmente esta señal, se propone operar NEXT-DEMO en un imán solenoidal (TPC90) disponible en el CERN.

Este proyecto propone también la demostración de un avance radical en tecnología: el uso de sensores de estado sólido (SiPMs) de bajo ruido y capaces de funcionar en campo magnético, y que reemplazarían a los fotomultiplicadores con los que NEXT-DEMO mide la energía de los sucesos. Este desarrollo tendría inmediatas aplicaciones a la técnica PET-MRI que combina tomografía de electrón-positrón con resonancia magnética y necesita, por tanto, cubrir amplias áreas con sensores de bajo ruido capaces de operar en campo magnético.   

