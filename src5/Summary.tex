NEXT (Neutrino Experiment with a Xenon TPC) is an experiment to search for neutrinoless double beta decay events. Detecting such events would be a major discovery that would demonstrate that the neutrino is a Majorana particle (that is, its own antiparticle). 

The first phase of NEXT has been completed successfully. During this period the IFIC group has built the prototype NEXT-DEMO, which operates in our laboratory since 2011 and has been essential to demonstrate the main features of the technology, namely: excellent energy resolution and characterization of the signal by means of the reconstruction of the trajectories of the two electrons emitted in the \bbonu\ decay.

This project requires funding to demonstrate a new idea that would boost dramatically the potential to make a discovery and would place NEXT as the World leader for the field. Specifically we propose to add a magnetic field to the experiment, capable to bring in an additional characterization to the signal, since the trajectories of the two electrons emitted in the \bbonu\ decay would follow a double helix. To demonstrate experimentally this signal, we propose to operate NEXT-DEMO in a solenoid magnet (TPC90) available at CERN.

This project proposes also the demonstration of a radical advance in technology: the use of solid state sensors (SiPMs) of low noise capable to operate in a magnetic field. The SiPMs would replace the PMTs that NEXT-DEMO uses to measure the energy of the events. This development would have immediate applications to the PET-MRI technology where electron-positron tomography is combined with magnetic resonance imaging. PET-MRI devices need to cover large areas with low-noise sensors capable to operate in a magnetic field, such as the SiPMs that we propose to use for this project
