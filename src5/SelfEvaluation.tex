\vspace{6pt}

Responda de forma escueta a las siguientes preguntas:

\vspace{6pt}

\noindent\textbf{1. Cuál o cuáles son, en su opinión, los puntos más importantes por los que su propuesta deba ser financiada?}

\noindent {\color{blue}{Máximo 500 caracteres}}
\vspace{6pt}

\noindent To demonstrate: a) a scientific breakthrough, namely the fact that adding a magnetic field to NEXT would increase the rejection of backgrounds by an order of magnitude, turning it into the best detector in the World to discover \bbonu\ decays; b) a technological break-through, namely, the demonstration of the possibility of using newly available low-noise SiPM to measure the energy of events in a magnetic field replacing the conventional PMTs which do not work well in magnetic fields.   

\vspace{6pt}

\noindent\textbf{2. Cuál o cuáles son, en su opinión, los puntos más débiles de su propuesta? En otras palabras, ¿por qué su solicitud sería rechazada en una convocatoria clásica de otros programas?}

\noindent{\color{blue}{Máximo 500 caracteres}}

\vspace{6pt}

\noindent The NEXT experiment is a high-pressure xenon chamber based in the use of PMTs for the measurement of the energy and operating without a magnetic field. This proposal is a new development that departs from the approved design, introducing risk (the use of newly developed devices as opposed to very well understood PMTs, and operation in a magnetic field). Precisely because of this risk, a classical call would likely turn down or postpone this development. 

\vspace{6pt}

\noindent\textbf{3. Resalte los aspectos de su experiencia investigadora que demuestran que está usted (y si es el caso, su equipo) preparado para afrontar este estudio.}

\noindent {\color{blue}{Máximo 500 caracteres}}

\vspace{6pt}

\noindent I am the proponent of the NEXT experiment, the spokesperson of the NEXT collaboration, and the PI of the most important team in NEXT. My group has built NEXT-DEMO, a large scale (1/20) prototype of the detector (which would be reused for this proposal) and is now completing the construction of the first-phase of the NEXT detector. My group includes very experienced senior physicists, post-docs and engineers. We are considered to be one of the top groups in the field. 

\vspace{6pt}

\noindent\textbf{4. Indique si ha presentado esta propuesta, u otra de contenido similar, a alguna convocatoria de proyectos de investigación.}

\noindent {\color{blue}{Máximo 500 caracteres}}

\vspace{6pt}

\noindent I am the PI of a coordinated project, presented this year, whose focus is the construction, commission and operation of the NEXT experiment. However, this is a new proposal, which, if successful, would open the possibility of a major upgrade of the technology. It makes use of the availability of the DEMO prototype, and the availability of a suitable magnet at CERN.

\vspace{6pt}
\noindent\textbf{5. Cuál es el grado de confidencialidad que solicita para la evaluación de su propuesta?}

\noindent Muy alto~  ;~~~~~~~~~~ Alto  ; ~~~~~~~~~~ Normal  x;  ~~~~~~~~~~No es relevante ;

\vspace{6pt}

\noindent\textbf{6. Justificación del presupuesto.}

\noindent {\color{blue}{Máximo 2000 caracteres}}

\vspace{6pt}


\noindent This proposal re-uses the NEXT-DEMO prototype, which was essential to demonstrate the NEXT technology and is currently in operation at IFIC. DEMO is a full TPC, containing about 1 kg of xenon at 10 bar, and reading the signals produced in the chamber with a "tracking plane" made with circa 300 1 mm$^2$ SiPMs, and an "energy plane" made with 19 pressure-resistant PMTs. The system includes a field cage, high-voltage, pressure vessel, vacuum equipment, gas system, front-end electronics, data acquistion, and offline processing of the data. Although technologically advanced, the system is robust and modular and can be transported. This proposal requires funding for: a) upgrade the energy plane, replacing the PMTs by new-generation, low-noise SiPMs;  b) transport and install the system in an experimental area at CERN, inside the TPC90 magnet, which is available for the use of NEXT (thanks to our status as CERN recognised experiment); c) support the experimental campaign at CERN (essentially travel and living expense for the team). The availability at no cost of the full DEMO setup and the TPC90 magnet makes this proposal very competitive funding-wise. No funds for hiring personnel are required, since the IFIC group has built and operated DEMO for 3 years. The experience, motivation and well organised logistics of the group, which includes many top-class post-docs and grad-students, allows us to include this project in our research program without jeopardising the main research line (the construction of the NEXT experiment).  



