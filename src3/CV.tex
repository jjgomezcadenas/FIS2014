After his Ph.D., the researcher was a Fulbright Fellow at the Stanford Linear Accelerator Center (SLAC), USA, from 1987 to 1988. From 1988 to 1990 he was a postdoctoral associate at the Santa Cruz Institute for Particle Physics (SCIPP), USA. From 1990 to 1992 he was a CERN fellow and from 1992 to 1994 he was CERN Research Staff. From 1994 to 1996 he was an assistant Professor at the University of Massachusetts (Amherst). From 1996 to 1998 he was CERN Research Staff. In 1998 he joined the faculty at the Department of Atomic and Nuclear Physics at the University of Valencia, first as an associated professor then as a full professor. In 2006 he joined the Spanish Council for Research (CSIC) as a full professor. He has been visiting professor at the University of Harvard, at the University of Geneve and at the Japanese Laboratory for High Energy Physics, KEK. 

The researcher has made outstanding contributions to particle-physics experiments, mostly related to the physics of leptons. These include:
%
{\bf (1) SLAC (1988--1990)}, where he played a leading role in the data analysis of the Mark-II experiment, carrying the first search for lepton-flavor violating events involving $\tau$ leptons, and in the physics studies for the design of the Tau-Charm factory\footnote{Phys.\ Rev.\ Lett.\ {\bf66},1991, 1007--1010, Phys.\ Rev.\ D {\bf39} (1989) 1370; Phys.\ Rev.\ D {\bf41} (1990) 2179; Phys.\ Rev.\ D {\bf42} (1990) 3093--3099}. 
{\bf (2) DELPHI (1990--1994)}, where he was the $\tau$-physics analysis convener, and led, among others, the first analysis on $\tau$ branching ratios and polarisation. He was also one of the leaders in the construction of the upgrade of the DELPHI microvertex detector\footnote{Z.\ Phys.\ C {\bf55} (1992) 555--568; Nucl.\ Instrum.\ Meth.\ A {\bf368} (1996) 314--332}. 
{\bf (3) NOMAD (1994--1998)}.  He was one of the leaders of the experiment, both in the oscillation analysis \footnote{Phys.\ Lett.\ B {\bf431} (1998) 219--236} and as chief proponent and spokesperson of the NOMAD-STAR detector, the first silicon detector used in a neutrino experiment, which allowed the tagging of charm mesons\footnote{Phys.\ Lett.\ B {\bf431} (1998) 219--236;Nucl.\ Instrum.\ Meth.\ A {\bf506} (2003) 217--237}.
{\bf (3) Future neutrino facilities (1998--2003)}, a field where he has made 
high impact contributions, including the studies of the physics potential of the so-called Neutrino Factory and Beta-Beam. He was one of the principal authors of several seminal papers\footnote{Nucl.\ Phys.\ B {\bf579} (2000) 17--55, 571 citations; Nucl.\ Phys.\ B {\bf608} (2011) 301--318, 338 citations.} that demonstrated the feasibility of measuring leptonic CP violation in future experiments. 
{\bf K2K and T2K (2003--2009)}.  He led the Spanish effort to join K2K, the experiment that observed for the first time neutrino oscillations in a man-made neutrino beam.  The researcher also lead the initial contribution of his group to T2K.
{\bf NEXT (2009--present)}. He proposed the NEXT experiment (jointly with Dr. Nygren), and formed the NEXT collaboration, of which he is the spokesperson. NEXT is a leading experiment to search for neutrino less double beta decays, whose discovery would imply that the neutrino is its own antiparticle. 
