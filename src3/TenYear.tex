
\subsubsection*{T2K, K2K and HARP publications}
\begin{itemize}
\item (1) {\bf T2K Collaboration} (K.~Abe {\it et al.}), {\it Indication of Electron Neutrino Appearance from an Accelerator-produced Off-axis Muon Neutrino Beam}, Phys.\ Rev.\ Lett.\  {\bf 107}, 041801 (2011). ({\it 855 citations}). (2) {\bf K2K Collaboration} (M.~H.~Ahn {\it et al.}), {\it Measurement of Neutrino Oscillation by the K2K Experiment}, Phys.\ Rev.\ D {\bf 74}, 072003 (2006). ({\it 637 citations}). (3) {\bf HARP Collaboration} (M.G.~Catanesi \textit{et al.}), \textit{Measurement of the production cross-section of positive pions in the collision of 8.9-GeV/c protons on beryllium}, Eur.\ Phys.\ J.\ C {\bf52} (2007) 29--53. ({\it 96 citations}).
\end{itemize}
The first two articles are renowned papers. In the first one,the T2K collaboration (400 authors, signature by alphabetic order) sees strong hints of muon neutrino to electron neutrino oscillation. In the second one the K2K collaboration (250 authors, signature by alphabetic order) measures neutrino oscillations  through the disappearance of $\nu_\mu$. The researcher was the leader of the IFIC T2K group until 2009, and made a major contribution to the K2K paper, through the measurement of the far-near ratio, an essential ingredient to reduce the systematic error of the measurement. The far-near ratio was computed using data obtained by the HARP collaboration (105 authors, signature by alphabetic order). The researcher was the analysis convener of HARP and played a major role in the design, operation and data analysis of the experiment.
 	
\subsubsection*{Future neutrino facilities}
%
\begin{itemize}
\item (1) A. Cervera, A. Donini, M.B. Gavela, {\bf J.J. Gomez Cadenas}, P. Hernandez, Olga Mena, S. Rigolin {\it Golden measurements at a neutrino factory}, 
Nucl.Phys. B579 (2000).DOI: 10.1016/S0550-3213(00)00221-2. {\it 571 citations}
(2) R. Bayes, A. Laing, F.J.P. Soler, A. Cervera Villanueva, {\bf J.J. Gomez Cadenas}, P. Hernandez, J. Martin-Albo, J. Burguet-Castell, {\it 
The Golden Channel at a Neutrino Factory revisited: improved sensitivities from a Magnetised Iron Neutrino Detector}. Phys.Rev. D86 (2012) 093015. DOI: 10.1103/PhysRevD.86.093015. 
({\it 14 citations}).
\end{itemize}
Two papers are selected to represent the extensive work that the researcher has carried out in the field of future neutrino facilities, where he continues active. The first is a renowned paper where the golden signature of muons in a neutrino factory is proposed and the concept of the magnetic detector to detect them is described. This paper was published in 2000 and r revised in a  paper published in 2012, confirming the very good physics potential found in 2000 with a much more sophisticated simulation and analysis.  

\subsubsection*{Neutrinoless double beta decay and the NEXT experiment}
\begin{itemize}
\item	(1) {\bf NEXT collaboration} (F. Granena et al.), \textit{NEXT, a HPGXe TPC for neutrinoless double beta decay searches}. Jul 2009.e-Print: arXiv:0907.4054 ({\it 48 citations}) 
(2){\bf NEXT Collaboration} (V.~\'Alvarez {\it et al.}), \textit{NEXT-100 Technical Design Report (TDR): Executive Summary}, JINST {\bf 7}, T06001 (2012). ({\it 38 citations})
\end{itemize}

The first paper is the Letter of Intent (LOI) presented in 2009 to the Canfranc Scientific Committee. This LOI defined the NEXT experiment (and created, in practice the collaboration), based in ideas by Dr. D. Nygren (inventor of the TPC) and the researcher. The second paper is the Technical Design Report, which was written 3 years later, and specified the technical solutions adopted by the NEXT experiment. NEXT is a collaboration of about 60 authors. The researcher is the spokesperson of the collaboration. 

\begin{itemize}
\item (1) {NEXT Collaboration} (V.~\'Alvarez {\it et al.}), \textit{	
Operation and first results of the NEXT-DEMO prototype using a silicon photomultiplier tracking array}, 
JINST 8 (2013) P09011. DOI: 10.1088/1748-0221/8/09/P09011. 
e-Print: arXiv:1306.0471 ({\it 9 citations}).
(2) {\bf NEXT Collaboration} (V.~\'Alvarez {\it et al.}), \textit{Initial results of NEXT-DEMO, a large-scale prototype of the NEXT-100 detector},  JINST 8 (2013) P04002. 
DOI: 10.1088/1748-0221/8/04/P04002. 
e-Print: arXiv:1211.4838.  ({\it 12 citations}).
\end{itemize}
This two papers demonstrate the excellent energy resolution and the topological signature of the NEXT experiment, using the  NEXT-DEMO prototype. The papers have been determinant for the positive NSAC evaluation of NEXT, and the concession of an Advanced Grant of the ERC to the researcher. 
%Although the number of citations is modest (they are in the lower part of the ``known papers'' rank, defined by Spires) one has to take into account  the fact that these are instrumental  papers in a specialised field, where the number of citation is typically small. 

\begin{itemize}
\item (1) {\bf J.J.~G\'omez-Cadenas}, J.~Mart\'in-Albo, M.~Sorel, P.~Ferrario, F.~Monrabal, J.~Mu\~noz-Vidal, P.~Novella, A.~Poves, \textit{Sense and sensitivity of double beta decay experiments}, JCAP {\bf 1106} (2011) 007. ({\it 35 citations})  
(2) {\bf J.J.~G\'omez-Cadenas}, J.~Mart\'in-Albo, M.~Mezzetto, F.~Monrabal, M.~Sorel, {\em The search for neutrinoless double beta decay}, Riv.\ Nuovo Cim.\ {\bf35} (2012) 29--98, {\tt arXiv:1109.5515 [hep-ex]}. (94 citations)
\end{itemize}
These are two known review papers written by the researcher and collaborators (including several of his students), reviewing the field of neutrino less double beta decay.  

%\subsection*{Invited Talks}
%\begin{enumerate}
%%% 1
%\item \textit{Ettore Majorana through the looking glass (searching for neutrinoless double beta decay)}, Harvard Monday Colloquium, 22 October 2012; Fermilab Colloquium, 24 October, 2012; University of Wisconsin Madison seminar, 26 October 2012, CERN colloquium, January, 2013.
%%% 2
%\item \textit{NEXT, high-pressure xenon gas experiments for ultimate sensitivity to Majorana neutrinos}, invited talk at the 14th International Workshop on Radiation Imaging Detectors (iWoRID 2012), Figueira da Foz, Coimbra (Portugal), 1-5 July, 2012. 
%%% 3
%\item \textit{Xenon for DM and DBD searches}, invited talk at IDPASC Dark Matter Workshop, \'Evora (Portugal), 2011.
%%% 4
%\item \textit{Status of the NEXT experiment}, at DBD'11: International Workshop on Double Beta Decay and Neutrinos, Osaka (Japan), 2011.
%%% 5
%\item \textit{How to probe anti-neutrino = neutrino and the absolute neutrino mass scale}, International Neutrino Summer School, Geneva (Switzerland), 2011.
%%% 6
%\item \textit{Sense and sensitivity in \bbonu\ experiments} at XIV International Workshop on Neutrino Telescopes, Venice (Italy), 2010.
%%% 7
%\item {\it Ettore Majorana meets his shadow (searching for neutrino less double beta decay)}, Wednesday colloquium, Weizmann institute, 24 November 2010.
%%% 8
%\item \textit{The physics case of the Neutrino Factory}, at 23rd International Conference on Neutrino Physics and Astrophysics (Neutrino 2008), Christchurch (New Zealand).
%%% 9
%\item {\it Lectures on Neutrino Physics}, CERN Summer Student Programme, Geneva (Switzerland), 2004--2009.
%%% 10
%\item \textit{Measuring leptonic CP violation in future neutrino facilities} at the school \textit{CP violation: From quarks to leptons}, Varenna (Italy), 2005.
%\end{enumerate}
%
%\subsubsection*{Organization of international conferences}
%\begin{enumerate}
%%% 1
%%% 1
%\item \textit{Neutrino 2014}, Boston, 2014. \emph{Member of the International Advisory Committee.}
%%% 1
%\item \textit{Weak Interactions and Neutrinos, WIN 2013}, Natal, Brazil, September 2013. \emph{Member of the International Advisory Committee.}
%
%\item \textit{1st Workshop on Xenon-based Detector}, LBNL, Berkeley, Nov 2009. \emph{Member of the International Advisory Committee.}
%%% 2
%\item \textit{10th International Workshop on Neutrino Factories, Super-beams and Beta-beams (NuFact 08)}, IFIC, Valencia, July 2008. \emph{Co-chair.}
%%% 3
%\item \textit{International Workshop on the Golden Channel at a Neutrino Factory}, IFIC, Valencia, June 2007. \emph{Chair}.
%\end{enumerate}

