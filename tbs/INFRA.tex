\subsection{Costs of the NEXT infrastructures}
The operation of NEXT at the LSC requires extensive infrastructures. In addition to the xenon gas, owned by the LSC (100 kg enriched and 100 kg natural), the laboratory has built the working platform, seismic pedestal and lead castle needed to host the experiment. The NEXT experiment provides the gas system needed to recirculate and clean the gas. Such system is expensive, given the safety requirements, and has been purchased with AdG funds. The AdG also provides funds to buy the clean tent, radon suppression and monitoring system and miscellaneous expenses for a total of some
437 k\euro. Importantly, the infrastructures are fully funded with the contributions of the LSC, CUP and AdG. 
Table \ref{tab.n100:INFRA} summarises the total costs of the infrastructures. The costs detailed in the tables are very accurate, since most of the components have already been acquired. 


  
\begin{table}[h!]
\begin{center}
\begin{tabular}{|l|c|c|c|c|}
\hline
 Cost &	Total& 	CUP & AdG &  LSC \\
 \hline
Gas System &	434,177 &	 68,970 &	365,207 &	0 \\
Platform and Castle	& 250,600 & 	0	&0 &	250,600 \\
Xenon (100 kg + 100 kg)	& 1,056,000	& 0 &	0 & 1,056,000 \\
Lead cleaning	& 44,581 &44,581 &	0 & 0 \\
Cleaning equipment	& 18,150	& 18,150	& 0	& 0	\\
Clean tent	 & 59,878	&	0& 59,878 &	0	\\
Radon suppression 	& 5,400 &	0 &	5,400 &	0	\\
Radon monitoring &	6,700 &	0	& 6,700	& 0	\\
 \hline
{\bf Total Infrastructures} &	{\bf1,875,486}& 1131,701 & 437,185 & 1,306,600 \\	
 \hline\hline
\end{tabular}  
\caption{Costs of the NEXT infrastructures at the LSC.}
\label{tab.n100:INFRA}
\end{center}
\end{table} 

