\subsection{Costs of personnel for NEXT construction}

The construction, commissioning and operation of the NEXT detectors require of a team of specialised physicists and engineers. This team comes from both national and international universities and research institutions. The contributions of the international collaboration, in particular of the USA groups during the period of R\&D and design of NEXT have been very important for the development of the project. Currently, the spanish groups, in particular those participating in this co-ordinated project have absorbed the know-how brought to the collaboration by the crucial contributions of the Berkeley group (prof. David Nygren, the inventor of the TPC technology) and Texas group (the late prof. James White, who was the leading World expert in high pressure gas chambers). 

The CUP grant have made possible the creation of a world class group at IFIC, which includes the PI, the technical coordinator (Dr. Igor Liubarsky, a renewed expert in the field), two R\&C fellows, one of them senior (Dr. Sorel) and one of them junior (Dr. Novella, who starts this year in the group), six post-docs  (Laing, Ferrario, L\'opez-March,  Renner, Martin-Albo and Monrabal) and 4 Ph.D. students, 2 of whom will present their Ph.D. thesis in 2014 or early 2015. Last but not least, the group has formed several engineers. S. C\'arcel is leading the development of mechanics (with the help of technical mechanics engineer A. Mart\'inez) and J. Rodr\'iguez leads the development of electronics (with the help of technical electronics engineer V. Alvarez).

The group at the UPV brings the essential expertise in front-end electronics and data acquisition. The group includes three experienced engineers, all of the professors at the UPV. Last, but not least, prof. J.A. Hernando, form the University of Santiago has taken the important role of calibration and reconstruction coordinator in NEXT. 

We require co-funding to keep essential personnel for the project. A substantial contribution to personnel at IFIC will come from the  AdG grant, which will provide funds for amount of 1,097,258 \euro\ over the period requested for this grant. This will cover the salary of the technical coordinator (Liubarsky) and 4 post-docs. We expect to obtain 4 post-doc years from national and international grants, in particular from the Marie Curie program (these are 2 years positions). Consequently, the IFIC group requires the equivalent to one full post-doc per for years to this grant.  

IFIC also requests funding to keep our 2 senior engineers (C\'arcel and Rodr\'iguez), who are in charge of essential parts of the project and one technical engineer (external funds, from local agencies, such as the Generalitat Valenciana will be seek to fund the second technical engineer in the team). 

The UPV is in charge of the full development of the electronics and brings in essential man power (with permanent positions). The personnel needed is: a technical engineer to help with the development of the front-end electronics, lead by the electronics coordinator of NEXT (prof J. Toledo), and a senior enginner/computer scientist to help with the dual task of DAQ development (task lead for the DAQ coordinator of NEXT, professor R. Esteve) and online computing.  

Last but not least, we request a post-doc to reinforce the group at the University of Santiago. The group is lead by prof. J.A. Hernando, a renown physicist who has made major contributions to neutrino physics and to flavour physics. Hernando is now full time in NEXT and has taken the role of calibration and reconstruction coordinator. A post-doc to help in these tasks, essential for the performance of NEXT is requested. Santiago will seek for local support and international fellowships for a second post-doc.

The NEXT project is extremely well suited as a training ground for students and post-docs. The project involves the construction, commissioning, operation and data analysis of the most advanced HPXe in the World, and the possibility to participate in a major discovery. The teams are very experienced and well organised. At IFIC, four senior physicist (the PI, Dr. Liubarsky, Dr. Sorel and Dr. Novella) are looking forward to advise students. At UPV, students can work with two of the leading experts in front-end electronics and DAQ in the field. At the US, prof. Hernando is already working with two students in calibration and reconstruction.

At the same time, graduate students are very important for the future of the project and for its impact in science and society. Consequently, the groups in this coordinated project require 3 FPI grants, one per group (we will seek for other grants, such as FPU to enrol further graduate students). 

Table \ref{tab.P} summaries the personnel requested. Table \ref{tab.new:PT} details the standard salaries
payed to post-docs and engineers. Finally, table \ref{tab.new:PC} describes the personnel costs required to this project.

\begin{table}[h!]
\begin{center}
\begin{tabular}{|l|c|c|c|c|}
\hline
Group &	post-docs	& engineers &	technical engineers & FPI\\
 \hline
IFIC &	1 &	2	&1 &	1\\			
UPV	  & 0	&1 &	1 &	1 \\			
US	& 1 &	0 &	0 &	 1\\			
 \hline
{\bf Total} & 2 & 3 & 2 & 3 \\
 \hline\hline
\end{tabular}  
\caption{Personnel requested.}
\label{tab.P}
\end{center}
\end{table} 

\begin{table}[h!]
\begin{center}
\begin{tabular}{|l|c|c|c|}
\hline
Cost &	post-docs	& engineers &	technical engineers \\
 \hline
 &	40,000 &	40,000	&30,000 \\					
 \hline\hline
\end{tabular}  
\caption{Table of costs.}
\label{tab.new:PT}
\end{center}
\end{table} 

\begin{table}[h!]
\begin{center}
\begin{tabular}{|l|c|c|c|c|c|}
\hline
Group &	post-docs	& engineers &	technical engineers &  Total \\
 \hline
IFIC	&160,000 &	320,000 &	120,000 & 600,000 \\
UPV	 &	0 & 160,000 &	30,00 &	190,000 \\
US	& 160,000 & 0 & 0 &	160,000\\
\hline
{\bf Total} & 320,000 & 480,000 & 150,000 & {\bf 950,000} \\
 \hline\hline
\end{tabular}  
\caption{Personnel costs.}
\label{tab.new:PC}
\end{center}
\end{table} 

Notice that the total costs requested in this project are slightly below those provided by external fund sources such as the AdG. 

