\paragraph{Existing equipment and infrastructures.}

In order to carry out this project, the COORD, ENG and CALREC groups will make use of the following equipment and infrastructures:

\begin{enumerate}
\item {\bf A state-of-the-art laboratory at IFIC}, developed and financed with funds from CUP (see Sec.~\ref{subsec:Grants}). The DEMO detector is operating in this laboratory. The laboratory includes a full gas system, infrastructures for vacuum and high pressure operation, high voltage and slow control systems, and a full DAQ plus computing system. 
\item {\bf A state-of-the-art laboratory electronics laboratory at IFIC}, which has made possible the very fast development of the tracking plane. 
\item {\bf A state-of-the-art electronics laboratory at UPV}, which has made possible the very fast development of the FEE and DAQ. Of particular interest is the collaboration of UPV with CERN, in the context of the RD-51 effort. This collaboration is one of the main reasons why NEXT has been declared a CERN recognised experiment.
\item {\bf Computing resources at US}, where there is a large cluster (Tier 2 class) for distributed LHCb analysis, with about 1.500 processor cores. The NEXT experiment, through the PI of the CALREC subproject, will have access to this cluster for Monte Carlo simulation and data reconstruction. 
\item {\bf Experimental laboratory at US}, which includes a clean room (30 m$^2$ class 100.000) with an automatic ultrasonic wedge bonding machine. This laboratory is ideal to develop the gas-additive program to be carried out by US. 
\item {\bf Infrastructures at LSC}, which include: a) working platform and seismic pedestal, b) lead castle, c) gas system, d) clean tent, and e) radon suppression system. In addition, the LSC provides general support for the experiments. 
\item {\bf A state-of-the-art facility for radio purity measurements}, located at the LSC. 
\end{enumerate}

On the other hand, for the successful development of the BATA subproject, CLPU will provide the required human and technological resources. CLPU is the centre of reference in Spain regarding laser technology, and takes active part in several international and national projects.  Moreover, {\bf CLPU considers this project as high priority and consequently will offer the collaboration of the entire scientific department}, consisting of a multidisciplinary team with broad experience in laser technology and development, and laser-matter interaction. The CLPU will provide a substantial part of the required infrastructures and equipment for the \BATA\ subproject, including:
 
 \begin{itemize}
    \item \textbf{Laboratory room for all the experiments}. Experiments will be performed at the CLPU laboratories.
    
\item \textbf{Workshop service}. The CLPU has a mechanical workshop capable of manufacturing high level designs. 

\item \textbf{Electronics workshop}. This workshop offers general electronics service. 

\item \textbf{Auxiliary equipment}. KHz amplified Ti:Saphire laser (Spitfire, Spectraphysics, 7 mJ 100 fs), laser micro-machining station, Optical and Scanning electron microscopes (ZEISS), plist DAQ services, vacuum equipment and optical equipment.   

%\item \textbf{Data acquisition equipment:} Three oscilloscopes up to 1 GHZ (Tektronix), data acquisition computer, 3 GHZ RF spectrum analyzer (Tektronix). 
%
%\item \textbf{Vacuum equipment:} Primary pumps, turbo-molecular pumps, general vacuum hardware.
%
%\item \textbf{Temperature control equipment:} Thermoelectrical chiller, 
%
%\item \textbf{Optical equipment (general, all for vis/nir):} Laser spectrum analyser (500-1500 nm), Laser spectrum analyser (1000-2600 nm), photodiodes (Ge, Si), ultra-fast photodiodes (300 ps), CCD cameras, two Laser Beam profilers (Gentec), optical autocorrelator (up to 50 fs, sweep), home-made optical autocorrelator (single shot), alignment lasers (HeNe), shutters, chopper (Stanford), filters, lenses, mirrors, general optomechanics (posts, post holders, clamping forks, waveplates, diaphragms, etc), two infrared viewers (up to 1350 nm), optics and optomechanics (mirrors, lenses, LBO crystals for 800 and 1030 nm, posts, clamping forks, mounts, etc.)

\item \textbf{Lasers}. A 493.54 nm CW LASER: Pump laser (Coherent, Verdi G 20 W, 532 nm, CW); a tunable Ti:Saphire laser CW (750-1020 nm, more than 4 W).
%
%\item \textbf{Spectrometer (UV-VIS-NIR, MIGHTEX)}
%
%\item \textbf{Power and energy meters:} thermal and pyroelectrical. 

\end{itemize}


