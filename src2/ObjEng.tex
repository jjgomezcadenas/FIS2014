\paragraph{Objectives of the ENG subproject.}

The ENG subproject centralises the front-end electronics, data acquisition (DAQ), online system and slow controls of the NEW and NEXT-100 detectors. It is coordinated by the UPV. NEXT (via the UPV team) has co-developed a new readout and DAQ concept named SRS\footcite{Toledo2011,SRS2013} for the international RD-51 collaboration at CERN. NEXT front-end modules are connected via copper links to the SRS DAQ interface modules. The CERN standard DATE environment is used as DAQ software. This brings a number of advantages, such as counting on a large base of users and developers, reducing production costs and profiting from other groups' developments. SRS has been successfully used in NEXT-DEMO (PMT and SiPM readout, DAQ interface and trigger modules)\footcite{Gil2012,Herrero2012,Esteve2012} and newer versions of these modules are to be used in NEW and NEXT-100\footcite{TWEPP2014}.

The specific objectives of this subproject are:

\begin{enumerate}

\item {\bf FEE (Front End Electronics)}. Design, fabrication and commissioning of the front-end electronics for the PMTs and the SiPMs for NEW and NEXT-100. The co-PI responsible for this objective is J. Toledo.
 
\item {\bf DAQ}. Design, fabrication and commissioning of the data acquisition modules for NEW and NEXT-100. The co-PI responsible for this objective is R. Esteve. 

\item {\bf Slow control}. Design, fabrication and commissioning of the slow control for NEW and NEXT-100. The goals are (1) monitor critical detector parameters, mostly temperature and pressure, (2) control power supplies for the sensors, detector grids and electronics and (3) implement an automatic emergency response monitor. The project leader is technical engineer V. Álvarez, under the supervision of co-PI J. Toledo.

\item {\bf Online}. Design and commissioning of the online system for NEW and NEXT-100. The co-PI responsible for this objective is R. Esteve.

\end{enumerate}