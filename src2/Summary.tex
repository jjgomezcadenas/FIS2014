%%

NEXT (Neutrino Experiment with a Xenon TPC) is an experiment to search for the neutrinoless double beta decay process (\bbonu). The detection of such a process would demonstrate that neutrinos are Majorana particles (that is, their own antiparticles) and would have profound consequences in physics and cosmology.  

The isotope chosen by NEXT is  \XE. The collaboration has access to one hundred kilograms of xenon gas enriched at 90\% in \XE, owned by the Underground Laboratory of Canfranc (LSC). The NEXT technology is based on the use of time projection chambers operating at a typical pressure of 15 bar (\HPXE). The main advantages of the experimental technique are: a) excellent energy resolution; b) the ability to reconstruct the trajectory of the two electrons emitted in the decays, further contributing to the suppression of backgrounds; c) scalability to large masses; and d) the possibility to reduce the background to negligible levels thanks to the barium tagging technique (\BATA).

The NEXT road map was designed in four stages: i) Demonstration of the \HPXE\ technology with prototypes deploying a mass of natural xenon in the range of 1 kg; ii) Characterisation of the backgrounds to the \bbonu\ signal and measurement of the \bbtnu\ signal with the NEW detector, deploying 12 kg of enriched xenon and operating at the LSC; iii) Search for \bbonu\ decays with the NEXT-100 detector, which scales up the NEW detector by a factor 2:1 in size (8:1 in mass) and deploys, thus, 100 kg of enriched xenon. iv) Search for \bbonu\ decays with the BEXT detector (Barium-tagging Experiment with a Xenon TPC), which will deploy a mass in the ton scale and will introduce the \BATA\ technique in order to reduce backgrounds to negligible levels.  

The first stage of NEXT has been successfully completed during the period 2009-2013. The prototypes NEXT-DEMO (IFIC) and NEXT-DBDM (Berkeley) were built and demonstrated the main features of this technology. The experiment is currently developing its second phase. The NEW detector is being constructed during 2014 and will operate in the LSC during 2015. The funding for the construction and operation of NEW comes from an ERC Advanced Grant (AdG/ERC) granted to the co-coordinator of this project in 2013 (project duration: February 2014 - January 2019). The NEXT-100 detector is the third phase of the experiment. It will be built and commissioned during 2016 and 2017, and will start data taking in 2018. The NEXT-100 detector already has a significant discovery potential. Its findings will influence the fourth phase of the experiment (BEXT), which could start in 2020. 

NEXT is an international collaboration, led by Spanish groups (G\'omez-Cadenas, co-coordinator of this proposal, is the spokesperson of the collaboration) and with a very significant contribution of US groups. The laser technology needed for the BEXT phase is being developed in collaboration with the Spanish Pulsed Laser Center (CLPU). 

This proposal requires {\em co-funding} to complete the phase three of the experiment. Specifically we request: a) funds to co-finance the construction of the NEXT-100 detector (which is being partially funded by the AdG as well as by the international collaboration, primarily US groups); b) funds to co-finance personnel; and c) a  contribution to the R\&D to develop the \BATA\ technology.   

