The NEXT experiment is organised as an international collaboration, which includes groups from Spain, Portugal, Russia, USA, and Colombia. The Spanish groups participating in NEXT are: Instituto de Física Corpuscular (IFIC), a joint center of the University of Valencia (UV) and the Spanish National Research Council (CSIC), Polytechnic University of Valencia (UPV), University of Santiago de Compostela (US), Autonomic University of Madrid (UAM), and University of Zaragoza (UZ). 

The groups participating in this coordinated project form the core of the collaboration. The spokesperson (and co-coordinator of this project), the analysis co-coordinator (and co-coordinator of this project), the technical coordinator, the reconstruction co-coordinator and the leaders of the detector construction, are from IFIC. The coordinators of the electronics, DAQ, and slow controls are members of the UPV. The coordinator of calibration is the PI of the US group. The groups participating in this coordinated project invest 100\% of their research time and resources in the NEXT project. 

Furthermore, a strong collaboration has been  formed between NEXT and the Spanish Pulsed Laser Center (CLPU after the initials of the center in Spanish), to develop the laser technology which could be used to tag the barium ion emitted in the \bb\ decays, resulting (when combined with the excellent energy resolution of NEXT and its topological signature) in a virtually background-free experiment. The ambitious R\&D program that could produce a viable scheme for barium tagging (BATA) is also described in this proposal.  

A key task for the experiment is that of radio purity, coordinated by the UAM (prof. Luis Labarga) who also presents a research project to this call. The project of prof. Labarga includes a proposal to participate in the Super Kamiokande experiment, and for this reason we have considered more appropriated to present separated proposals. The task of radio purity and the corresponding request for resources is described in his project.
%
%Two additional independent projects (in the modality of young researchers) are presented in parallel with this coordinated project. The PI of one of them, Dr. Ferrario, is the coordinator of Monte Carlo and Reconstruction, and her project includes an innovative collaboration with the Fermi National Laboratory (Fermilab) in USA, to handle massive production of the Monte Carlo data needed for NEXT. The PI of the second project, Dr. Laing is the run coordinator of DEMO and will serve as run coordinator of NEW and NEXT-100. His project includes an innovative idea to upgrade the energy plane of the DEMO detector using last-generation SiPMs.


