The COORD subproject is lead by the IFIC, the largest group in NEXT. The IFIC group includes: The PI (a CSIC professor, and spokesperson of NEXT). The analysis co-coordinator (Dr. Sorel, R\&C). Dr. P. Novella (R\&C); the technical coordinator (Dr. I. Liubarsky); the run coordinator and collaboration expert in coating techniques (Dr. N. Yahlali), the project leader (PL) of the construction of the energy plane (Dr. Laing); the PL of tracking plane (senior engineer J. Rodriguez); the PL of field cage (Dr. March); the PL of mechanics (pressure vessel, infrastructures), senior engineer S. Carcel. The run coordinator (Dr. Laing); the reconstruction and Monte Carlo coordinator (Dr. Ferrario) and 6 graduate students. In addition the group has two technical electronics engineering developing essential components of the tracking and energy plane, and one technical mechanical engineering, with expertise in mechanical design. 

The PI and the two R\&C positions are funded by CSIC. The positions of one senior physicist (Liubarsky), two post-docs (Renner, March), and two of the three technical engineers positions, will be funded by the AdG. Dr. Laing and Dr. Ferrario are applying to independent grants (young researcher modality). 

COORD requests funding for three {\bf essential} positions. Coating expert Dr. Yahlali (4 years). Senior electronics engineer J. Rodríguez, the PL of the NEXT tracking plane construction for NEW and NEXT-100, developer of the NEXT KDBs, and responsible of the NEXT-IFIC electronics laboratory (4 years). And senior mechanical engineer S. Cárcel, the PL of the pressure vessel and NEXT infrastructures and integration coordinator (4 years). In addition we request funding for 3 years for one electronics technical engineer.  

For the ENG subproject we request two positions.

A technical electronics engineer (3 years), whose tasks will be: a) carry out installation, maintenance, repairing and support of the front-end and DAQ interface electronics for NEW and NEXT-100, as well as the power supplies for sensors and electronics; b) will also take part in the design and construction of small circuits (like filters or cabling); c) will help in the commissioning phases and in the coordination of the electronic modules purchases and production and will carry out functional tests of the new units. 

A computer scientist/engineer, expert in Linux systems administration and Ethernet computer networks (4 years). Will configure, administer, maintain and support the different PC clusters (mostly running CERN Scientific Linux) in the DAQ, Online and Offline systems. Will configure the DATE environment for the DAQ system. Will scale these systems according to specific needs (like the upgrade from NEW to NEXT-100). Will manage the data storage. Will carry out R\&D programs to enhance performance. Construction and commissioning. 

For the CALREC subproject we request one post-doc position (4 years) and one technical engineer position (3 years). The post-doc will assist the PI in the essential tasks of setting up the calibration system with sources and with krypton; carrying out the calibration runs for the sensors, energy and position, including data taking, analysis, and maintenance of calibration data base. Develop energy and tracking reconstruction algorithms, using calibration data to assess and improve the performance of the detector. In addition, the post-doc will assist the PI in the R\&D program for gas additives. The role of the technical engineer is to help in the mechanical design of the calibration sources, the modifications to the NEXT gas system needed to introduce the rubidium source, and the setting up of the gas additives laboratory. 

The CLPU team includes the participation of Dr. V\'azquez  (PI of the BaTa subproject,  expert in laser-matter interaction). Dr. Peralta, head of the Scientific Division of CLPU, and expert in laser-matter interaction. Dr. Rico, the CLPU expert in the
design and construction of laser sources, Dr. Api\~naniz, an expert in laser-plasma interactions, and Dr. S\'anchez, an expert in laser-matter interaction in the femtosecond
timescale.

For the BATA subproject we request one post-doc position (4 years) to assist the PI and the rest of the CLPU team in the experimental program (setting up the experimental lab, tuning up the blue laser, assist in the experiments, analysis of data, and participate in the development of the IR laser), and one technical electronics engineer (3 years) to assist in the electronics tasks associated to the development of the IR laser.  

