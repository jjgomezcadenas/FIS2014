\paragraph{COORD subproject.}

The COORD subproject is led by IFIC, the largest group in NEXT. The IFIC group includes: the co-coordinators of this project (Prof.~J.J. G\'omez-Cadenas, CSIC professor and spokesperson of NEXT, and Dr.~M. Sorel, CSIC RyC researcher and NEXT analysis co-coordinator), Dr.~P. Novella (CSIC RyC researcher), the technical coordinator (Dr. I. Liubarsky), the run coordinator and collaboration expert in coating techniques (Dr. N. Yahlali), the project leader (PL) for the energy plane construction (Dr. Laing), the tracking plane PL (senior engineer J. Rodriguez), the field cage PL (Dr. March), the PL of mechanics related to pressure vessel and infrastructures (senior engineer S. Carcel), the run coordinator (Dr. Laing), the reconstruction and Monte Carlo coordinator (Dr. Ferrario), and 6 graduate students. In addition the group has two technical electronics engineers developing essential components of the tracking and energy plane, and one technical mechanical engineer with expertise in mechanical design. 

G\'omez-Cadenas, Sorel and Novella are funded by CSIC. The positions of one senior physicist (Liubarsky), two post-docs (Renner, March), and two of the three technical engineers positions will be funded by the AdG. Laing and Ferrario are applying to independent grants (young researcher modality). 

COORD requests funding for three {\bf essential} positions. First, coating expert Dr. Yahlali (4 years). Second, senior electronics engineer J. Rodríguez, the PL of the NEXT tracking plane construction for NEW and NEXT-100, developer of the NEXT KDBs, and responsible of the NEXT-IFIC electronics laboratory (4 years). Third, senior mechanical engineer S. Cárcel, the PL of the pressure vessel and NEXT infrastructures and integration coordinator (4 years). In addition, we request funding for 3 years for one electronics technical engineer.  

\paragraph{ENG subproject.}

For the ENG subproject we request two positions. The first position is for a technical electronics engineer (3 years), whose tasks will be: a) to carry out installation, maintenance, repairs and support of the front-end and DAQ interface electronics for NEW and NEXT-100, as well as the power supplies for sensors and electronics; b) to take part in the design and construction of small circuits (filters, cabling); c) to contribute to the commissioning phases, to the coordination of the electronic modules purchases and production, and to carry out functional tests of the new units. 

The second position is for a computer scientist/engineer, expert in Linux systems administration and Ethernet computer networks (4 years). He/she will: a) configure, administer, maintain and support the different PC clusters (mostly running CERN Scientific Linux) in the DAQ, online and offline systems; b) configure the DATE environment for the DAQ system; c) scale these systems according to specific needs (such as the upgrade from NEW to NEXT-100); d) manage the data storage; e) carry out R\&D programs to enhance the computing system performance; f) contribute to detector construction and commissioning. 

\paragraph{CALREC subproject.}

For the CALREC subproject we request one post-doc position (4 years) and one technical engineer position (3 years). The post-doc will assist the PI in the essential tasks of setting up the calibration system with sources and with krypton. He/she will also: a) carry out the calibration runs for the sensors, energy and position (including data taking, analysis, and maintenance of calibration data base); b) develop energy and tracking reconstruction algorithms using calibration data to assess and improve the performance of the detector; c) assist the PI in the R\&D program for gas additives. The role of the technical engineer is to contribute to: a) the mechanical design of the calibration sources; b) the modifications to the NEXT gas system needed to introduce the rubidium source; c) the creation of the gas additives laboratory. 

\paragraph{BATA subproject.}

The CLPU team includes the participation of Dr. V\'azquez (PI of the \BATA\ subproject, expert in laser-matter interaction), Dr. Peralta (head of the Scientific Division of CLPU and expert in laser-matter interaction), Dr. Rico (CLPU expert in the design and construction of laser sources), Dr. Api\~naniz (expert in laser-plasma interactions), and Dr. S\'anchez (expert in laser-matter interaction at the femtosecond timescale).

For the \BATA\ subproject we request one post-doc position (4 years) to assist the PI and the rest of the CLPU team in the experimental program, namely to set up the experimental laboratory, to tune the blue laser, to assist in the experiments and in data analysis, and to participate in the development of the IR laser. We also request one technical electronics engineer (3 years) to assist in the electronics tasks associated to the development of the IR laser.  

