\paragraph{COORD subproject.}

The COORD subproject is led by IFIC, the largest group in NEXT. The IFIC group includes: the co-coordinators of this project (Prof.~J.J. G\'omez-Cadenas, CSIC professor and spokesperson of NEXT, and Dr.~M. Sorel, CSIC RyC researcher and NEXT analysis co-coordinator), Dr.~P. Novella (CSIC RyC researcher), the technical coordinator (Dr. I. Liubarsky), the run coordinator and project leader (PL) for the energy plane construction (Dr. Laing), the tracking plane PL (senior engineer J. Rodriguez), the field cage PL (Dr. March), the PL of mechanics related to pressure vessel and infrastructures (senior engineer S. Carcel), the run coordinator (Dr. Laing), the reconstruction and Monte Carlo coordinator (Dr. Ferrario), and 6 graduate students. In addition the group has two technical electronics engineers developing essential components of the tracking and energy plane, and one technical mechanical engineer with expertise in mechanical design. 

G\'omez-Cadenas, Sorel and Novella are funded by CSIC. The positions of one senior physicist (Liubarsky), two post-docs (Renner, March), and two of the three technical engineers positions will be funded by the AdG. Laing and Ferrario are applying to independent grants (young researcher modality). 

COORD requests funding for three {\bf essential} engineering positions. 

{\bf One senior electronics engineer}, a position currently occupied by J. Rodríguez, the PL of the NEXT tracking plane construction for NEW and NEXT-100, developer of the NEXT KDBs, and responsible of the NEXT-IFIC electronics laboratory (4 years). The responsibilities of the tracking plane leader (TPL) include: design, testing and certification of the KDBs making up the tracking plane; design, testing and certification of the tracking plane feedthroughs; coordination of the production and quality control of the KDBs; design, testing and production of the inner and outer cables of the tracking plane and the connector boards; commissioning of the full system; link with the UPV personnel for testing and commissioning of the front-end electronics. The TPL will also investigate possible upgrades of the tracking plane components, which can include improvements in the SiPMs, KDBs and cabling. The TPL will also be in charge of the NEXT-IFIC electronics laboratory. 

{\bf One senior mechanical engineer}, a position currently occupied by S. Cárcel, the PL of the pressure vessel and NEXT infrastructures and integration coordinator (4 years). The mechanical engineering tasks include: construction, commissioning and operation (CCO) of the pressure vessels for NEW and NEXT-100; CCO of the internal copper shield (ICS); CCO of remaining infrastructures at LSC, in particular the emergency recovery system and the radon control systems. The engineer will also play the role of integration manager, and will be in charge of the transport, cleaning (radiopure standards) and integration procedures for each NEXT subsystem.

Finally, the development of the electronic components of the tracking and energy planes requires {\bf one technical technical electronics engineer} (3 years) to assist the tracking plane leader and the electronics leader. The tasks of the technical engineer are: design, testing and production of the ultra-radiopure, low noise PMT bases; testing and certification of all PMT sensors in the energy plane. The engineer will also assist with the heavy work load related with the production and testing of the KDBs, and with the assembly and commissioning of the full energy and tracking plane systems.

\paragraph{ENG subproject.}

For the ENG subproject we request two positions. The first position is for a {\bf technical electronics engineer} (3 years), whose tasks will be: a) to carry out installation, maintenance, repairs and support of the front-end and DAQ interface electronics for NEW and NEXT-100, as well as the power supplies for sensors and electronics; b) to take part in the design and construction of small circuits (filters, cabling); c) to contribute to the commissioning phases, to the coordination of the electronic modules purchases and production, and to carry out functional tests of the new units. 

The second position is for a {\bf computer scientist/engineer}, expert in Linux systems administration and Ethernet computer networks (4 years). He/she will: a) configure, administer, maintain and support the different PC clusters (mostly running CERN Scientific Linux) in the DAQ, online and offline systems; b) configure the DATE environment for the DAQ system; c) scale these systems according to specific needs (such as the upgrade from NEW to NEXT-100); d) manage the data storage; e) carry out R\&D programs to enhance the computing system performance; f) contribute to detector construction and commissioning. 

\paragraph{CALREC subproject.}

For the CALREC subproject we request {\bf one post-doc position}. The post-doc will assist the PI in the essential tasks of setting up the calibration system with sources and with krypton. He/she will also: a) carry out the calibration runs for the sensors, energy and position (including data taking, analysis, and maintenance of calibration data base); b) develop energy and tracking reconstruction algorithms using calibration data to assess and improve the performance of the detector.

\paragraph{BATA subproject.}

The CLPU team includes the participation of Dr. V\'azquez (PI of the \BATA\ subproject, expert in laser-matter interaction), Dr. Peralta (head of the Scientific Division of CLPU and expert in laser-matter interaction), Dr. Rico (CLPU expert in the design and construction of laser sources), Dr. Api\~naniz (expert in laser-plasma interactions), and Dr. S\'anchez (expert in laser-matter interaction at the femtosecond timescale).

For the BATA subproject we request {\bf one post-doc position} (4 years) to assist the PI and the rest of the CLPU team in the experimental program, namely to set up the experimental laboratory, to tune the blue laser, to assist in the experiments and in data analysis, and to participate in the development of the IR laser. We also request {\bf one technical electronics engineer} (3 years) to assist in the electronics tasks associated to the development of the IR laser.  

