\paragraph{Objectives of the COORD subproject.}
The specific objectives of the COORD subproject are:

\begin{enumerate}
\item {\bf Construction of NEW}, foreseen to be completed in Q2'15, and involving the construction of the NEW pressure vessel, field cage, energy plane, and tracking plane. The co-PI responsible for this objective is J.J. G\'omez-Cadenas.

\item {\bf Commissioning of NEW and evaluation of performance}. The NEW detector will be brought online in Q2'15, and extensive system testing will be performed to certify safe and stable operation (no leaks, no sparks), as well as testing and integration of all the subsystems. We expect to complete commissioning in Q3'15. During Q4'15, we will evaluate the performance of the detector. Such evaluation will allow us to correct for design problems (if they arise) or to introduce improvements in the engineering if needed. We will also assess the overall radioactive budget of the detector, to ensure the absence of ``hot spots'' (excess of radioactivity introduced accidentally in the detector). The co-PI responsible for this objective is J.J. G\'omez-Cadenas.

\item {\bf NEW physics run}. During 2016, we will operate continuously the NEW detector at the LSC. The physics run of NEW has several goals: a) measurement, using radioactive sources, of the energy resolution as a function of the energy, and in particular at \Qbb (CALREC subproject); b) measurement, using radioactive sources, of single electrons mimicking background events, as well as double electrons produced by the double escape peak of Tl-208 and used to characterise the signal (COORD and CALREC subprojects); c) measurement of the standard decay \bbtnu\ of \XE\ (collaboration-wide studies); and d) a full measurement of the energy spectrum to validate, from the data themselves, the background model (collaboration-wide studies). The co-PI responsible for this objective is M. Sorel.
%

\item {\bf Construction of NEXT-100}. The fabrication of NEXT-100 will proceed through 2016, although some parts (such as the pressure vessel) have already been built. The construction will take 12 months. The co-PI responsible for this objective is J.J. G\'omez-Cadenas. 

\item {\bf Commissioning of NEXT-100}. The commissioning of NEXT-100 will benefit from the experience gained commissioning and operating NEW. We consider feasible to commission the detector during Q1'17 and Q2'17, but our project management plan allows for two extra quarters. The main reason is to guarantee enough time to run with normal xenon before circulating the precious enriched xenon in the gas system and the detector. Notice that the detector can be fully calibrated, and the backgrounds can be characterised with normal xenon. The co-PI responsible for this objective is J.J. G\'omez-Cadenas. 

\item {\bf Physics run of NEXT-100}. The physics run to search for the \bbonu\ decay of \XE\ may start in Q3'17, but the project plan foresees it for Q1'18. The calibration procedures are identical to those developed for NEW. After one year of run, NEXT-100 should reach the sensitivity of the current leading experiments. We currently foresee to run for three years (2018 to 2020), achieving a sensitivity to \mbb\ of about 70-190~meV (see Sec.~\ref{subsubsec:stateoftheart}). The co-PI responsible for this objective is M. Sorel.

\end{enumerate}
