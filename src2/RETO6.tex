%Si la memoria se presenta a la convocatoria de RETOS INVESTIGACIÓN, deberá identificarse el reto cuyo estudio se pretende abordar y la relevancia social o económica prevista.
%
This research project is presented within the program of ``Challenges of society'', specifically, challenge number 6: {\bf Change and social innovation.}

We argue that this project represents a major innovation in the way that particle physics is conducted in Spain, and thus marks a path to a more productive approach to research.

Particle physics is a clear example of the so-called ``big-science''. The discovery of the Higgs boson is a quintessential example of such big science. It has required the construction and operation of the LHC, one the most impressive scientific machines ever built by humankind. The gargantuan scale of the effort could only be met by a collective effort centralised at CERN, the largest particle physics laboratory in the World.  

Big science involves big budgets, often invested in purchasing equipment to be installed at CERN and in paying scientific staff whose activity also develops at CERN. Such large budgets are often justified in terms of industrial and scientific returns. While those returns certainly exist, {\bf they tend to be larger for countries who are already well developed scientifically}. Specifically, the positions of leadership in the large CERN experiments, and in the CERN scientific and technical divisions, are dominated by countries like Germany, Switzerland, U.K., France and Italy. 

Remarkably, the countries leading the big science at CERN and other laboratories have also developed ``national science'' physics programs. A case of great interest is Italy, a country not very different from Spain, in terms of GDP and social habits. However, the international impact and the returns of physics in Italy is much larger than in Spain. For example, the number of spanish staff members at CERN is 115, to be compared with 275 corresponding to Italy (which has the second largest staff population, after France, who co-hosts the lab). Adding fellows and associates (that is, temporary CERN contracts, often given to scientists), the figures for Spain are 363, to be compared with 1726 for Italy\footnote{\href{http://council.web.cern.ch/council/en/Governance/TREF-PersonnelStatistics2012.pdf}{http://council.web.cern.ch/council/en/Governance/TREF-PersonnelStatistics2012.pdf}}. Several Italians have served as CERN general directors, and have led or are leading the LHC experiments. The next CERN general director (and perhaps the first woman to occupy such position in the history of the lab) may be the ex-spokesperson of ATLAS, the italian physicist Fabiola Gianotti. Moreover, Italy has four Nobel prizes in physics (Marconi, 1909, Fermi, 1938, Segrè 1959, Rubbia 1984), while Spain has none. 

Remarkably Italy also boasts the best underground laboratory of Europe, and one of the best of the world, the LNGS. The lab hosts 20 experiments including three searching for \bbonu\ processes (GERDA, CUORE and COBRA) and two experiments searching for Dark Matter (WARP and XENON). 

Through these experiments, the italian physics{\bf attracts external talent} (some of the best physicists from Europe and USA participate in experiments at LNGS) {\bf and external funding}, complementing the big science at CERN with physics of a smaller scale concerning human resources and budgets. However, such ``local'' physics results in discoveries of great scientific impact (such as the discovery of neutrino oscillations, which has been the result of a world-wide effort involving underground laboratories in Italy, USA, Canada, Russia and Japan). It also allows the training of students and post-docs in experiments where young physicists can make a major impact at all levels, ranging from the construction of the detector to the analysis of the data. Last, but not least, such local science has an important impact in the italian industry and in the appreciation of science by the public in general. 

{\bf We argue that, in order to balance and optimise the current big-science effort in Spain, it is necessary to develop the physics at the LSC, in analogy to the italian case.} NEXT is the flagship experiment of our national laboratory, and has achieved intentional recognition, as demonstrated by the fact that is a recognised CERN experiment and has obtained an AdG/ERC, {\bf the first grant of this type in the field of particle physics}. 

We, therefore, consider that the NEXT project is a clear example of social innovation, as it has the potential of implementing profound changes in spanish science. As described in this project, NEXT, through its various stages, can hit a major discovery. It will bring international credit and visibility to our science and to the LSC. And it has an important impact both in local industry (through contracts to many national firms, and development of high technology) and in the public perception of science. 

Furthermore, the on-going collaboration with the CLPU further reinforces the above arguments, since the effort involves now a second national scientific installation. In addition, the \BATA\ program implies a major example of inter disciplinarity, and can result in a number of important technological returns (development of infra-red laser technology, which has a myriad scientific and technological applications).

Finally it is important to remark that, while the usual operation of big-science in Spain implies to finance the participation of our groups (including the annual CERN quota, the common-fund of the experiments and the contributions to construction and operation of the CERN experiments), the national science that NEXT represents obtains external funding through ERC projects (including the AdG and several H2020 actions currently in progress involving LSC), as well as the contributions of the international collaboration to detector construction and operation (in particular, in the case of NEXT through the USA groups led by Prof. Dave Nygren, the inventor of the technology in which NEXT is based). NEXT also attracts external talent to our country (as the intense collaboration with top USA universities demonstrates). The NEXT group is very international, and several of our post-docs are or have been financed by EC grants (such as the Marie Curie). 

Last but not least, the NEXT experiment, and in particular the collaboration with the CLPU, involves the extensive development of photonics listed as one of the  ``Facilitating Essential Technologies''.