\subsubsection{Scientific and technological impact}

This project involves the construction of detectors which are unique in the World, implementing the HPXe technology with EL readout, widely considered as one of the most promising ones in the field of \bbonu\ searches. Therefore: 
\begin{enumerate}
\item {\bf The NEXT experiment has the potential of being in the forefront of a major scientific discovery}. The sensitivity of NEXT-100 will reach that achieved by GERDA, EXO and KamLAND-Zen in 2018, without being yet background-saturated. Consequently, from 2018 onwards, NEXT will probe a region of \mbb\ yet unexplored. If the NME is sufficiently large, NEXT could register \bbonu\ events, thus making (or participating in the making, together with other leading experiments of the field) a major discovery.
\item {\bf The NEXT will bring innovation to the industry-science relation in Spain}. The experiment uses and develops high technology, involving national and international firms. We have written statements of interest by companies in the mechanical sector (AIMPLAS, ACEX), gas and pressure (Swagelock, SERA), and light sensors (SENSL, Hamamatsu), among others. We are starting join R\&D  projects with a number of them (radio-pure SiPMs with SENSL, UV sensitive SiPMs with Hamamatsu, new plastic materials with AIMPLAS).  
\item {\bf Our approach to \BATA\ is opening a new inter-disciplinary field}, which incorporates elements of atomic, nuclear, particle physics, laser-matter interactions and photonics. It involves the technology of \HPXE\ chambers, ion sources, magnetic-traps and visible as well as IR lasers. There is a clear potential to develop experimental techniques associated to our experiments that go beyond our particular needs and have wide applications.
\end{enumerate}

\subsubsection{Dissemination of results}

The results of the experiment will be amply advertised. NEXT and CLPU keep modern web pages, and a firm presence in social media. In addition, the PI of NEXT is the science-advisor of the well known JotDown magazine\footcite{JotDown}, and develops an intense outreach activity which involves interviews to scientific personalities\footcite{JotDownNygrenBettini, JotDownCattaiGonzalez, JotDownHalzen} as well as a well read scientific blog\footcite{JotDownBlog}.

\subsubsection{Technology transfer}

Among the potential applications, scientific and industrial returns, we will mention: a) the development of solid-state MIR laser technology, and b) the development of matrices of sensors (SiPMs) and the associated software suitable for medical imaging. 
\begin{enumerate}
\item Lasers in the MIR spectrum are invisible to the human eye and not absorbed by the atmosphere, having them obvious applications in the surveillance industry. At the same time, IR also enhances night vision equipment capabilities by flooding forward positions with invisible light that enhances light gathering performance and increasing the ability to detect objects at greater lengths. 
\item The dense SiPM arrays used in the NEXT tracking plane, together with the associated electronics and software, may have direct applications to novel imaging systems in medicine capable of better resolution and/or operating with lower doses. This is due to the capabilities of SIPMs of reconstructing 3D positions for dim signals. 
\end{enumerate}