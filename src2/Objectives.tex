%\subsection*{Objectives and methodology}

%2. La hipótesis de partida y los objetivos generales perseguidos con el proyecto coordinado en su conjunto, así como la adecuación del proyecto a la Estrategia Española de Ciencia y Tecnología y de Innovación y, en su caso, a Horizonte 2020 o a cualquier otra estrategia nacional  o internacional de 

The overall objectives of this research proposal are:

\begin{enumerate}
\item Construction, commissioning and operation of the NEW and NEXT-100 detectors, during a period of 4 years, from 2015 to 2019.
\item Demonstrate the feasibility of barium tagging in an HPXe, performing a systematic set of small, focused, prove-of-concept experiments. As a part of this objective, an infrared laser using innovative technology will be developed. 
\end{enumerate}
  
The COORD subproject leads the construction of the NEW and NEXT-100 detectors, while the ENG subproject leads the deployment of the electronics, DAQ and slow controls. The CALREC subproject leads the calibration of the detector. The R\&D for barium tagging is lead by the BATA subproject (CLPU), with the participation of all the groups.   

The specific objectives of all the sub projects are integrated in the NEXT Project Management Plan. 
The PMP coordinates the construction of the NEW and NEXT-100 detectors. It is under the direct supervision of the Spokesperson (SP) and the Project Manager (PM). The PM of NEXT is Dr. I. Liubarsky, part of the IFIC group. 

The PMP defines a set of Working Packages (WP) and follows the progress of each one, monitors deliverables and dead lines and keeps track of invested resources including personnel. It also identifies potential show-stoppers and synergies (and possible conflicts) between the different projects and optimises the sharing of resources. 

%Figure \ref{fig.Gantt} shows an example of the Gantt chart for the whole NEW project, up to installaton at the LSC. 

%The objectives defined for the different sub projects match the various WP in the PMP, as can be seen in Figure \ref{Fig:PMP}. 

The methodology of each WP includes: a) the definition of the associated tasks; b) the identification of the resources needed; c) the temporal organisation of the tasks; d) the definition of milestones and the deliverables associated to them; e) the relations with other WP. Each WP has a leader, which reports directly to the PM. The progress of each WP is reviewed on a weekly basis. Milestones and potential showstoppers are discussed, and the tracking charts updated if needed. The PMP is reviewed every six months by the LSC scientific committee.  

The objectives presented in this project are very well aligned to the spanish program for science, as demonstrated by the fact that NEXT has been supported by the CONSOLIDER-INGENIO project CUP. The support of the AdG/ERC makes it clear that the projects suits perfectly well the goals of H2020. NEXT is a CERN recognised experiment and has been listed by NSA\footcite{NLDBD} as one of the key \bbonu\ experiments in the field, and the one with best future prospects.




