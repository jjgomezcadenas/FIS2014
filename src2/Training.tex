This coordinated project requests 3 FPI fellowships, one by COORD, one by CALREC and one by BATA. The students will be enrolled in the Nuclear and Particle Physics doctorate at the UV (COORD), US (CALREC) and USAL (BATA). As part of their training they will attend at least one international and one national school. They will also spent a fraction of their time in international institutions participating in NEXT, in particular Coimbra and Texas Arlington. 

This co-ordinated project is  highly multi disciplinar, offering the students the capability of cross-training and developing skills in a variety of areas, ranging from laser-matter interactions (BATA subproject) to sophisticated data reconstruction with algorithms suitable for medical physics (one of the projects in CALREC) and state-of-the art instrumentation (construction of NEW and NEXT). We intend that the students in this coordinated project rotate between different areas of work (and groups), to acquire a wide background, before settling in a specific topic. 

The NEXT collaboration has recently approved a publication model in which the main authors of the analysis are the first authors (as opposite to the general trend in the field, which uses alphabetical order). The goal of this policy is to facilitate the visibility of graduate students and young postdocs (for example, in \footcite{Lorca:2014sra}, the first two authors are graduate students and the other two post-docs) and to encourage them to take leading roles in the development of the experiment and the analysis of the data. 

\subsubsection*{Training: COORD}
The IFIC group has currently 2 soon-to-be new doctors (J. Martin-Albo and F. Monrabal, who present their Ph.Ds in October 2014) and 4 graduate students (Lorca, Serra, Simo and Nebot), who will defend their Ph.D. thesis in 2015 (Lorca and Serra) and 2016 (Nebot and Simo). 

The PI of this project has advised a total of 11 students (De Fez, Lozano, Hernando, Cervera, Manel, Burguet, Tornero, Catala, Novella, Martin-Albo and Monrabal). De Fez, Lozano, Hernando and Cervera have obtained academic positions (Emirates, Granada, U. of Santiago and Valencia). Novella has been a Marie Curie fellow and has just re-joined the group with a R\&C. Burguet works in a software company developing internet products and Tornero has a permanent position in medical physics (radiotherapy). Catala, has continued his career as a teacher. Martin-Albo and Monrabal have already offers for post-doc position in the US.

The construction, commissioning and operation of NEXT is a great opportunity for graduate students, who can take since very early, leading roles in important parts of the experiment. In particular, we plan to enrol 2-3 students in the next few years, replacing those that graduate. One position is requested to this project, through an FPI grant and two others will be seemed through other national and international grants. In addition of the PI, the group includes experience senior physicists, such as Dr. Sorel, the analysis convener who has already co-advised the Ph.D. of Catala. Other senior physicists in the group include Dr. Liubarsky, Dr. Yahlali and Dr. Novella. 

\subsubsection*{Training: CALREC}

The PI of CALREC has advised two high-impact Ph.D. thesis in the LHCb (Martínez Santos
and Cid Vidal). Both students continue in research were they are following bright careers.   
Dr. Mart\'inez Santos was research fellow at CERN and is now a postdoctoral associate at NIKHEF institute, Amsterdam, as well as CERN Corresponding Associate. He is the coordinator of the $B_s$ mixing phase, $\phi_s$, (the second main LHCb result after the $B_s \to \mu^+\mu^-$ search). He was awarded in 2013 with the Young Experimental Physicist Prize by the European Physical Society (EPS) for his work at the LHCb trigger and the search of the $B_s \to \mu^+ \mu^-$ decay. Dr. Cid Vidal is currently a CERN fellow. He is currently working on the identification of the Higgs boson to b,b-bar jets at LHCb and leading the strange mesons physics at LHCb. He has recently presented a Marie Curie ITN  (International Training Network) proposal to extend kaon physics, and to apply multivariate methods used in HEP into other fields, for example to study the evolution of financial markets.
 
 \subsubsection*{Training capabilities of the team}
The CLPU has a large staff of scientists who can advise a student. In particular, three
Ph.D. thesis are currently under way at the CLPU. One on the construction of a extreme flux proton source (Valle Brozas), a second related with the use of particle acceleration towards hadron therapy (Stockhaussen) and a third related with the development of Femtosecond X-ray sources from laser-driven electron accelerators (D{\oe}pp). 


