\subsubsection*{\label{subsubsec:training}Training plan}

This coordinated project requests 2 Ph.D. fellowships, one for the COORD subproject and one for the CALREC subproject. The students will be enrolled in the Nuclear and Particle Physics doctorate programs at University of Valencia (COORD) and University of Santiago de Compostela (CALREC). As part of their training, students will attend at least one international and one national school. Students will also spend a fraction of their time in international institutions participating in NEXT, in particular Coimbra, LBNL and Texas University. 

This coordinated project is highly multi-disciplinary, offering the students the capability of cross-training and of developing skills in a variety of areas, including laser-matter interactions, sophisticated data reconstruction with algorithms suitable for medical physics, and state-of-the-art instrumentation. Students in this coordinated project will rotate between different areas of work (and groups) to acquire a wide background, before settling onto a specific topic. 

Finally, the construction, commissioning and operation of NEXT is a great opportunity for graduate students, who typically take leading roles in important parts of the experiment. The NEXT collaboration has recently approved a publication model in which the main authors of the analysis are the first authors (as opposed to the general trend in the field of particle physics, which uses alphabetical order). The goal of this policy is to facilitate the visibility of graduate students and young postdocs, and to encourage them to take leading roles in the development of the experiment and the analysis of the data. As an example, in \footcite{Lorca:2014sra}, the first two authors are graduate students and the other two post-docs.



\subsubsection*{Supervision of Ph.D. theses and career development of former Ph.D. students}

\paragraph{COORD subproject.}
The IFIC group has currently two Ph.D. students presenting his thesis in October (J. Martin-Albo and F. Monrabal), two students to defend their Ph.D. theses in 2015 (Lorca and Serra), and two to finish in 2016 (Nebot and Simo). For the future, we plan to enrol 2-3 students in the next few years, replacing those that graduate. One Ph.D. fellowship is requested for this subproject and two others will be made available through other national and international grants. The group includes several senior physicists with extensive experience in mentoring students, such as Prof. G\'omez-Cadenas, Dr. Sorel, Dr. Liubarsky and Dr. Novella. 

Prof.~G\'omez-Cadenas (COORD co-PI) has advised a total of 11 students: De Fez (Medida de las fracciones de desintegración topologicas del leptron Tau, 1994), Lozano (Estudio del canal electrónico de desintegración del Lepton TAU en LEP, 1995), Hernando (Busqueda de oscilaciones numu-nutau en el experimento Nomad del CERN, 1998), Cervera (Diffractive Production de A1+Mesons by neutrinos in Nomad, 2002), Vidal (The HARP time projection chamber, 2003), Tornero (Study of neutral pion production via neutrino-induced, charged-current interactions in the K2K scibar detector, 2008), Burguet (Physics of the Neutrino Factory and related long baseline designs, 2008), Novella (Experimental studies of neutrino nature: from K2K to Supernemo, 2009), Catala ( Measurement of Neutrino Induced charged current neutral pion production cross section at Sciboone, 2014), Martin-Albo (The NEXT experiment for neutrinoless double beta decay searches, 2014) and Monrabal (Demonstration of electroluminescent TPC technology for neutrinoless double beta searches using the NEXT-DEMO detector, 2014). Dr.~Sorel (COORD co-PI) has co-advised Tornero and Catala. De Fez, Lozano, Hernando and Cervera have obtained academic positions (Emirates, Granada, U. of Santiago and IFIC). Novella has been a Marie Curie fellow and has just re-joined the group with a RyC position. Burguet works in a software company developing Internet products and Tornero has a permanent position in medical physics (radiotherapy). Catala and Vidal have continued their careers as teachers. Martin-Albo and Monrabal have already offers for post-doc position in the USA.

\paragraph{CALREC subproject.}

Prof.~Hernando (CALREC PI) has advised two high-impact Ph.D. theses in the LHCb experiment: Martínez Santos (Search for the very rare decay $B_s \to \mu^+\mu^-$ in the LHCb experiment, 2010) and Cid Vidal (Search for the rare decays $B_s \to \mu^+\mu^-$ and $K_S \to \mu^+\mu^-$ in the LHCb with 1 fb$^{-1}$ integrated luminosity, 2012). Both students continue in research where they are following bright careers. Dr. Mart\'inez Santos was research fellow at CERN and is now a postdoctoral associate at the NIKHEF institute, Amsterdam, as well as CERN Corresponding Associate. He is the coordinator of the $B_s$ mixing phase, $\phi_s$, (the second main LHCb result after the $B_s \to \mu^+\mu^-$ search). He was awarded the 2013 Young Experimental Physicist Prize by the European Physical Society (EPS) for his work on the LHCb trigger and on the search for the $B_s \to \mu^+ \mu^-$ decay. Dr. Cid Vidal is currently a CERN fellow. He is currently working on the identification of the Higgs boson to $b$,$\overline{b}$ jets at LHCb, and leading the strange mesons physics at LHCb. He has recently submitted a Marie Curie ITN  (International Training Network) proposal on kaon physics, and to apply multivariate methods used in HEP into other fields, for example to study the evolution of financial markets.
 

