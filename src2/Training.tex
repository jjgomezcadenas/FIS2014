This coordinated project requests 2 FPI fellowships, to go to the COORD, and the CALREC sub projects. The students will be enrolled in the Nuclear and Particle Physics doctorate at the UV (COORD), and US (CALREC). As part of their training they will attend at least one international and one national school. They will also spent a fraction of their time in international institutions participating in NEXT, in particular Coimbra, LBNL and Texas U. 

This co-ordinated project is  highly multi disciplinar, offering the students the capability of cross-training and developing skills in a variety of areas, ranging from laser-matter interactions (BATA subproject) to sophisticated data reconstruction with algorithms suitable for medical physics (CALREC) and state-of-the art instrumentation (COORD). We intend that the students in this coordinated project rotate between different areas of work (and groups), to acquire a wide background, before settling in a specific topic. 

The NEXT collaboration has recently approved a publication model in which the main authors of the analysis are the first authors (as opposite to the general trend in the field, which uses alphabetical order). The goal of this policy is to facilitate the visibility of graduate students and young postdocs (for example, in \footcite{Lorca:2014sra}, the first two authors are graduate students and the other two post-docs) and to encourage them to take leading roles in the development of the experiment and the analysis of the data. 

\subsubsection*{Training: COORD}
The IFIC group has currently 2 soon-to-be new doctors (J. Martin-Albo and F. Monrabal, who present their Ph.Ds in October 2014) and 4 graduate students (Lorca, Serra, Simo and Nebot), who will defend their Ph.D. thesis in 2015 (Lorca and Serra) and 2016 (Nebot and Simo). 

The PI of this project has advised a total of 11 students: De Fez (94), Lozano (95), Hernando (98), Cervera (02), Vidal (03), Burguet (08), Tornero (08), Novella (09), Catala (14) Martin-Albo (14) and Monrabal (14)\footnote{Medida de las fracciones de desintegración topologicas del leptron Tau; Estudio del canal electrónico de desintegración del Lepton TAU en LEP ; Busqueda de oscilaciones numu-nutau en el experimento Nomad del CERN; Diffractive Production de A1+Mesons by neutrinos in Nomad; The HARP time projection chamber; Study of neutral pion production via neutrino-induced, charged-current interactions in the K2K scibar detector; Physics of the Nutrino Factory and related long baseline designs;Experimental studies of neutrino nature: from K2K to Supernemo; Measurement of Neutrino Induced charged current neutral pion production cross section at Sciboone; }. Dr. Sorel has co-advised Tornero and Catala. De Fez, Lozano, Hernando and Cervera have obtained academic positions (Emirates, Granada, U. of Santiago and Valencia). Novella has been a Marie Curie fellow and has just re-joined the group with a R\&C. Burguet works in a software company developing internet products and Tornero has a permanent position in medical physics (radiotherapy). Catala and Vidal, have continued their career as teacheres. Martin-Albo and Monrabal have already offers for post-doc position in the US.

The construction, commissioning and operation of NEXT is a great opportunity for graduate students, who typically take leading roles in important parts of the experiment. In particular, we plan to enrol 2-3 students in the next few years, replacing those that graduate. One position is requested to this project, through an FPI grant and two others will be seemed through other national and international grants. In addition of the PI, the group includes experienced senior physicists, such as Dr. Sorel, Dr. Liubarsky, Dr. Yahlali and Dr. Novella. 

\subsubsection*{Training: CALREC}

The PI of CALREC has advised two high-impact Ph.D. thesis in the LHCb. Martínez Santos
(2010) and Cid Vidal (2012)\footnote{Search for the rare decays $B_s \to \mu^+\mu^-$
and $K_S \to \mu^+\mu^-$ in the LHCb with 1 fb$^{-1}$ integrated luminosity; Search for the very rare decay $B_s \to \mu^+\mu^-$ in the LHCb experiment}. Both students continue in research were they are following bright careers.   
Dr. Mart\'inez Santos was research fellow at CERN and is now a postdoctoral associate at NIKHEF institute, Amsterdam, as well as CERN Corresponding Associate. He is the coordinator of the $B_s$ mixing phase, $\phi_s$, (the second main LHCb result after the $B_s \to \mu^+\mu^-$ search). He was awarded in 2013 with the Young Experimental Physicist Prize by the European Physical Society (EPS) for his work at the LHCb trigger and the search of the $B_s \to \mu^+ \mu^-$ decay. Dr. Cid Vidal is currently a CERN fellow. He is currently working on the identification of the Higgs boson to b,b-bar jets at LHCb and leading the strange mesons physics at LHCb. He has recently presented a Marie Curie ITN  (International Training Network) proposal to extend kaon physics, and to apply multivariate methods used in HEP into other fields, for example to study the evolution of financial markets.
 

