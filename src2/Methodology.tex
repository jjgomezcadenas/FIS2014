The specific objectives of all the sub projects are integrated in the NEXT Project Management Plan. The PMP coordinates the construction of the NEW and NEXT-100 detectors. It is under the direct supervision of the Spokesperson (SP) and the Project Manager (PM). The PM of NEXT is Dr. I. Liubarsky, part of the IFIC group. 

The PMP defines a set of Working Packages (WP) and follows the progress of each one, monitors deliverables and dead lines and keeps track of invested resources including personnel. It also identifies potential show-stoppers and synergies (and possible conflicts) between the different projects and optimises the sharing of resources. 

%Figure \ref{fig.Gantt} shows an example of the Gantt chart for the whole NEW project, up to installaton at the LSC. 

%The objectives defined for the different sub projects match the various WP in the PMP, as can be seen in Figure \ref{Fig:PMP}. 

The methodology of each WP includes: a) the definition of the associated tasks; b) the identification of the resources needed; c) the temporal organisation of the tasks; d) the definition of milestones and the deliverables associated to them; e) the relations with other WP. Each WP has a leader, which reports directly to the PM. The progress of each WP is reviewed on a weekly basis. Milestones and potential showstoppers are discussed, and the tracking charts updated if needed. The PMP is reviewed every six months by the LSC scientific committee.  