%%%%%%%%%%%%%%%%%%%%%%%%%%%%%%%%%%%%%%%%%%%%%%%%%%%%%%%%%%%%
\documentclass[a4paper,11pt,oneside]{article}
\usepackage[a4paper,vmargin={1.5cm,1.5cm},width=16cm]{geometry}
\usepackage[style=verbose-inote,doi=false,sortcites=true,block=space,backend=bibtex]{biblatex}
\usepackage[utf8]{inputenc}
\usepackage{textcomp}
\usepackage[spanish]{babel}
\usepackage{microtype}
\usepackage{lmodern}
\usepackage{graphicx}
\usepackage{fancyhdr}
\usepackage{booktabs}
\usepackage{eurosym}
\usepackage{mathptmx}
\usepackage[T1]{fontenc}
\usepackage{hyperref}
%% Added to help mimic structure.
\usepackage{tcolorbox}
\usepackage{soul}
\usepackage{color}
\usepackage{lastpage}
%%%%%%%%%%%%%%%%%%%%%%%%%%%%%%%%%%%%%%%%%%%%%%%%%%%%%%%%%%%%
%% HEADERS
%\setlength{\headheight}{1cm}
%\setlength{\headsep}{0.5cm}
%\pagestyle{fancyplain}
%\fancyhf{}
%\lhead{\fancyplain{}{\sc Memoria científico técnica de proyectos coordinados}}
%\rhead{\fancyplain{}{\sc Parte A}}
%\cfoot{\thepage}
%\renewcommand{\headrulewidth}{0pt} % remove lines
%\renewcommand{\footrulewidth}{0pt}


%%% HEADER
\setlength{\headheight}{1cm}
%\setlength{\headwidth}{20cm}
\setlength{\headsep}{0.5cm}
\pagestyle{fancyplain}
\fancyheadoffset[HR,HL]{2cm}
\fancyhf{}
\lhead{\raisebox{-0.4\height}{\includegraphics[height=0.9cm,keepaspectratio=true]{img/miniLogo}}}
\rhead{\fancyplain{}{\fontsize{10}{12} \selectfont \textbf{\underline{Memoria científico técnica de proyectos coordinados}}}}
\cfoot{\thepage\, / parte A}
\renewcommand{\headrulewidth}{0pt} % remove lines
\renewcommand{\footrulewidth}{0pt}
%%%%%%%%%%%%%%%%%%%%%%%%%%%%%%%%%%%%%%%%%%%%%%%%%%%%%%%%%%%%
%% Hack to make math formulas bold in section titles
\makeatletter
\DeclareRobustCommand*{\bfseries}{%
  \not@math@alphabet\bfseries\mathbf
  \fontseries\bfdefault\selectfont
  \boldmath
}
\makeatother

%%%%%%%%%%%%%%%%%%%%%%%%%%%%%%%%%%%%%%%%%%%%%%%%%%%%%%%%%%%%
\def\thesection{\bf \textsf{\Alph{section}}}

%\nobibliography{biblio}
%\bibliographystyle{JHEP}

\bibliography{biblio}


%%%%%%%%%%%%%%%%%%%%%%%%%%%%%%%%%%%%%%%%%%%%%%%%%%%%%%%%%%%%
\begin{document}

%% Some useful definitions
\input{src/Commands.tex}

%% Heading
\begin{tcolorbox}[colback=white,arc=0pt,outer arc=0pt,colframe=black,boxrule=0.6pt]
\begin{center}
Convocatorias 2014\\ 
Proyectos de I+D ``Excelencia'' y Proyectos de I+D+I ``Retos Investigación" \\ 
Dirección General de Investigación Científica y Técnica \\
Subdirección General de Proyectos de Investigación
\end{center} 
\end{tcolorbox}

\begin{tcolorbox}[colback=yellow,arc=0pt,outer arc=0pt,colframe=black,boxrule=0.6pt,left=0mm,right=0mm]
  \begin{center}
    AVISO IMPORTANTE\\
  \end{center}
    En virtud del art\'iculo 11 de la convocatoria \ul{\textbf{NO SE ACEPTAR\'AN NI SER\'AN SUBSABABLES MEMORIAS CIENT\'IFICO-T\'ECNICAS}} que no se presenten en este formato.\\
    \\
    \textbf{Lea detenidamente las instrucciones que figuran al final de este documento para rellenar correctamente la memoria cient\'ifico-t\'ecnica.}
    \\
  %\end{center}
\end{tcolorbox}
\vspace{3pt}
\begin{tcolorbox}[colback=yellow,arc=0pt,outer arc=0pt,colframe=black,boxrule=0.6pt,left=0mm]
  \noindent\textbf{Parte A: RESUMEN DE LA PROPUESTA/SUMMARY OF THE PROPOSAL}
  %\section{RESUMEN DE LA PROPUESTA/SUMMARY OF THE PROPOSAL}
\end{tcolorbox}

%%%%%%%%%%%%%%%%%%%%%%%%%%%%%%%%%%%%%%%%%%%%%%%%%%%%%%%%%%%%%%%%%%%%%%%%%%%%%

%\section{RESUMEN DE LA PROPUESTA/SUMMARY OF THE PROPOSAL}

%%%%%%%%%%%%%%%%%%%%%%%%%%%%%%%%%%%%%%%%%%%%%%%%%%%%%%%%%%%%%%%%%%%%%%%%%%%%%

%\subsection{DATOS DEL PROYECTO COORDINADO}
\noindent\textbf{A.1. DATOS DEL PROYECTO COORDINADO}
\vspace{6pt}

\noindent\textbf{INVESTIGADOR COORDINADOR PRINCIPAL 1:} (Nombre y apellidos)

\noindent Juan José Gómez Cadenas.
\vspace{6pt}

\noindent\textbf{INVESTIGADOR COORDINADOR PRINCIPAL 2:} (Nombre y apellidos)

\noindent Michel Sorel
\vspace{6pt}

\noindent\textbf{TÍTULO GENERAL DEL PROYECTO COORDINADO:} Construcción operación e I+D+i para el experimento NEXT en el LSC.
\vspace{6pt}

\noindent\textbf{ACRÓNIMO DEL PROYECTO COORDINADO:} NEXT.


\noindent\textbf{RESUMEN DEL PROYECTO COORDINADO} 
{\color{blue}{M\'aximo 3500 caracteres (incluyendo espacios en blanco):}}
\vspace{12pt}

%%

NEXT (Neutrino Experiment with a Xenon TPC) es un experimento para buscar desintegraciones doble beta sin neutrinos (\bbonu), cuya detección demostraría unívocamente que el neutrino es una partícula de Majorana (es decir su propia antipartícula) y supondría un descubrimiento con profundas consecuencias en física de partículas y cosmología. 

La primera fase de NEXT ha sido completada con éxito. Durante esta etapa el grupo del IFIC ha construido el prototipo NEXT-DEMO que opera en nuestro laboratorio desde 2011 y ha sido esencial para  demostrar las características principales de la tecnología, a saber: excelente resolución de energía y caracterización de la señal mediante la reconstrucción de las trayectorias de los dos electrones emitidos en la desintegración \bbonu. 

Este  proyecto requiere financiación para demostrar una nueva idea que aumentaría dramáticamente las posibilidades de realizar un descubrimiento y situaría a NEXT como el líder mundial del campo. Concretamente se propone añadir un campo magnético al experimento, capaz de aportar una caracterización adicional a la señal, ya que las trayectorias de los dos electrones emitidos en el proceso \bbonu\ se curvarían siguiendo una doble hélice. Para demostrar experimentalmente esta señal, se propone operar NEXT-DEMO en un imán solenoidal (TPC90) disponible en el CERN.

Este proyecto propone también la demostración de un avance radical en tecnología: el uso de sensores de estado sólido (SiPMs) de bajo ruido y capaces de funcionar en campo magnético, y que reemplazarían a los fotomultiplicadores con los que NEXT-DEMO mide la energía de los sucesos. Este desarrollo tendría inmediatas aplicaciones a la técnica PET-MRI que combina tomografía de electrón-positrón con resonancia magnética y necesita, por tanto, cubrir amplias áreas con sensores de bajo ruido capaces de operar en campo magnético.   

 
\vspace{12pt}

\noindent\textbf{PALABRAS CLAVE DEL PROYECTO COORDINADO:} neutrinos, TPC, HPXe, xenón, desintegración doble beta, Canfranc, alta presión, electroluminescencia. 

\vspace{12pt}

\noindent\textbf{TITLE OF THE COORDINATED PROJECT:} Construction operation and R\&D for the NEXT experiment at the LSC. 
\vspace{6pt}

\noindent\textbf{ACRONYM OF THE COORDINATED PROJECT:} NEXT.
\vspace{6pt}

\noindent\textbf{SUMMARY OF THE COORDINATED PROJECT} 
\vspace{6pt}

%%

NEXT (Neutrino Experiment with a Xenon TPC) is an experiment to search for the neutrinoless double beta decay process (\bbonu). The detection of such a process would demonstrate that neutrinos are Majorana particles (that is, their own antiparticles) and would have profound consequences in physics and cosmology.  

The isotope chosen by NEXT is  \XE. The collaboration has access to one hundred kilograms of xenon gas enriched at 90\% in \XE, owned by the Underground Laboratory of Canfranc (LSC). The NEXT technology is based on the use of time projection chambers operating at a typical pressure of 15 bar (\HPXE). The main advantages of the experimental technique are: a) excellent energy resolution; b) the ability to reconstruct the trajectory of the two electrons emitted in the decays, further contributing to the suppression of backgrounds; c) scalability to large masses; and d) the possibility to reduce the background to negligible levels thanks to the barium tagging technique (\BATA).

The NEXT road map was designed in four stages: i) Demonstration of the \HPXE\ technology with prototypes deploying a mass of natural xenon in the range of 1 kg; ii) Characterisation of the backgrounds to the \bbonu\ signal and measurement of the \bbtnu\ signal with the NEW detector, deploying 12 kg of enriched xenon and operating at the LSC; iii) Search for \bbonu\ decays with the NEXT-100 detector, which scales up the NEW detector by a factor 2:1 in size (8:1 in mass) and deploys, thus, 100 kg of enriched xenon. iv) Search for \bbonu\ decays with the BEXT detector (Barium-tagging Experiment with a Xenon TPC), which will deploy a mass in the ton scale and will introduce the \BATA\ technique in order to reduce backgrounds to negligible levels.  

The first stage of NEXT has been successfully completed during the period 2009-2013. The prototypes NEXT-DEMO (IFIC) and NEXT-DBDM (Berkeley) were built and demonstrated the main features of this technology. The experiment is currently developing its second phase. The NEW detector is being constructed during 2014 and will operate in the LSC during 2015. The funding for the construction and operation of NEW comes from an ERC Advanced Grant (AdG/ERC) granted to the co-coordinator of this project in 2013 (project duration: February 2014 - January 2019). The NEXT-100 detector is the third phase of the experiment. It will be built and commissioned during 2016 and 2017, and will start data taking in 2018. The NEXT-100 detector already has a significant discovery potential. Its findings will influence the fourth phase of the experiment (BEXT), which could start in 2020. 

NEXT is an international collaboration, led by Spanish groups (G\'omez-Cadenas, co-coordinator of this proposal, is the spokesperson of the collaboration) and with a very significant contribution of US groups. The laser technology needed for the BEXT phase is being developed in collaboration with the Spanish Pulsed Laser Center (CLPU). 

This proposal requires {\em co-funding} to complete the phase three of the experiment. Specifically we request: a) funds to co-finance the construction of the NEXT-100 detector (which is being partially funded by the AdG as well as by the international collaboration, primarily US groups); b) funds to co-finance personnel; and c) a  contribution to the R\&D to develop the \BATA\ technology.   



 \vspace{12pt}

\noindent\textbf{KEYWORDS OF THE COORDINATED PROJECT:} neutrinos, TPC, HPXe, xenon, double beta decay, Canfranc, high pressure, electroluminescence. 

 \vspace{12pt}
%\newpage

%%%%%%%%%%%%%%%%%%%%%%%%%%%%%%%%%%%%%%%%%%%%%%%%%%%%%%%%%%%%%%%%%%%%%%%%%%%%%

\noindent\textbf{A.2. DATOS DE LOS SUBPROYECTOS/ DATA OF SUBPROJECTS }
 \vspace{12pt}
 
\noindent\textbf{SUBPROYECTO 1 / SUBPROJECT 1} (el investigador o investigadores principales son los coordinadores del proyecto coordinado)

\vspace{6pt}
\noindent\textbf{TÍTULO / TITLE :} Coordination of the NEXT project (COORD)
%
%\begin{itemize}
%\item {\bf IP 1 / PI 1}: Juan José Gómez Cadenas (co-coordinator)
%\item {\bf IP 2 / PI 2}: Michel Sorel (co-coordinator)
%\item {\bf Título / Title}: Coordination of the NEXT project (COORD)
%\end{itemize}

 \vspace{12pt}
 
\noindent\textbf{SUBPROYECTO 2 / SUBPROJECT 2}
\vspace{6pt}

\noindent\textbf{INVESTIGADOR PRINCIPAL 1 / PRINCIPAL RESEARCHER 1 } (nombre y apellidos)

\vspace{6pt}
\noindent Jos\'e F. Toledo

\vspace{12pt}
\noindent\textbf{INVESTIGADOR PRINCIPAL 2 / PRINCIPAL RESEARCHER 2} (nombre y apellidos)

\vspace{6pt}
\noindent Ra\'ul Esteve

\vspace{6pt}
\noindent\textbf{TÍTULO / TITLE :} Electronics and DAQ for NEXT

\vspace{12pt}
\noindent\textbf{SUBPROYECTO 3 / SUBPROJECT 3}
\vspace{6pt}

\noindent\textbf{INVESTIGADOR PRINCIPAL 1 / PRINCIPAL RESEARCHER 1 } (nombre y apellidos)

\vspace{6pt}
\noindent José A. Hernando Morata

\vspace{6pt}
\noindent\textbf{TÍTULO / TITLE :} Calibration and Reconstruction for NEXT (CALREC)

\vspace{12pt}
\noindent\textbf{SUBPROYECTO 4 / SUBPROJECT 4}
\vspace{6pt}

\noindent\textbf{INVESTIGADOR PRINCIPAL 1 / PRINCIPAL RESEARCHER 1 } (nombre y apellidos)

\vspace{6pt}
\noindent Alicia Vázquez

\vspace{6pt}
\noindent\textbf{TÍTULO / TITLE :} Barium Tagging (BATA)


%\rhead{\fancyplain{}{\sc Parte B}}
%\newpage
%\setcounter{page}{1}

%%%%%%%%%%%%%%%%%%%%%%%%%%%%%%%%%%%%%%%%%%%%%%%%%%%%%%%%%%%%%%%%%%%%%%%%%%%%%
\newpage
\setcounter{page}{1}
\cfoot{\fancyplain{}{\thepage\, / parte B}}

%\section{INFORMACIÓN ESPECÍFICA DEL EQUIPO / TEAM INFORMATION}

\begin{tcolorbox}[colback=yellow,arc=0pt,outer arc=0pt,colframe=black,boxrule=0.6pt,left=0mm]
  \noindent\textbf{PARTE B: INFORMACIÓN ESPECÍFICA DEL EQUIPO / TEAM INFORMATION}
  %\section{RESUMEN DE LA PROPUESTA/SUMMARY OF THE PROPOSAL}
\end{tcolorbox}

\vspace{12pt}
\noindent\textbf{B.1 RELACIÓN DE LAS PERSONAS NO DOCTORES QUE COMPONEN EL EQUIPO DE TRABAJO / WORKING TEAM (MEMBERS WITHOUT PH.D) }
%B.1. RELACIÓN DE LAS PERSONAS NO DOCTORES QUE COMPONEN EL EQUIPO DE TRABAJO (se recuerda que los doctores del equipo de trabajo y los componentes del equipo de investigación no se solicitan aquí porque deberán incluirse en la aplicación informática de solicitud). Repita la siguiente secuencia tantas veces como precise para cada uno de los subproyectos.

%1. Nombre y apellidos: 
%Titulación: licenciado/ingeniero/graduado/máster/formación profesional/otros (especificar)
%Tipo de contrato: en formación/contratado/técnico/ entidad extranjera/otros (especificar)%
%Duración del contrato: indefinido/temporal
%Subproyecto al que pertenece (nombre y apellidos del investigador principal):

\begin{enumerate}
\item {\bf Nombre / Name}: David Lorca
\begin{itemize}
\item {\bf Titulación /Title}: Licenciado. 
\item {\bf Tipo de contrato /Contract type}: Formación/Training. 
\item {\bf Duración del contrato /Contract scope}: Temporal/Temporary. 
\item {\bf Subproyecto /Subproject}: COORD. 
\end{itemize}
\item {\bf Nombre / Name}: Ander Simó
\begin{itemize}
\item {\bf Titulación /Title}: Licenciado. 
\item {\bf Tipo de contrato /Contract type}: Formación/Training. 
\item {\bf Duración del contrato /Contract scope}: Temporal/Temporary. 
\item {\bf Subproyecto /Subproject}: COORD. 
\end{itemize}
\item {\bf Nombre / Name}: Luis Serra
\begin{itemize}
\item {\bf Titulación /Title}: Licenciado. 
\item {\bf Tipo de contrato /Contract type}: Formación/Training. 
\item {\bf Duración del contrato /Contract scope}: Temporal/Temporary. 
\item {\bf Subproyecto /Subproject}: COORD. 
\end{itemize}
\item {\bf Nombre / Name}: Miquel Nebot
\begin{itemize}
\item {\bf Titulación /Title}: Licenciado. 
\item {\bf Tipo de contrato /Contract type}: Formación/Training. 
\item {\bf Duración del contrato /Contract scope}: Temporal/Temporary. 
\item {\bf Subproyecto /Subproject}: COORD. 
\end{itemize}
\item {\bf Nombre / Name}: Vicente Álvarez
\begin{itemize}
\item {\bf Titulación /Title}: Ingeniero Técnico / Technical Engineer.  
\item {\bf Tipo de contrato /Contract type}: Técnico / Technical. 
\item {\bf Duración del contrato / Contract scope}: Temporal/Temporary. 
\item {\bf Subproyecto /Subproject}: COORD. 
\end{itemize}
\item {\bf Nombre / Name}: Marc Querol
\begin{itemize}
\item {\bf Titulación /Title}: Ingeniero Técnico / Technical Engineer.  
\item {\bf Tipo de contrato /Contract type}: Técnico / Technical. 
\item {\bf Duración del contrato / Contract scope}: Temporal/Temporary. 
\item {\bf Subproyecto /Subproject}: COORD. 
\end{itemize}
\item {\bf Nombre / Name}: Alberto Martínez
\begin{itemize}
\item {\bf Titulación /Title}: Ingeniero Técnico / Technical Engineer.  
\item {\bf Tipo de contrato /Contract type}: Técnico / Technical. 
\item {\bf Duración del contrato / Contract scope}: Temporal/Temporary. 
\item {\bf Subproyecto /Subproject}: COORD. 
\end{itemize}
\item {\bf Nombre / Name}: Gonzalo Mart\'inez Lema
\begin{itemize}
\item {\bf Titulación /Title}: Master en F\'isica de Part\'iculas / Master in Particle Physics.  
\item {\bf Tipo de contrato /Contract type}: Formación / Training. 
\item {\bf Duración del contrato / Contract scope}: Temporal/Temporary. 
\item {\bf Subproyecto /Subproject}: CALREC. 
\end{itemize}
\end{enumerate}


%%%%%%%%%%%%%%%%%%%%%%%%%%%%%%%%%%%%%%%%%%%%%%%%%%%%%%%%%%%%%%%%%%%%%%%%%%%%%

%\subsection{\label{subsec:Grants}FINANCIACIÓN PÚBLICA Y PRIVADA (PROYECTOS Y/O CONTRATOS DE I+D+I) DEL EQUIPO DE INVESTIGACIÓN / FUNDING OF RESEARCH TEAM}

\noindent\textbf{B.2 FINANCIACIÓN PÚBLICA Y PRIVADA (PROYECTOS Y/O CONTRATOS DE I+D+I) DEL EQUIPO DE INVESTIGACIÓN / FUNDING OF RESEARCH TEAM}
\vspace{12pt}

{\sc SUBPROYECTO 1 / SUBPROJECT 1}

\begin{enumerate}
\item {\bf Investigadores / Researchers }: Juan José Gómez Cadenas, M. Sorel, I. Liubarsky, F. Monrabal, J.Martin-Albo, S. Cárcel, J. Rodríguez, N. López, J. Renner.
\begin{itemize}
\item {\bf Entidad financiadora / Funding Agency}: ERC. Advanced Grant 339787-NEXT 
\item {\bf Título / Title}:  NEXT.
\item {\bf Duración / Duration}: 01/02/2014 -- 01/02/2018. 
\item {\bf Financiación recibida /Grant}: 2.8 M\euro. 
\item {\bf Relación con el proyecto presentado /Relation with this subproject}: Mismo tema /Same topic. 
\item {\bf Estado del proyecto / Status of project}: Concedido / Granted
\item {\bf PI del proyecto / PI of project}: Juan José Gómez Cadenas
\end{itemize}
\item {\bf Investigadores / Researchers }: Juan José Gómez Cadenas, M. Sorel, N. Yahlali, I. Liubarsky, F. Monrabal, J.Martin-Albo, S. Cárcel, J. Rodríguez.
\begin{itemize}
\item {\bf Entidad financiadora / Funding Agency}: Ministerio de Educaci\'on y Ciencia, CSD2008-00037.
\item {\bf Título / Title}:  Canfranc Underground Physics (CUP).
\item {\bf Duración / Duration}: 01/09/2014 -- 31/12/2015. 
\item {\bf Financiación recibida /Grant}: 5.0 M\euro. 
\item {\bf Relación con el proyecto presentado /Relation with this subproject}: Mismo tema /Same topic. 
\item {\bf Estado del proyecto / Status of project}: Concedido / Granted
\item {\bf PI del proyecto / PI of project}: Concepción González-García 
\end{itemize}
\item {\bf Investigadores / Researchers }: Juan José Gómez Cadenas, M. Sorel, N. Yahlali, I. Liubarsky, F. Monrabal, J.Martin-Albo, S. Cárcel, J. Rodríguez.
\begin{itemize}
\item {\bf Entidad financiadora / Funding Agency}:  Ministerio de Econom\'ia y Competitividad, FIS2012-37947-C04-01.
\item {\bf Título / Title}:  Coordination of NEXT Project.
\item {\bf Duración / Duration}: 01/01/2013 -31/12/2014. 
\item {\bf Financiación recibida /Grant}:256,000\euro. 
\item {\bf Relación con el proyecto presentado /Relation with this subproject}: Mismo tema /Same topic. 
\item {\bf Estado del proyecto / Status of project}: Concedido / Granted
\item {\bf PI del proyecto / PI of project}: Juan José Gómez Cadenas 
\end{itemize}
\item {\bf Investigadores / Researchers }: Juan José Gómez Cadenas, M. Sorel.
\begin{itemize}
\item {\bf Entidad financiadora / Funding Agency}:  Ministerio de Educaci\'on y Ciencia, FPA2009-13697-C04-04
\item {\bf Título / Title}:  Física Experimental de Neutrinos en el IFIC / Experimental neutrino physics at IFIC
\item {\bf Duración / Duration}: 01/01/2010 -31/12/2012. 
\item {\bf Financiación recibida /Grant}:534,437\euro. 
\item {\bf Relación con el proyecto presentado /Relation with this subproject}: Muy relacionado / Very related. 
\item {\bf Estado del proyecto / Status of project}: Concedido / Granted
\item {\bf PI del proyecto / PI of project}: Juan José Gómez Cadenas 
\end{itemize}
\end{enumerate}

{\sc SUBPROYECTO 2 / SUBPROJECT 2}

\begin{enumerate}
\item {\bf Investigador / Researcher }: R.Esteve Bosch, F.J.Mora Más, J.F. Toledo Alarcón
\begin{itemize}
\item {\bf Entidad financiadora / Funding Agency}: Ministerio de economía y competitividad: FIS2012-37947-C04-04.
\item {\bf Título / Title}:  Adquisición de datos y diseño mecánico para el experimento NEXT
\item {\bf Duración / Duration}: 01/01/2013 -31/12/2014
\item {\bf Financiación recibida /Grant}: 119.340 \euro. 
\item {\bf Relación con el proyecto presentado /Relation with this subproject}: Mismo tema /Same topic. 
\item {\bf Estado del proyecto / Status of project}: Concedido / Granted
\item {\bf PI del proyecto / PI of project}: J.F. Toledo Alarcón
\end{itemize}
\item {\bf Investigador / Researcher }: JR.Esteve Bosch, F.J.Mora Más, J.F. Toledo Alarcón
\begin{itemize}
\item {\bf Entidad financiadora / Funding Agency}: Ministerio de Educaci\'on y Ciencia, CSD2008-00037.
\item {\bf Título / Title}:  Canfranc Underground Physics (CUP).
\item {\bf Duración / Duration}: 01/09/2014 -- 31/12/2015. 
\item {\bf Financiación recibida /Grant}: 5.0 M\euro. 
\item {\bf Relación con el proyecto presentado /Relation with this subproject}: Mismo tema /Same topic. 
\item {\bf Estado del proyecto / Status of project}: Concedido / Granted
\item {\bf PI del proyecto / PI of project}: Concepción González-García 
\end{itemize}
\item {\bf Investigador / Researcher }: JR.Esteve Bosch, F.J.Mora Más, J.F. Toledo Alarcón
\begin{itemize}
\item {\bf Entidad financiadora / Funding Agency}: Ministerio de Educaci\'on y Ciencia, FPA2007-65013-C02-02-AR07.
\item {\bf Título / Title}:  ISTEMA DE ADQUISICION Y PROCESADO DE DATOS PARA UN SISTEMA DE DIAGNOSTICO PET PARA ENFERMEDADES NEUROLOGICAS
\item {\bf Duración / Duration}:01/10/2007 - 01/11/2010 
\item {\bf Financiación recibida /Grant}: 111.588,38 \euro. 
\item {\bf Relación con el proyecto presentado /Relation with this subproject}: Algo relacionado /Related. 
\item {\bf Estado del proyecto / Status of project}: Concedido / Granted
\item {\bf PI del proyecto / PI of project}: Ángel Sebastiá Cortés 
\end{itemize}
\end{enumerate}


{\sc SUBPROYECTO 3 / SUBPROJECT 3}

\begin{enumerate}
\item {\bf Investigador / Researcher }: José Ángel Hernando
\begin{itemize}
\item {\bf Entidad financiadora / Funding Agency}: Ministerio de Econom\'ia y Competitividad, FPA2011-23608.  
\item {\bf Título / Title}:  Medidas de precisi\'on en F\'isica del sabor en el LHCb
\item {\bf Duración / Duration}: 01/01/2012 -- 31/12/2014
\item {\bf Financiación recibida /Grant}: 563.000 \euro 
\item {\bf Relación con el proyecto presentado /Relation with this subproject}: Relacionado /Related. 
\item {\bf Estado del proyecto / Status of project}: Concedido / Granted
\item {\bf PI del proyecto / PI of project}: Bernardo Adeva Andany
\end{itemize}
\item {\bf Investigador / Researcher }: José Ángel Hernando
\begin{itemize}
\item {\bf Entidad financiadora / Funding Agency}: Ministerio de Educaci\'on y Ciencia, CSD2008-00037.
\item {\bf Título / Title}:  Canfranc Underground Physics (CUP).
\item {\bf Duración / Duration}: 01/09/2014 -- 31/12/2015. 
\item {\bf Financiación recibida /Grant}: 5.0 M\euro. 
\item {\bf Relación con el proyecto presentado /Relation with this subproject}: Mismo tema /Same topic. 
\item {\bf Estado del proyecto / Status of project}: Concedido / Granted
\item {\bf PI del proyecto / PI of project}: Concepción González-García 
\end{itemize}
\item {\bf Investigador / Researcher }: José Ángel Hernando
\begin{itemize}
\item {\bf Entidad financiadora / Funding Agency}: Ministerio de Educaci\'on y Ciencia, FPA2007-65268
\item {\bf Título / Title}:  Contribuci\'on a la reconstrucci\'on de trayectorias y al sistema de disparo de alto nivel del experiment LHCb.
\item {\bf Duración / Duration}: 01/01/2009--31/12/2009
\item {\bf Financiación recibida /Grant}: 190.000 \euro 
\item {\bf Relación con el proyecto presentado /Relation with this subproject}: Algo relacionado / Related. 
\item {\bf Estado del proyecto / Status of project}: Concedido / Granted
\item {\bf PI del proyecto / PI of project}: José Ángel Hernando
\end{itemize}
\end{enumerate}


{\sc SUBPROYECTO 4 / SUBPROJECT 4}

The CLPU has participated or is participating in the following national/european projects: 

\begin{enumerate}

\item Title: ``Laser for Applications at Accelerators"\\
Ref:   LA3NET-ITN     / UE - FP7 \\
Budget:  461.908,8 \euro \\
IP:  Carsten Welsch \\
Institution:  European Network of labs \\
Period:   01/10/2011 - 30/06/2015 \\


\item Title: ``Desarrollo de un l\'aser de femtosegundos low cost para la industria"\\
Ref:  IPT-2011-1121-020000     / MICINN - INNPACTO "Femtolaser" \\
Budget: 117.252 \euro \\
IP:  Luis Roso (for CLPU) \\
Institution:  Easy Laser, s.l.; Univ. Valencia; ICMM and CLPU \\
Period:  03/05/2011 to 31/03/2015 \\

\item Title: ``Investigaci\'on y desarrollo de sistemas avanzados de separaci\'on de gases atmosf\'ericos por ionizaci\'on y magnetismo y su aplicaci\'on a la captura de CO2"\\
Ref:   IPT-2011-1137-310000     / MICINN - INNPACTO "SIGMA" \\
Budget:  312.864 \euro \\
IP:  Luis Roso (for CLPU) \\
Institution:  Iberdrola Ingenier\'ia y Construcci\'on, S.A. and CLPU \\
Period: 03/05/2011 to 31/03/2015 \\

\item Title: ``Dise\~no y desarrollo de elementos tecnol\'ogicos para la aceleraci\'on de part\'iculas mediante l\'aseres ultracortos y ultraintensos"\\
Ref:  IPT-2011-0862-900000     / MICINN - INNPACTO \\
Budget: 242.477 \euro \\
IP:  Luis Roso (for CLPU) \\
Institution:  Proton Laser Applications, S.L.; Univ. Valencia and CLPU \\
Period:   03/05/2011 to 31/02/2014 \\

\item Title: ``CONSOLIDER SAUUL - Science and Applications of Ultrashort Ultraintense Lasers"\\
Ref:  CSD2007-00013     / MEC \\
Budget:  4.500.000 \euro (for the whole network) \\
IP:  Luis Roso (coordinator of 8 labs) \\
Institution:  8 national laser labs \\
Period:  10/12/2007 - 31/12/2013 \\

\item Title: ``L\'aseres ultracortos ultraintensos: F\'isica a intensidades extremas"\\
Ref:  FIS2006-04151     / MEC \\
Budget:  350.900  \euro \\
IP:  Luis Roso \\
Institution: Universidad de Salamanca \\
Period: 01/10/2006 - 30/09/2009 \\



\item Title: ``L\'aseres ultraintensos y su interacci\'on con la materia"\\
Ref:    GR27   / Junta de Castilla y Le\'on \\
Budget: 203.372 \euro \\
IP:  Luis Roso \\
Institution:  CLPU \\
Period: 2008 - 2010



\end{enumerate}



\newpage
\setcounter{page}{1}
\cfoot{\fancyplain{}{\thepage\, de \pageref{LastPage} / parte C}}

%%%%%%%%%%%%%%%%%%%%%%%%%%%%%%%%%%%%%%%%%%%%%%%%%%%%%%%%%%%%%%%%%%%%%%%%%%%%%

%\section{DOCUMENTO CIENTÍFICO / SCIENTIFIC DOCUMENT}

\begin{tcolorbox}[colback=yellow,arc=0pt,outer arc=0pt,colframe=black,boxrule=0.6pt,left=0mm]
  \noindent\textbf{Parte C: DOCUMENTO CIENTÍFICO / SCIENTIFIC DOCUMENT}
  %\section{RESUMEN DE LA PROPUESTA/SUMMARY OF THE PROPOSAL}
\end{tcolorbox}

\vspace{12pt}

\noindent\textbf{C.1. JUSTIFICACIÓN DE LA COORDINACIÓN / JUSTIFICATION OF THE COORDINATION}
\vspace{12pt}

The NEXT experiment is organised as an international collaboration, which includes groups from Spain, Portugal, Russia, USA, and Colombia. The Spanish groups participating in NEXT are: Instituto de Física Corpuscular (IFIC), a joint center of the University of Valencia (UV) and the Spanish National Research Council (CSIC), Polytechnic University of Valencia (UPV), University of Santiago de Compostela (US), Autonomic University of Madrid (UAM), and University of Zaragoza (UZ). 

The groups participating in this coordinated project form the core of the collaboration. The spokesperson (and co-coordinator of this project), the analysis co-coordinator (and co-coordinator of this project), the technical coordinator, the reconstruction co-coordinator and the leaders of the detector construction, are from IFIC. The coordinators of the electronics, DAQ, and slow controls are members of the UPV. The coordinator of calibration is the PI of the US group. The groups participating in this coordinated project invest 100\% of their research time and resources in the NEXT project. 

Furthermore, a strong collaboration has been  formed between NEXT and the Spanish Pulsed Laser Center (CLPU after the initials of the center in Spanish), to develop the laser technology which could be used to tag the barium ion emitted in the \bb\ decays, resulting (when combined with the excellent energy resolution of NEXT and its topological signature) in a virtually background-free experiment. The ambitious R\&D program that could produce a viable scheme for barium tagging (BATA) is also described in this proposal.  

%A key task for the experiment is that of radio purity, coordinated by the UAM (prof. Luis Labarga) who also presents a research project to this call. The project of prof. Labarga includes a proposal to participate in the Super Kamiokande experiment, and for this reason we have considered more appropriated to present separated proposals. The task of radio purity and the corresponding request for resources is described in his project.
%
%Two additional independent projects (in the modality of young researchers) are presented in parallel with this coordinated project. The PI of one of them, Dr. Ferrario, is the coordinator of Monte Carlo and Reconstruction, and her project includes an innovative collaboration with the Fermi National Laboratory (Fermilab) in USA, to handle massive production of the Monte Carlo data needed for NEXT. The PI of the second project, Dr. Laing is the run coordinator of DEMO and will serve as run coordinator of NEW and NEXT-100. His project includes an innovative idea to upgrade the energy plane of the DEMO detector using last-generation SiPMs.




\vspace{12pt}

\noindent\textbf{C.2. PROPUESTA CIENTÍFICA / SCIENTIFIC PROPOSAL}

\subsubsection*{\label{subsubsec:stateoftheart}Estado actual y antecedentes /  State-of-the-art and previous work}

%%%%%%%%%%%%%%%%%%%%%%%%%%%%%%%%%%%%%%%%%%%%%%%%%%%%%%%%%%%%
\subsubsection*{Introduction}
Neutrinos, unlike the other fermions of the Standard Model of particle physics, could be Majorana particles, that is, indistinguishable from their antiparticles. The existence of Majorana neutrinos would have profound implications in particle physics and cosmology. 

 If neutrinos are Majorana particles, there must exist a new scale of physics (at a level inversely proportional to the neutrino masses) that characterises underlying dynamics beyond the Standard Model. The existence of such a new scale provides the simplest explanation of why neutrino masses are so much lighter than the charged fermions. Indeed,  understanding the new physics that underlies neutrino masses is one of the most important open questions in particle physics, and it could have profound implications in our comprehension of the mechanism of symmetry breaking, the origin of mass and the flavour problem. 

The existence of Majorana neutrinos would imply that lepton number is not conserved, which could be the origin of the matter-antimatter asymmetry observed in the Universe. The new physics related to neutrino masses could provide a new mechanism to generate the asymmetry called leptogenesis. Although the predictions are model dependent, two essential ingredients must be confirmed experimentally: 1) the violation of lepton number and 2) CP violation in the lepton sector. 

The only practical way to establish experimentally that neutrinos are their own antiparticles and that lepton number is not conserved is the detection of neutrinoless double beta decay (\bbonu). This is a hypothetical, very slow nuclear transition in which a nucleus with $Z$ protons decays into a nucleus with $Z+2$ protons and the same mass number $A$, emitting two electrons that carry essentially all the energy released (\Qbb). The process can occur if and only if neutrinos are Majorana particles. 

%%%%%%%%%%%%%%%%%%%%%%%%%%%%%%%%%%%%%%%%%%%%%%%%%%%%%%%%%%%%
\subsubsection*{The experimental landscape}

The detectors used in double beta decay searches are designed to measure the energy of the radiation emitted by a \bb\ source. In the case of \bbonu, the sum of the kinetic energies of the two released electrons is fixed by the mass difference between the parent and the daughter nuclei: $Q_{\bb} \equiv M(Z,A)-M(Z+2,A)$. However, due to the finite energy resolution of any detector, \bbonu\ events are reconstructed within an energy region centered around \Qbb, typically following a gaussian distribution (Region of Interest or ROI). Other processes occurring in the detector can fall in the ROI, becoming a background and compromising drastically the expected sensitivity. It follows that \bbonu\ experiments require {\bf excellent energy resolution}, and indeed the field was traditionally dominated by germanium calorimeters, devices with superb resolution.

All double beta decay experiments have to deal with an intrinsic background, the \bbtnu, the standard process of a double $\beta$-decay with the emission of two neutrinos, that can only be suppressed by means of good energy resolution. Backgrounds of cosmogenic origin force the {\bf underground operation of the detectors}. Natural radioactivity emanating from the detector materials and surroundings can easily overwhelm the signal peak, and hence {\bf careful selection of radiopure materials is also essential}. 
{\bf Additional experimental signatures} that allow the distinction between signal and background are certainly a bonus, and this has been in the last few years an important line of work to increase the sensitivity of \bbonu\ detectors. Several other factors such as {\bf detection efficiency} or the {\bf scalability to large masses} must be also taken into account during the design of a double beta decay experiment.
 
 \subsubsection*{Recent results}
 The status of the field has been reviewed recently by the PI\footnote{http://www.inss2014.physics.gla.ac.uk/International\_Neutrino\_Summer\_School\_2014/INSS2014.html}.
 Three new-generation experiments, with fiducial masses in the range of the 100 kg, have recently published the results of their searches for \bbonu\ processes. These are: GERDA, a high resolution calorimeter based in Ge-76 diodes; KamLAND-Zen, a low resolution, high-mass, self-shielding liquid scintillator calorimeter, with xenon dissolved in the scintillator; and EXO-200, a liquid xenon (LXe) TPC. All the experiments published negative results and therefore a limit in the period of \bbonu\ processes, \Tonu. This limit can be translated into a limit in the \emph{effective Majorana mass} of the electron neutrino defined as:
\begin{equation}
\mbb = \Big| \sum_{i} U^{2}_{ei} \ m_{i} \Big| \, ,
\end{equation}
%
where $m_{i}$ are the neutrino mass eigenstates and $U_{ei}$ are elements of the neutrino mixing matrix. \mbb\ is related to the period through the equation:

\begin{equation}
(T^{0\nu}_{1/2})^{-1} = G^{0\nu} \ \big|M^{0\nu}\big|^{2} \ \mbb^{2} \, .
\label{eq:Tonu}
\end{equation}

Here, $G^{0\nu}$ is an exactly-calculable phase-space integral for the emission of two electrons and $M^{0\nu}$ is the nuclear matrix element (NME) of the transition, which has to be evaluated theoretically. The uncertainty in the NME affects the value of \mbb\ which can be obtained from \Tonu.
 
{\bf GERDA} \footcite{Agostini:2013mzu} has a resolution of $\sim$0.2 \% FWHM around the \Qbb\ of \GE. The specific background rate in the ROI is $10^{-2}$ \ckky\ and the total exposure deployed 21.6 kg $\times$ yr. The experiment sets a limit $\Tonu > 2 \times 10^{25}$~yr, which translates  in a range for \mbb\ of $[258-649]$~milli electronvolts (meV). The lowest value of \mbb\ corresponds to the IBM2 NME set, while the highest value corresponds to the ISM set.

{\bf EXO} \footcite{Albert:2014awa} achieves an energy resolution of 3.6\% FWHM at \Qbb, and a background rate of $ 4 \times 10^{-3}\ckky$. The total exposure used for the published result is 100 kg$\cdot$~yr. They have published a limit on the half-life of \bbonu\ in \XE\ of $T_{1/2}^{0\nu}(\XE) > 2 \times 10^{25}$~yr (assuming background only). The limit translates into a range for \mbb\ of $[125-352]$~meV.

{\bf KamLAND-Zen} \footcite{TheKamLAND-Zen:2014lma} compensates a worse energy resolution of 10\% FWHM at \Qbb\ with a very small background rate of $\sim 4 \times 10^{-4}$ \ckky. After an exposure of 108.8 kg$\cdot$~yr, they obtain a limit  $T_{1/2}^{0\nu}(\XE) > 2.6 \times 10^{25}$~yr, which translates into a range for \mbb\ of $[110-309]$~meV.

 \subsubsection*{Potential for discovery}
 
 %%%%%%
\begin{figure}
\centering
\includegraphics[width=0.75\textwidth]{img/SensiCRR.png}
\caption{The allowed \mbb\ region, as a function of the sum of the neutrino masses, assuming that 
$\sum m_i = 0.32$~eV. The blue lines mark the sensitivity of EXO and KamLAND-ZEN, the xenon-based detectors currently leading the field. The red line shows the sensitivity of NEXT after 3 years operation, which gives the experiment a sizeable chance of making a discovery.} 
\label{fig.mbb}
\end{figure}
%%%%%%

 Several analyses from recent cosmological results suggest that the sum of the masses of the three neutrinos could be $\sim$ 0.3 eV\footcite{PhysRevLett.112.051303}. The PI and collaborators have demonstrated that, in this case, if the neutrino is a Majorana particle, then, $\mbb \sim [20-150]$~ meV \footcite{GomezCadenas:2013ue}, as shown in Figure \ref{fig.mbb}. In this scenario, the sensitivity of GERDA is outside the ``cosmologically relevant region'' (CRR), while both EXO-200 and KamLAND-Zen would have already explored a significant fraction of CRR {\em for the most optimistic NME set} (while they would be outside CRR for the most pessimistic). 
 
% Clearly, the experimental effort to determine if the neutrino is a Majorana particle, far from being completed is, rather, in its infancy. To establish unambiguously that the neutrino is (or not) a Majorana particle, even in this favourable scenario in which the sum of the neutrino masses is relatively high, experiments must be sensitive to $\mbb \sim 20$~meV, {\em even for the most pessimistic NME} set. On the other hand, a xenon experiment probing a $\Tonu >2.6 \times 10^{25}$~yr, has chances of making a discovery.
% 
 
%%%%%%%%%%%%%%%%%%%%%%%%%%%%%%%%%%%%%%%%%%%%%%%%%%%%%%%%%%%%
\subsubsection*{The NEXT experiment and its innovative concepts}
\begin{figure}
\centering
\includegraphics[width=0.9\textwidth]{img/NEXT.png}
\caption{\small A drawing of the NEXT-100 detector showing its main parts.  The pressure vessel (PV),  (130 cm inner diameter, 222 cm length, 1cm thick walls, with a total mass of 1\,200 kg) is made of a radio pure steel-titanium alloy.
The inner copper shield (ICS), is made of ultra-pure copper bars, 12 cm thick, with a total mass of 9\,000 kg.The electrical system includes the field cage, cathode, EL grids and HV penetrators.
The light tube is made of thin teflon sheets coated with TPB (a wavelength shifter). 
The energy plane is made of 60 PMTs housed in copper enclosures (cans).
The tracking plane is made of MPPCs arranged into dice boards (DB). 
}
\end{figure}

The \emph{Neutrino Experiment with a Xenon TPC} (NEXT)\footnote{\href{http://next.ific.uv.es/}{http://next.ific.uv.es/}} will search for \bbonu\ in \XE\ using  high-pressure xenon gas  time projection chambers (\HPXE). The advantages of the technology are: 
a) {\bf excellent energy resolution}, with an intrinsic limit of about 0.3\% FWHM at \Qbb, close to that of \GE\ detectors; b)
{\bf tracking capabilities} that provide a powerful topological signature to discriminate between signal (two electron tracks with a common vertex) and background (mostly, single electrons); c)
{\bf a fully active and homogeneous detector}, with no dead regions; d) {\bf scalability} of the technique to larger masses; e) the possibility of exciting the barium ion produced in the xenon decay from the fundamental state \TwoS\ to the state \TwoP, using a ``blue'' laser (493.54 nm), and observing the ``red light'' emitted in the transition from \TwoP to \TwoD, thus ``tagging'' the presence of a barium atom in the xenon gas, which cannot be produced by any known background. 

The design of the NEXT-100 detector (Figure \ref{fig.NEXT100}) is optimised for energy resolution by using proportional electroluminescent (EL) amplification of the ionisation signal. The detection process involves using the prompt scintillation light from the gas as start-of-event time, drifting the ionisation charge to the anode by means of an electric field ($\sim0.3$ kV/cm at 15 bar) where secondary EL scintillation will be produced in the region defined by two highly transparent meshes, between which there is a field of $\sim20$ kV/cm at 15 bar. The detection of EL light provides an energy measurement (in the energy plane, made of photomultipliers (PMTs), located behind the cathode) as well as providing tracking through its detection a few mm away from production at the anode plane, via a dense array of silicon photomultipliers called (the \emph{tracking plane}).

\subsubsection*{The NEW detector}
\label{sec.new}

%%%%%%%%%%
\begin{figure}
\centering
\includegraphics[height=9cm]{img/NEW.png}
\caption{The NEW apparatus.} \label{fig:NEW}
\end{figure} 

The NEW (NEXT-WHITE) apparatus\footnote{The name honours the memory of the late Professor James White, one of the key scientists of the NEXT Collaboration.}, shown in Figure \ref{fig:NEW} is the first phase of the NEXT detector to operate underground. NEW 
%has a triple goal:
%
%\begin{enumerate}
%\item {\bf Technology}: it will validate the technological solutions adopted by NEXT-100.
%\item {\bf Radiopurity}: it will allow the NEXT collaboration an extra step in the implementation of a radiopure detector.
%\item {\bf Physics}: it will demonstrate with measurements of the \BI\ and \TL\ lines, as well as with the measurement of the \bbtnu\ spectrum, the physics capabilities of NEXT-100.
%\end{enumerate}
%
is a scale 1:2 in size (1:8 in mass) of NEXT-100. The energy plane contains 12 PMTs (20 \% of the 60 PMTs deployed by NEXT-100). The tracking plane technology consists of 30 Kapton Dice Boards (KDB) deploying 1800 SiPMs (also 20\% of the sensors). The field cage has a diameter of 50 cm and a length of 60 cm (the dimensions of the NEXT-100 field cage are roughly 1 cm length and 1.2 m radius). 

NEW is a necessary step\footnote{As formally stated by the scientific committee of the LSC, who recommended its construction in 2013.} towards the construction of NEXT-100. It will validate the technological solutions adopted by the collaboration and, as discussed below, is essential in the definition of the project methodology. Furthermore, The NEXT background model is currently based on a sophisticated Monte Carlo simulation of all expected background sources in each part of the detector. NEW will allow the validation of the background model with the data themselves. 
%Last but not least, NEW operation will demonstrate with measurements of the \BI\ and \TL\ lines, as well as with the measurement of the \bbtnu\ spectrum, the physics capabilities of NEXT-100.

%Furthermore, the calibration of NEW with 
%sources of higher energy, will allow a precise study of the evolution of the resolution with the energy. 
%In particular it will be plausible to measure the resolution near \Qbb\ using a Thorium source, which provides 2.6 MeV gammas. Last, but not least, we intend to 
%reconstruct the spectrum of \bbtnu. Those events are topologically identical to signal events (\bbonu) and can be used to demonstrate with data the power of the topological signature. 
%
\subsubsection*{Discovery potential of NEXT-100}

The excellent resolution of NEXT (0.5 \% FWHM) and the combination of low radioactive budget and topological signature (which yields an expected background rate of $5 \times 10^{-4} \ckky$, will allow the NEXT-100 detector to reach a sensitivity on the \bbonu\ period of $\Tonu > 7 \times 10^{25}$~yr for a exposition of 300 kg$\times$yr. This translates in a range for \mbb\ of $[67-187]$~meV. Therefore NEXT-100 will have a substantial chance of making a discovery if the NME is sufficiently high. 

\subsubsection*{Towards a ton-scale high-pressure xenon TPC. BEXT}

%%%%%%
\begin{figure}
\centering
\includegraphics[width=0.55\textwidth]{img/levelscheme2.pdf}
\caption{The \BATA\ concept.} \label{fig.BATA}
\end{figure}
%%%%%%

If no discovery is made by the current generation of experiments, the full exploration of the CRR region (corresponding to the inverse hierarchy of neutrino masses) requires detectors of larger mass (at least 1 ton), good resolution and extremely low specific background. The \HPXE\ technology has the potential to provide the most sensitive detector in the ton scale, by scaling the detector to a mass in the range of the ton and adding additional handles to further suppress the background. 

One of the most promising possibilities is to develop the technology to unambiguously tag the barium ion produced in the xenon decay, $Xe \rightarrow Ba^{++} + 2 e$. The conceptual idea to tag $Ba^{+}$ is illustrated in Figure \ref{fig.BATA}. A ``blue'' laser of wavelength 493.54 nm excites (``pumps'') the S state, inducing $S \rightarrow P$~transitions, with a lifetime of $\sim$ 10 ns. About 30 \% of the times the \TwoP\ states decay to the state \TwoD, emitting ``red'' (649.76 nm) fluorescence in a characteristic time of 30 ns. The state \TwoD\ is metastable, but a second laser of suitable wavelength (4.1 $\mu$m) can be used to induce the transition to the ground state (this is known as ``deshelving'').  The whole cycle takes less than 50 ns, and therefore several millions of red fluorescence photons can be emitted by a single ion. 

Of course, the practical application of this beautiful conceptual idea is by no means easy, and in fact, it has been shown to be extremely difficult in liquid xenon by the work of the EXO collaboration\footcite{Dolinski:2012dta}. However, it may be feasible in an \HPXE\ detector, where a number of fortunate conditions may occur. These conditions are: a) charge reduction of the emitted barium ion, from $Ba^{++}$~to $Ba^{+}$, which can be induced by collisions with xenon atoms, or by the addition of a suitable quencher; b) ``trapping'' of the barium ion ``in situ'' by the surrounding Xe atoms, which result in a very low drift velocity for the ion; c) location of the ion, via the reconstruction of the event topology. 

All the above needs to be demonstrated with a systematic R\&D program, which must also address many other experimental issues such as pressure broadening of the laser, filtering of Rayleigh scattering, etc. Most importantly, such an experimental program must be carried out by an interdisciplinary group, combining the experience in laser spectroscopy and atomic physics, with the experience in \HPXE\ instrumentation.

The on-going collaboration between the IFIC (and other groups of NEXT) and the Center for Pulsed Lasers (CLPU) \footnote{http://www.clpu.es}, a national facility dedicated to ultra-intense lasers research and development has made possible to create precisely the interdisciplinary team needed for a successful R\&D program, which can culminate in a ``Barium-tagging Experiment with a Xenon TPC'' (BEXT). 
A future detector of 1 ton mass, with a resolution of 0.5 \% FWHM and a background rate in the range of $10^{-6} \ckky$~(thanks to the implementation of barium-tagging) would be able to fully cover the CRR (inverted hierarchy) region in less than 5 years run, assuming a favourable scenario for the NME. Even the most pessimistic scenario could be fully explored, however, with a longer run, since the sensitivity to period increases in this case (virtually background free experiment) linearly with exposure. 

Clearly the construction of a ton-scale \HPXE\ detector implementing a full \BATA\ technology is a very challenging enterprise. On the other hand, we believe that the incremental approach devised by the NEXT collaboration will also work in this case. The construction of the NEW detector is progressing without significant problems thanks to the expertise and know-how gained during DEMO phase, and we expect that NEXT-100 will fully benefit from the experience gained with NEW. Similarly, the \BATA\ technology could be ready in a period of 4 years (in parallel to the construction and commissioning of NEXT and ready to be used when NEXT enters operation) by approaching the problem step by step. 


%Si el proyecto es continuación de otro previamente financiado, individual o coordinado, deben indicarse con claridad los objetivos y los resultados ya alcanzados de manera que sea posible evaluar el avance real que se propone en el nuevo proyecto. Si el proyecto aborda un tema nuevo, deben indicarse los antecedentes y contribuciones previas de los equipos de investigación que justifiquen su capacidad para llevarlo a cabo.
%

This research project is the continuation of the CONSOLIDER-INGENIO project CUP (2010-2014), and the SEIDI project FIS2012-37947-C04 (2012-2014). 

The initial phase of the project (called the DEMO phase), included a learning period (2010-2011), needed to acquire the very innovative technology the detector is based on, and to equip state-of-the-art laboratories at IFIC and other participating institutions. 
During 2011 and 2012, the collaboration selected the technology to be implemented and a Technical Design Report\footcite{Alvarez:2012haa} was published in 2012. This phase of the project culminated  with the construction, commissioning and operation of the NEXT-DEMO prototype located at IFIC, and the NEXT-DBDM prototype operating at LBNL. Between 2012 and 2014, the results of the prototypes were analysed and published, showing the excellent performance (energy resolution, electron reconstruction) of the apparatus, as well as the robustness of the EL technology\footcite{Alvarez:2012hh, Alvarez:2012nd, Alvarez:2012hu,Alvarez:2013gxa,Lorca:2014sra}. 

\subsubsection*{NEXT prototypes}

\begin{figure}
\centering
\includegraphics[width=0.7\textwidth]{img/DemoSetup.jpg}
\caption{\small The NEXT-DEMO prototype setup at IFIC.} \label{fig.DEMO}
\end{figure}
%%%%%%%%%%

NEXT-DEMO, shown in figure \ref{fig.DEMO}, is as a large-scale prototype of NEXT-100. The pressure vessel has a length of 60 cm and a diameter of 30 cm. The vessel can withstand a pressure of up to 15 bar and hosts typically 1-2 kg of xenon. NEXT-DEMO is  equipped with an energy plane made of 19 Hamamatsu R7378A PMTs and a tracking plane made of 256 Hamamatsu SiPMs. 

The detector has been operating successfully for more than two years and has demonstrated: (a) very good operational stability, with no leaks and very few sparks; (b) good energy resolution ; (c) track reconstruction with PMTs and with SiPMs coated with TPB; (d) excellent electron drift lifetime, of the order of 20 ms.Its construction, commissioning and operation has been instrumental in the development of the required knowledge to design and build the NEXT detector.

The NEXT-DBDM prototype is a smaller chamber, with only 8 cm drift, but an aspect ratio (ratio diameter to length) similar to that of NEXT-100. The device has been used to perform detailed energy resolution studies. NEXT-DBDM achieves a resolution of 1\% FWHM at 660 keV and 15 bar, which extrapolates to 0.5\% at \Qbb.

\subsubsection*{Topological signature}

%%%%%
\begin{figure}
\centering
\includegraphics[width=0.9\textwidth]{img/Topo2.png}
\caption{\small NEXT has a topological signature, not available in most \bbonu\ detectors. The panel shows the reconstruction of a Monte Carlo signal (topleft) and background (bottomleft) event. The signal has two electrons (two blobs). The background has only one electron (one blob) and the associated emission of a 35 keV X-ray. The color codes energy deposition in the TPC. An scatter plot of the energy of the two blobs shows a clear separation between signal and background regions.}\label{fig.ETRK2}
\end{figure}
%%%%%

%%%%%%
%\begin{figure}
%\centering
%\includegraphics[width=0.9\textwidth]{img/ElectronsDataMC.png}
%\caption{\small A comparison between data (left) and Monte Carlo (right) for electrons of different energies recorded in the DEMO chambers, showing the very good agreement between both data sets and therefore the robustness of the topological signal, unique of the NEXT experiment.}\label{fig.ETRK3}
%\end{figure}
%

Double beta decay events leave a distinctive topological signature in HPXe: a continuous track with larger energy depositions (\emph{blobs}) at both ends due to the Bragg-like peaks in the d$E$/d$x$ of the stopping electrons (figure \ref{fig.ETRK2}, topleft). In contrast, background electrons are produced by Compton or photoelectric interactions, and are characterised by a single blob and, often, by a satellite cluster corresponding to the emission of $\sim30$-keV fluorescence x-rays by xenon (figure \ref{fig.ETRK2}, bottomleft).
Reconstruction of this topology using the tracking plane provides a powerful means of background rejection, as can be observed in the figure. 
%In our TDR we chose a conservative cut to separate double--blob from single--blob events which provided a suppression factor of 20 for the background while keeping 80\% of the signal.  DEMO has reconstructed single electrons from \NA\ and \CS\ sources, as well as double electrons from the double escape peak of \TL\, demonstrating the robustness of the topological signal. 

%
%Figure \ref{fig.ETRK3} shows a comparison between data and Monte Carlo for electrons interacting in the DEMO detector. Two radioactive sources were used: Na-22, producing single electrons of 511 keV, and Tl-208, whose double escape peak produced {\em double electrons}, at the energy of 1.6 MeV. Both data sets allow us to ``mimic'' signal and background and thus have a robust assessment of the performance of the topological signal comparing the Monte Carlo simulation and the actual results obtained with DEMO. The agreement between both data sets is very good, revealing the robustness of the topological signal. 

\subsubsection*{Energy resolution}

%%%%%
\begin{figure}
\centering
\includegraphics[width=0.8\textwidth]{img/EResolution.png}
\caption{\small Left: the full energy spectrum measured for electrons of 511 keV in the DEMO detector. Right the spectrum near the photoelectric peak for 662 keV electrons in NEXT-DBDM. The resolution at 662 keV is 1\% FWHM (0.5\% FWHM at \Qbb). The resolution extrapolated from 511 keV is 0.7\%.}\label{fig.ERES}. 
\end{figure}
%%%%

Figure \ref{fig.ERES} shows the resolution obtained with the NEXT-DBDM apparatus. A resolution of 1\% FWHM with 
662 keV photons, has been measured, which extrapolates to 0.5\% FWHM at \Qbb. This result is not far from the expected limit obtained adding in quadrature the different factors that contribute to the resolution (Fano factor, photoelectron statistics and electronic noise). The resolution measured in NEXT-DEMO extrapolates to 0.7\% FWHM. The difference between both prototypes is due to better photoelectron statistics and aspect ratio in DBDM. The results, are, in any case, better than the target of 1\% FWHM described in the TDR.

The status of the NEXT experiment and the results achieved by the prototypes have been described in a recent
paper \footcite{Gomez-Cadenas:2013lta}.



\subsubsection*{Objetivos generales y adecuación al Programa Estatal de I+D+i orientada a los Retos de la Sociedad / General objectives and match to the National Programme for research aimed at the Challenges of Society}

%\subsection*{Objectives and methodology}

%2. La hipótesis de partida y los objetivos generales perseguidos con el proyecto coordinado en su conjunto, así como la adecuación del proyecto a la Estrategia Española de Ciencia y Tecnología y de Innovación y, en su caso, a Horizonte 2020 o a cualquier otra estrategia nacional  o internacional de 

The overall objectives of this research proposal are:

\begin{enumerate}
\item Construction, commissioning and operation of the NEW and NEXT-100 detectors, during a period of 4 years, from 2015 to 2019.
\item Demonstrate the feasibility of barium tagging in an HPXe, performing a systematic set of small, focused, prove-of-concept experiments. As a part of this objective, an infrared laser using innovative technology will be developed. 
\end{enumerate}
  
The COORD subproject leads the construction of the NEW and NEXT-100 detectors, while the ENG subproject leads the deployment of the electronics, DAQ and slow controls. The CALREC subproject leads the calibration of the detector. The R\&D for barium tagging is lead by the BATA subproject (CLPU), with the participation of all the groups.   

The specific objectives of all the sub projects are integrated in the NEXT Project Management Plan. 
The PMP coordinates the construction of the NEW and NEXT-100 detectors. It is under the direct supervision of the Spokesperson (SP) and the Project Manager (PM). The PM of NEXT is Dr. I. Liubarsky, part of the IFIC group. 

The PMP defines a set of Working Packages (WP) and follows the progress of each one, monitors deliverables and dead lines and keeps track of invested resources including personnel. It also identifies potential show-stoppers and synergies (and possible conflicts) between the different projects and optimises the sharing of resources. 

%Figure \ref{fig.Gantt} shows an example of the Gantt chart for the whole NEW project, up to installaton at the LSC. 

%The objectives defined for the different sub projects match the various WP in the PMP, as can be seen in Figure \ref{Fig:PMP}. 

The methodology of each WP includes: a) the definition of the associated tasks; b) the identification of the resources needed; c) the temporal organisation of the tasks; d) the definition of milestones and the deliverables associated to them; e) the relations with other WP. Each WP has a leader, which reports directly to the PM. The progress of each WP is reviewed on a weekly basis. Milestones and potential showstoppers are discussed, and the tracking charts updated if needed. The PMP is reviewed every six months by the LSC scientific committee.  

The objectives presented in this project are very well aligned to the spanish program for science, as demonstrated by the fact that NEXT has been supported by the CONSOLIDER-INGENIO project CUP. The support of the AdG/ERC makes it clear that the projects suits perfectly well the goals of H2020. NEXT is a CERN recognised experiment and has been listed by NSA\footcite{NLDBD} as one of the key \bbonu\ experiments in the field, and the one with best future prospects.





\paragraph{Match to the National Programme for research aimed at the Challenges of Society.}
This research project is presented within the program of ``Challenges of society'', specifically, challenge number 6: {\bf Change and social innovation.}

We argue that this project represents a major innovation in the way that particle physics is conducted in Spain, and thus marks a path to a more productive approach to research.

Particle physics is a clear example of the so-called ``big science''. The discovery of the Higgs boson is a quintessential example of such big science. It has required the construction and operation of the LHC, one of the most impressive scientific machines ever built by humankind. The gargantuan scale of the effort could only be faced by a collective effort centralised at CERN, the largest particle physics laboratory in the world.  

Big science involves big budgets, often invested in purchasing equipment to be installed at CERN and in paying scientific staff whose activity also develops at CERN. Such large budgets are often justified in terms of industrial and scientific returns. While those returns certainly exist, {\bf they tend to be larger for countries who are already well developed scientifically}. Specifically, the positions of leadership in the large CERN experiments, and in the CERN scientific and technical divisions, are dominated by countries like Germany, Switzerland, U.K., France and Italy. 

Remarkably, the countries leading the big science at CERN and other laboratories have also developed ``national science'' physics programs. A case of great interest is Italy, a country not very different from Spain, in terms of GDP and social habits. However, the international impact and the returns of physics in Italy is much larger than in Spain. For example, the number of Spanish staff members at CERN is 115, to be compared with 275 for Italy (which has the second largest staff population, after France, who co-hosts the lab). Adding fellows and associates (that is, temporary CERN contracts, often given to scientists), the figures for Spain are 363, to be compared with 1726 for Italy\footcite{cernstats}. Several Italians have served as CERN general directors, and have led or are leading the LHC experiments. The next CERN general director (and perhaps the first woman to occupy such position in the history of the laboratory) may be the ex-spokesperson of ATLAS, the Italian physicist Fabiola Gianotti. Moreover, Italy has five Nobel prizes in physics (Marconi, 1909, Fermi, 1938, Segrè 1959, Rubbia 1984, Giacconi 2002), while Spain has none. 

Remarkably Italy also boasts the best underground laboratory of Europe, and one of the best of the world, the LNGS. The lab hosts 20 experiments including three searching for \bbonu\ processes (GERDA, CUORE and COBRA) and two experiments searching for Dark Matter (WARP and XENON). 

Through these experiments, the Italian physics {\bf attracts external talent} (some of the best physicists from Europe and the USA participate in experiments at the LNGS) {\bf and external funding}, complementing the big science at CERN with physics on a smaller scale concerning human resources and budgets. However, such ``local'' physics results in discoveries of great scientific impact (such as the discovery of neutrino oscillations, which has been the result of a world-wide effort involving underground laboratories in Italy, USA, Canada, Russia and Japan). It also allows the training of students and post-docs in experiments where young physicists can make a major impact at all levels, ranging from the construction of the detector to the analysis of the data. Last, but not least, such local science has an important impact in the Italian industry and in the appreciation of science by the public in general. 

{\bf We argue that, in order to balance and optimise the current big science effort in Spain, it is necessary to develop the physics at the LSC, in analogy to the Italian case.} NEXT is the flagship experiment of our national laboratory, and has achieved international recognition, as demonstrated by the fact that NEXT is a recognised CERN experiment and has obtained an AdG/ERC, {\bf the first grant of this type in Spain in the field of experimental particle physics}. 

We, therefore, consider that the NEXT project is a clear example of social innovation, as it has the potential of implementing profound changes in Spanish science. As described in this project, NEXT, through its various stages, can make a major discovery. It will bring international credit and visibility to our science and to the LSC. And it has an important impact both in local industry (through contracts to many national firms, and development of high technology) and in the public perception of science. 

Furthermore, the on-going collaboration with the CLPU further reinforces the above arguments, since the effort involves now a second national scientific installation. In addition, the \BATA\ program implies a major example of interdisciplinarity, and can result in a number of important technological returns (development of infra-red laser technology, which has a myriad of scientific and technological applications).

Finally it is important to remark that, while the usual operation of big science in Spain implies to finance the participation of our groups (including the annual CERN quota, the common fund of the experiments and the contributions to construction and operation of the CERN experiments), the national science that NEXT represents obtains external funding through ERC projects (including the AdG and several H2020 actions currently in progress involving LSC), as well as the contributions of the international collaboration to detector construction and operation (in particular, in the case of NEXT through the USA groups led by Prof. David Nygren, the inventor of the technology on which NEXT is based). NEXT also attracts external talent to our country, as the intense collaboration with top USA universities demonstrates. The NEXT group is very international, and several of our post-docs are or have been financed by EC grants (such as the Marie Curie). 

Last but not least, the NEXT experiment (and in particular the collaboration with the CLPU) involves the extensive development of photonics, listed as one of the  ``Facilitating Essential Technologies''.

\subsubsection*{Objetivos específicos / Specific objectives}
%

% Los objetivos específicos de cada uno de los subproyectos participantes, enumerándolos brevemente, con claridad, precisión y de manera realista (acorde con la duración prevista del proyecto).
%
% En los subproyectos con dos investigadores principales, deberá indicarse expresamente de qué objetivos específicos se hará responsable cada uno de ellos.
%

\subsubsection*{Objectives of the COORD subproject}

%\begin{figure}
%\centering
%\includegraphics[height=8cm]{img/PV.jpg}
%\caption{The pressure vessel of NEW (left) and NEXT-100 (right).} \label{fig:PV}
%\end{figure}

%%%%%
\begin{figure}[t!b!]
\begin{center}
\includegraphics[width=.9\textwidth]{img/FC3.jpg}
\end{center}
\caption{Left: The NEW field cage body made of HDPE, fabricated in Spain; right: The anode HVFT manufactured in Texas.
} \label{fig:FC}
\end{figure}

%%%%%
\begin{figure}[t!b!]
\begin{center}
\includegraphics[width=.9\textwidth]{img/KDBandPMT.jpg}
\end{center}
\caption{Top left: the flexible Kapton Dice Board (KDB) circuit developed by NEXT for the tracking plane; mid: the R11410-10 PMT from Hamamatsu; bottom right: a PMT can prototype.} \label{fig:sensors}
\end{figure}

The specific objectives of the COORD subproject are:

\begin{enumerate}
\item {\bf Construction of NEW}, foreseen to be completed in Q2'15, and involving:
\begin{enumerate}
\item {\em Construction of the NEW and NEXT-100 pressure vessels}.
The NEW and NEXT-100 pressure vessels, were designed to withstand pressures in excess of 20 bar, and to operate with negligible losses at 15 bar. They are built using a 316Ti alloy of low activity. 
The design was a collaboration between IFIC and LBNL groups. They have been manufactured by several Spanish companies (TRINOS, MOVESA and ACYM). They will be shipped to LSC in Q1'15 and commissioned during Q2'15. 

\item {\em Construction of the NEW field cage (NFC)}
The NEW field cage produces an uniform electric ($\sim$ 300 V/cm) field inside the  detector that drifts the ionisation electrons to the anode, where they are further accelerated in the electric field produced between a pair of transparent grids, called the electroluminescent (EL) grids. 

The main body of the field cage is a high density polyethylene (HDPE) cylindrical shell with a 2.5\,cm wall thickness (Figure \ref{fig:FC}, left).  The drift region consists of radiopure  copper strips connected with low radioactivity resistors.  The light tube consists of thin sheets of teflon, coated with tetraphenyl butadiene (TPB). The role of the TPB is to shift the UV light produced in xenon to the blue region (PMT and SiPMs operate best in this region, around 450 nm).  A high-voltage feedthrough (HVFT) in the cathode and another one in the anode, allow the definition of the voltages (Figure \ref{fig:FC}, right). The cathode HVFT is designed to withstand up to 100 kV, and the HVFT of the anode to withstand up to 40 kV. The design of the HVFT, identical for NEW and NEXT-100 improve those that were built for DEMO by the Texas A\&M group. The field cage and grids of NEW and NEXT-100 are also identical, to a scale 1:2. The grids use a stainless steel mesh with pitch 0.5 mm and wire diameter 30 microns, which results in an open area of 90\%. 

The construction of NFC is currently under way (Q4' 2014). The HDPE body has been fabricated by the spanish company AIMPLAS. The HVFT and the grids have been manufactured at Texas. The  NFC will be tested at IFIC before shipping and installation in the LSC in Q1'15. The field cage will be commissioned in Q2'15. System commission, together with the rest of the systems will occur during Q3'15 and Q4'15.

\item {\em Construction of the NEW energy plane (NEP)}
In NEW the energy measurement will be provided by the detection of EL light via PMTs, which will also record the scintillation light needed for $t_0$. Those PMTs will be located behind a transparent cathode.

A total of 12 low-background, high-QE PMTs, model R11410-10 from Hamamatsu (Figure  \ref{fig:sensors}, mid panel) covering 32.5\% of the cathode will be deployed. The R11410-10 are large tubes, with a 3'' photocathode and low levels of  activity of the order of 1 mBq per unit in the Uranium and Thorium series. The PMTs are sealed into individual pressure resistant, vacuum tight copper enclosures (called PMT cans) coupled to 
sapphire windows, coated with ITO (for electrical conductivity) and TPB (Figure  \ref{fig:sensors}, right panel).

The NEW energy plane (NEP)
is currently (Q4'15) under construction at IFIC. The ``PMT can'' design have been validated and production of the parts has started. The NEP will ship to LSC during Q1'15. It will be commissioned in Q2'15. System commission, together with the rest of the systems will occur during Q3'15 and Q4'15.

\item  {\em  Construction of the NEW tracking plane (NTP)}:
In NEW the tracking function is provided by a plane of multi-pixel photon counters (SiPMs) operating as a light-pixels and located behind the transparent EL grids. They are mounted in flexible radiopure Kapton Dice Boards (KDB). Each KDB hosts 64 SiPMs (Figure  \ref{fig:sensors}, left panel) .The NTP will deploy 28 such KDBs. 

The NTP is currently under construction at IFIC (Q4'14). The KDB production has been validated and prototypes have demonstrated excellent performance. The NEP will ship to LSC during Q1'15. It will be commissioned in Q2'15. System commission, together with the rest of the systems will occur during Q3'15 and Q4'15. 

\end{enumerate}
 
\item {\bf Commissioning of NEW and evaluation of performance}. The NEW detector will be brought online in Q2'15, and extensive system testing will be performed to certify safe and stable operation (no leaks, no sparks), as well as testing and integration of all the subsystems. We expect to complete commissioning in Q3'15.
During Q4'15, we will evaluate the performance of the detector. Such evaluation will allow us to correct for design problems (if they arise) or to introduce improvements in the engineering if needed. We will also assess the overall radioactive budget of the detector, to ensure the absence of ``hot spots'' (excess of radioactivity introduced accidentally in the detector). 

\item {\bf NEW physics run}. During 2016, we will operate continuously the NEW detector at the LSC. The physics runs of NEW has several goals: a) measurement, using radioactive sources, of the energy resolution as a function of the energy, and in particular at \Qbb (CALREC subproject) ; b) measurement, using radioactive sources, of single (``background'') electrons, as well as ``double electrons'' (produced by the double escape peak of Tl-208, and used to characterise the signal) (COORD and CALREC); c) measurement of the standard mode \bbtnu; and d) a full measurement of the spectrum, after selection cuts, thus quantifying, from the data themselves, the background model (collaboration wide studies). 
%

\item {\bf Construction of NEXT-100}. The fabrication of NEXT-100 will proceed through 2016, although some parts (such as the pressure vessel) have already been built. 
The construction will take 12 months. This fast schedule is possible thanks to several factors:
\begin{enumerate}
\item {\em Reuse of infrastructures}: the platform, pedestal, lead castle, gas system, clean tent, radon suppression system, online computing, slow controls and calibration hardware and procedures are common to NEW and NEXT-100 and will be extensively tested during NEW operation, thus ready for NEXT-100 phase.
\item {\em Early construction of pressure vessel}: The pressure vessel is a critical system, since it has to operate underground, at high pressure and with negligible losses. Designing, constructing and certifying it takes a long time. Luckily, this fact was understood at an early stage in the development of the project and the NEXT-100 pressure vessel (see Figure \ref{fig:PV}) was constructed during 2014, and has been tested and certified at the same time than the NEW pressure vessel. 
\item {\em Scalability of the field cage}: Some of the most delicate subsystems of the field cage (such as the HVFT) have been designed and tested to be operative in NEXT-100 (for example the HVFT in the cathode holds up to 100 kV voltage, while the nominal operation of NEXT-100 is 50 kV). The NEXT-100 field cage body will be constructed by the same company (AIMPLAS) that has built the NFC body, reusing the tools and procedures that have been put together for the task. This also applies to the construction of the EL grids. 
This makes possible to foresee an early assembly of the NEXT-100 field cage, in Q2'16, allowing for ample time for testing and debugging.

\item {\em Production chains for the energy and tracking planes}. The energy and tracking planes are composed of individual modules (PMT cans in the case of the energy plane, KDBs in the case of the tracking plane), mounted to supporting plates and connected to electronics. The structure of the systems is the same (as seen in Figure \ref{fig:EnergyPlane}). The number of modules (cans and KDBs) is larger in NEXT-100 (by about a factor 5), but, very importantly, the construction procedure is the same. This allow us to set {\em production chains}, PC, of both PMT cans and KDBs. The PCs will be extensively exercised during the construction of NEW, and are expected, therefore, to run smoothly during NEW construction.  

The production chains should be able to produce 20 PMT cans and 30 KDBs a month. We therefore, expect that the modules will be ready in Q1'16. Shipping and cleaning will take the best part of Q2'16. The systems should be assembled at the LSC in Q3'16, allowing for testing and debugging during Q4'16.
\end{enumerate}

\item {\bf Commissioning of NEXT-100}. The commissioning of NEXT-100 will benefit from the experience gained commissioning and operating NEW. We consider feasible to commission the detector during the first 2 quarters of 2017, but our project management plan allows for two extra quarters. The main reason is to guarantee enough time to run with normal xenon before circulating the precious enriched xenon in the gas system and the detector. Notice that the detector can be fully calibrated, and the backgrounds can be characterised with normal xenon.  

\item {\bf Physics run of NEXT-100}. The physics run may start in the third quarter of 2017, but the project plan foresees the first quarter of 2018. The calibration procedures are identical to those developed for NEW. After one year of run, NEXT-100 should reach the sensitivity of the current leading experiments. We currently foresee to run for three years (2018 to 2020), achieving a sensitivity to \mbb\ that makes a discovery possible if NME are sufficiently large and the neutrino is a Majorana particle. 

\end{enumerate}

\paragraph{Objectives of the ENG subproject.}

The ENG subproject centralises the front-end electronics, data acquisition (DAQ), online system and slow controls of the NEW and NEXT-100 detectors. It is coordinated by the UPV. NEXT (via the UPV team) has co-developed a new readout and DAQ concept named SRS\footcite{Toledo2011,SRS2013} for the international RD-51 collaboration at CERN. NEXT front-end modules are connected via copper links to the SRS DAQ interface modules. The CERN standard DATE environment is used as DAQ software. This brings a number of advantages, such as counting on a large base of users and developers, reducing production costs and profiting from other groups' developments. SRS has been successfully used in NEXT-DEMO (PMT and SiPM readout, DAQ interface and trigger modules)\footcite{Gil2012,Herrero2012,Esteve2012} and newer versions of these modules are to be used in NEW and NEXT-100\footcite{TWEPP2014}.

The specific objectives of this subproject are:

\begin{enumerate}

\item {\bf FEE (Front End Electronics)}. Design, fabrication and commissioning of the front-end electronics for the PMTs and the SiPMs for NEW and NEXT-100. The leader for this objective is the co-PI of the subproject, Prof. Jos\'e Toledo (UPV).
 
\item {\bf DAQ}. Design, fabrication and commissioning of the data acquisition modules for NEW and NEXT-100. The leader for this objective is the second co-PI of the subproject, Prof. Raul Esteve (UPV).

\item {\bf Slow control}. Design, fabrication and commissioning of the slow control for NEW and NEXT-100. The project leader is technical engineer Vicente Álvarez, under the supervision of co-PI J. Toledo. The goals are (1) monitor critical detector parameters, mostly temperature and pressure, (2) control power supplies for the sensors, detector grids and electronics and (3) implement an automatic emergency response monitor.

\item {\bf Online}. Design and commissioning of the online system for NEW and NEXT-100. The project leader is Prof. Raul Esteve (UPV). 

\end{enumerate}
\paragraph{Objectives of the CALREC subproject.}

The specific objectives of this subproject are:
\begin{enumerate}
\item {\bf Photo-detectors calibration system}. Design, construction, commissioning and operation of a calibration system to estimate the gain and noise of the SiPM and PMT sensors of NEW and NEXT-100. 

\item {\bf Energy calibration system}. Design, construction, commissioning and operation of an energy calibration system with external radioactive sources for NEW and NEXT-100.

\item {\bf Position calibration system}. Design, construction, commissioning and operation of a position calibration system using the X-rays produced by a $^{83}$Rb source. 

%\item {\bf Xenon additives}. Development of a dedicated testing system to evaluate additives capable to improve the performance of pure xenon. 

\end{enumerate}


\paragraph{Objectives of the BATA subproject.}

The \BATA\ subproject will focus on the R\&D program needed to clearly establish the feasibility of tagging (e.g., detecting) the barium (Ba$^{++}$) ion produced in the double beta decay of xenon. 
%Demonstrating that an efficient detection of barium is possible in an \HPXE\ would imply that the NEXT technology could be upgraded to the ton scale while at the same time reducing the background by two or more orders of magnitude, resulting in a virtually background-free experiment with enormous possibilities of hitting a discovery. 

The \BATA\ program involves a set of proof-of-concept experiments. It also proposes the development of a 4.1 $\mu$m laser, needed for the deshelving of the metastable state D. An attractive feature of such a laser is its wide range of scientific and technological applications. CLPU is especially well suited to develop the technology.

The different objectives of this subproject are, therefore:

\begin{enumerate}
	\item \textbf{Ba ions generation, phase 1}. Proof-of-principle experiment with Ba ions generated by means of an electrical discharge.
	
	\item \textbf{Ba ions generation, phase 2}. Proof-of-principle experiment with Ba ions generated by an ion source.	
	
%	\item \textbf{Proof of principle experiment with Ba ions generated by an ion source and with a magneto trap.}	
% In order to better control the experimental conditions, we will repeat the experiment but with the ions now confined in a magnetic trap. This trap will allow us to have an excellent degree of control over the experimental conditions and to approach the conditions that can be expected in the NEXT experiment. For instance we will carry out different measurements comparing the collected fluorescence signal as a function of the pressure of the Ba$^+$ ions and the pressure of the surrounding environment. These measurements are mandatory because the population dynamics is very sensitive to pressure, i.e., to collisions. 
%	
	\item \textbf{D state deshelving}. A likely scenario is that the collisional induced decay between the metastable state D and the ground state S is either not effective or too slow for obtaining an appreciable fluorescence signal. In this situation the population is trapped in the metastable state D and the fluorescence cycle can not be closed. To avoid this difficulty, deshelving the D state may be needed. A proof-of-principle experiment with an additional laser for deshelving the D state will be performed.

\item \textbf{Development of a state-of-the-art 4.1\,$\mu$m laser}. The laser needed for deshelving the D state must have a wavelength of around 4.1\,$\mu$m. While small commercial laser systems exist in this range (we will purchase one of them for the experiment described in the previous paragraph), no commercial system will satisfy the conditions of power and stability needed for a real \BATA\ experiment. Furthermore, such a laser, already well in the infrared region, has many potential applications.

\end{enumerate}

%\subsubsection*{\BATA\ subproject: Resources}
%
%For the successful development of this subproject, CLPU will provide the required human and technological resources. CLPU is the centre of reference in Spain regarding laser technology, and takes active part in several international and national projects. The leader of this subproject will be Alicia V. Carpentier who has a well recognised international trajectory in laser-matter interaction. Moreover, {\bf CLPU considers this project of high priority and consequently will offer the collaboration of all the scientific department} consisting of a multidisciplinar team with broad experience in laser technology and development, and laser-matter interaction. 
%
%Furthermore, CLPU will support this project with some of the already operating laser systems in its installation. This is extremely important because such systems usually cost of the order of several hundreds of thousand euros. The human resources needed to operate the laser systems will be provided by CLPU as well. For the construction of the ion source, and taking into consideration the specific requirements of this development, we will apply for an \emph{EXPLORA tecnología} in the 2014 call. 
%
%The budget of this subproject will be dedicated to: a) purchase small equipment for the proof-of-principle experiments; b) purchase a small commercial infrared laser for the initial deshelving experiment; c) develop a state-of-the-art, high power, very stable infrared laser to be used in a large system. It is important to insist that such a laser has many possible applications given the fact that its wavelength is not absorbed by the atmosphere as it lies in what is called the infrared atmospheric window.
%
%
%\subsubsection*{BATA subproject: schedule}
%
%
%


\subsubsection*{Metodolog\'ia / Methodology}

%4. El detalle de la metodologia propuesta en cada uno de los subproyectos participantes
The specific objectives of all the sub projects are integrated in the NEXT Project Management Plan. The PMP coordinates the construction of the NEW and NEXT-100 detectors. It is under the direct supervision of the Spokesperson (SP) and the Project Manager (PM). The PM of NEXT is Dr. I. Liubarsky, part of the IFIC group. 

The PMP defines a set of Working Packages (WP) and follows the progress of each one, monitors deliverables and dead lines and keeps track of invested resources including personnel. It also identifies potential show-stoppers and synergies (and possible conflicts) between the different projects and optimises the sharing of resources. 

%Figure \ref{fig.Gantt} shows an example of the Gantt chart for the whole NEW project, up to installaton at the LSC. 

%The objectives defined for the different sub projects match the various WP in the PMP, as can be seen in Figure \ref{Fig:PMP}. 

The methodology of each WP includes: a) the definition of the associated tasks; b) the identification of the resources needed; c) the temporal organisation of the tasks; d) the definition of milestones and the deliverables associated to them; e) the relations with other WP. Each WP has a leader, which reports directly to the PM. The progress of each WP is reviewed on a weekly basis. Milestones and potential showstoppers are discussed, and the tracking charts updated if needed. The PMP is reviewed every six months by the LSC scientific committee.  
\paragraph{Methodology of the COORD subproject.}

%\begin{figure}
%\centering
%\includegraphics[height=8cm]{img/PV.jpg}
%\caption{The pressure vessel of NEW (left) and NEXT-100 (right).} \label{fig:PV}
%\end{figure}

%%%%%
\begin{figure}[t!b!]
\begin{center}
\includegraphics[width=.9\textwidth]{img/FC3.jpg}
\end{center}
\caption{\small Left: The NEW field cage body made of HDPE, fabricated in Spain; right: The anode HVFT manufactured in Texas.
} \label{fig:FC}
\end{figure}

%%%%%
\begin{figure}[t!b!]
\begin{center}
\includegraphics[width=.9\textwidth]{img/KDBandPMT.jpg}
\end{center}
\caption{\small Top left: the flexible Kapton Dice Board (KDB) circuit developed by NEXT for the tracking plane; mid: the R11410-10 PMT from Hamamatsu; bottom right: a PMT can prototype.} \label{fig:sensors}
\end{figure}

The first objective of the COORD subproject is the {\bf construction of the NEW detector}. The construction of NEW will involve:

\begin{enumerate}
\item {\em Construction of the NEW and NEXT-100 pressure vessels}.
The NEW and NEXT-100 pressure vessels were designed to withstand pressures in excess of 20 bar, and to operate with negligible losses at 15 bar. They are built using a 316Ti alloy of low activity. 
The design was a collaboration between IFIC and LBNL groups. They have been manufactured by several Spanish companies (TRINOS, MOVESA and ACYM). They will be shipped to LSC in Q1'15 and commissioned during Q2'15. 

\item {\em Construction of the NEW and NEXT-100 field cage (NFC)}
The NEW and NEXT-100 field cage produces an uniform electric ($\sim$ 300 V/cm) field inside the  detector that drifts the ionisation electrons to the anode, where they are further accelerated in the electric field produced between a pair of transparent grids, called the electroluminescent (EL) grids. 

The main body of the field cage is a high density polyethylene (HDPE) cylindrical shell with a 2.5\,cm wall thickness (Figure \ref{fig:FC}, left, shows the NEW field cage body).  The drift region consists of radiopure  copper strips connected with low radioactivity resistors.  The light tube consists of thin sheets of teflon, coated with tetraphenyl butadiene (TPB). The role of the TPB is to shift the UV light produced in xenon to the blue region (PMT and SiPMs operate best in this region, around 450 nm).  A high-voltage feedthrough (HVFT) in the cathode and another one in the anode, allow the definition of the voltages (Figure \ref{fig:FC}, right). The cathode HVFT is designed to withstand up to 100 kV, and the HVFT of the anode to withstand up to 40 kV. The design of the HVFT, identical for NEW and NEXT-100 improve those that were built for DEMO by the Texas A\&M group. The field cage and grids of NEW and NEXT-100 are also identical, to a scale 1:2. The grids use a stainless steel mesh with pitch 0.5 mm and wire diameter 30 microns, which results in an open area of 90\%. 

The construction of FC for NEW is currently under way (Q4' 2014). The HDPE body has been fabricated by the spanish company AIMPLAS. The HVFT and the grids have been manufactured at Texas. The  NFC will be tested at IFIC before shipping and installation in the LSC in Q1'15. The field cage will be commissioned in Q2'15. System commission, together with the rest of the systems will occur during Q3'15 and Q4'15.

\item {\em Construction of the NEW and NEXT-100 energy plane }
In NEW (NEXT-100) the energy measurement will be provided by the detection of EL light via PMTs, which will also record the scintillation light needed for $t_0$. Those PMTs will be located behind a transparent cathode.

A total of 12 (60) low-background, high-QE PMTs, model R11410-10 from Hamamatsu (Figure  \ref{fig:sensors}, mid panel) covering 32.5\% of the cathode will be deployed for NEW (NEXT-100) . The R11410-10 are large tubes, with a 3'' photocathode and low levels of  activity of the order of 1 mBq per unit in the Uranium and Thorium series. The PMTs are sealed into individual pressure resistant, vacuum tight copper enclosures (called PMT cans) coupled to 
sapphire windows, coated with ITO (for electrical conductivity) and TPB (Figure  \ref{fig:sensors}, right panel).

The NEW energy plane (NEP) is currently (Q4'15) under construction at IFIC. The ``PMT can'' design have been validated and production of the parts has started. The NEP will ship to LSC during Q1'15. It will be commissioned in Q2'15. System commission, together with the rest of the systems will occur during Q3'15 and Q4'15.

\item  {\em  Construction of the NEW and NEXT-100 tracking plane (NTP)}:
In NEW and NEXT-100 the tracking function is provided by a plane of multi-pixel photon counters (SiPMs) operating as a light-pixels and located behind the transparent EL grids. They are mounted in flexible radiopure Kapton Dice Boards (KDB). Each KDB hosts 64 SiPMs (Figure  \ref{fig:sensors}, left panel) .The NEW (NEXT-100) tracking plane will deploy 28 (112) such KDBs. 

The NTP is currently under construction at IFIC (Q4'14). The KDB production has been validated and prototypes have demonstrated excellent performance. The NTP will ship to LSC during Q1'15. It will be commissioned in Q2'15. System commission, together with the rest of the systems will occur during Q3'15 and Q4'15. 

\end{enumerate}
 
%%%%%%

The main objective of the COORD subproject is the {\bf fabrication of NEXT-100}. The construction will start in 2016 and will take 12 months. This fast schedule is possible thanks to several factors:
%
\begin{enumerate}
\item {\em Reuse of infrastructures}: the platform, pedestal, lead castle, gas system, clean tent, radon suppression system, online computing, slow controls and calibration hardware and procedures are common to NEW and NEXT-100 and will be extensively tested during NEW operation, thus ready for NEXT-100 phase.
\item {\em Early construction of pressure vessel}: The pressure vessel is a critical system, since it has to operate underground, at high pressure and with negligible losses. Designing, constructing and certifying it takes a long time. Luckily, this fact was understood at an early stage in the development of the project and the NEXT-100 pressure vessel (see Figure \ref{fig:PV}) was constructed during 2014, and has been tested and certified at the same time than the NEW pressure vessel. 
\item {\em Scalability of the field cage}: Some of the most delicate subsystems of the field cage (such as the HVFT) have been designed and tested to be operative in NEXT-100 (for example the HVFT in the cathode holds up to 100 kV voltage, while the nominal operation of NEXT-100 is 50 kV). The NEXT-100 field cage body will be constructed by the same company (AIMPLAS) that has built the NFC body, reusing the tools and procedures that have been put together for the task. This also applies to the construction of the EL grids. 
This makes possible to foresee an early assembly of the NEXT-100 field cage, in Q2'16, allowing for ample time for testing and debugging.

\item {\em Production chains for the energy and tracking planes}. The energy and tracking planes are composed of individual modules (PMT cans in the case of the energy plane, KDBs in the case of the tracking plane), mounted to supporting plates and connected to electronics. The structure of the systems is the same (as seen in Figure \ref{fig:EnergyPlane}). The number of modules (cans and KDBs) is larger in NEXT-100 (by about a factor 5), but, very importantly, the construction procedure is the same. This allow us to set {\em production chains}, PC, of both PMT cans and KDBs. The PCs will be extensively exercised during the construction of NEW, and are expected, therefore, to run smoothly during NEW construction.  

The production chains should be able to produce 20 PMT cans and 30 KDBs a month. We therefore, expect that the modules will be ready in Q1'16. Shipping and cleaning will take the best part of Q2'16. The systems should be assembled at the LSC in Q3'16, allowing for testing and debugging during Q4'16.
\end{enumerate} 
\paragraph{Methodology of the ENG subproject.}

\begin{figure}[h!]
\begin{center}
\includegraphics[width=0.7\textwidth]{img/Electronics.jpg}
\end{center}
\caption{\label{Fig:FEE}\small Left: Front-end boards for SiPM (top) and PMTs (bottom) for NEW and NEXT-100. Right: SRS FEC modules in ATCA form factor (left) and “SRS classic” flavour (right).}
\end{figure}

The ENG subproject is responsible for the {\bf front-end electronics} for the PMTs and the SiPMs for NEW and NEXT-100. Critical items include:
%
\begin{enumerate}
\item	{\em Commissioning NEW front-end electronics in 2015}. As an outcome from two previous projects (CUP and FIS2012-37947-C04-04), the front-end electronics for both the PMT plane and the SiPM plane in NEW are available. The former is already being used in NEXT-DEMO while the latter, an evolution from the NEXT-DEMO electronics, exists as prototype boards and will be available in adequate quantities by the end of 2014. Required cabling, power supplies and mechanical structures have already been purchased.
These electronics will be tested at IFIC in Q1’15 and at the LSC in Q2’15. A study on reliability, performance and signal noise will be carried out during the commissioning phase, resulting either on the final validation of the current designs or on a proposal for changes. In this case, the changes will be carried out in Q4’15 and Q1’16.

\item {\em Production, installation and test of NEXT-100 front-end electronics in 2016}. Lessons learnt with NEW commissioning and initial operation will lead to a document on recommendations for installation and operation of electronics in NEXT-100. The specificity of LSC in terms of grounding and power distribution in the experimental area, together from noise in the power lines and generated by neighbouring experiments, will determine the required noise-reduction techniques (mostly ground connections and ad-hoc filtering). After completing the eventual modifications in the front-end modules (Q1'15), production for NEXT-100 in Q2’16, and Q3'16, module test in Q4’16 will complete the 2016 work plan.

\item {\em Commissioning and Operation of NEXT-100 front-end electronics in 2017}. Potential problems found after installation in a detector of the size of NEXT-100 will be addressed. Front-end firmware may require fine adjustments to (1) ease later energy measurement and tracking algorithms and (2) adjust the trigger algorithms as specified for the different physics campaigns. Commissioning is scheduled for Q1-Q2’17. Operations start on the second half of the year.
\end{enumerate}

%%%%%%%%%%%%%%%%%%%%%%

The second objective of this subproject is the design, fabrication and commissioning of the {\bf data acquisition modules} for NEW and NEXT-100. The following main tasks have been identified:

\begin{enumerate}

\item	{\em Commissioning NEW DAQ interface electronics in 2015}. The DAQ interface modules for NEW and NEXT-100 are also available in two flavours: “classic” SRS modules (19” Eurocard form factor, available from CERN Store) and industry-standard ATCA SRS modules. NEW DAQ will use in a first stage modules from both flavours.

These electronics will be tested at IFIC in Q1’15 and at LSC in Q2’15. A study on reliability and performance will be carried out during the commissioning phase, allowing to choose the best solution for NEW and NEXT-100. 
%A third option, a new SRS solution for very-close-to-detector readout codenamed OC-Box, which is to be developed in collaboration with RD51 and intended as an upgrade for NEXT, will be evaluated in 2015 as a future upgrade. It is based on newer FPGA technology and can replace current FEC modules.

\item	{\em Production, installation and test of NEXT-100 DAQ electronics in 2016}. Lessons learnt with NEW commissioning and initial operation will lead to a decision to go for “classic” or ATCA SRS flavor. Purchases for NEXT-100 in Q2’16, module test and firmware upgrade in Q3’16 and installation in Q4’16 will complete the 2016 work plan.

\item	{\em Commissioning and Operation of NEXT-100 DAQ electronics in 2017}. DAQ performance and functionalities will be verified and adjusted to meet final NEXT-100 requirements. DAQ firmware may require some modifications to (1) ease energy measurement and tracking algorithms and (2) adjust the trigger algorithms as specified.

%\item	NEXT-100 operation in 2018. The DAQ requires no maintenance during operation other than module replacement due to damage or malfunctioning. Placed outside the lead castle in 19” racks next to the front-end modules, the DAQ electronics are easily accessible.

\end{enumerate}

%%%%%%%%%%%%%%%%%%%%%%

Concerning the third objective of the ENG subproject, the {\bf slow controls system} for NEW and NEXT-100, six subsystems or partitions have been defined: (1) high-voltage sources for PMTs, (2) low-voltage sources for SiPMs and electronics, (3) high-voltage for grids, (4) gas system status -getters, pump and gas recirculation-, (5) a group of temperature and pressure sensors and (6) the main control panel (with cameras and status overview). Partitions are interconnected via Ethernet. Basic functionalities for all subsystems have been implemented in 2014. The Slow Controls for NEW will be completed in Q4’14, tested and debugged in Q1’15 and installed in LSC in Q2’15.

%Power supplies for PMTs and SiPMs are directly controlled to Ethernet. Power supplies for the electronics are controller from a PC via USB, being this PC connected to Ethernet. The other subsystems are controlled via a National Instrument’s Compact-RIO chassis with Ethernet link, embedded in a so-called “Slow Control Box”, which includes relays and connections. A LabView software application will be used for the main control panel.

%\begin{enumerate}
%\item {\em Design, production and commissioning NEW slow-controls in 2015}. The Slow Controls for NEW will be completed in Q4’14, tested and debugged in  Q1’15 and installed in LSC in Q2’15.

%\item {\em Production, installation and commissioning of NEXT-100 slow-controls electronics}. Monitoring the gas subsystem will be an addition to the Slow Controls for NEXT-100 in 2016.
%\end{enumerate}

%%%%%%%%%%%%%%%%%%%%%%

Finally, the ENG subproject is respnsible for the {\bf online system} for NEW and NEXT-100. The Online System for NEW comprises the following subsystems: (1) Data Storage, (b) Backup Storage, (c) Pre-processing and (d) Online Monitoring. The DAQ System (a PC farm comprising Local and Global Data Concentrators) produces approx. 30 MByte/s in NEW, which are stored into the NEW Data Storage (up to 8 servers with 8 disks each for up to 65 TByte). The Data Storage has power redundancy and multiple network cards per PC for enhanced reliability. Long-term data selected by the Offline System are stored into the Backup Storage (composed of a server PC and a tape server using 1.5 TByte tapes). The Pre-processing comprises two servers which read event data from the Data Storage and apply a format for later offline processing. Finally, a server in the Online Monitoring is used to gather statics, measure performance, find out bottlenecks and check the health of the different subsystems. A 1-Gb Ethernet switch interconnects the PC servers in the Online System, using virtual LANs to separate different data flows.
Backup Storage, Pre-Processing, Data Storage and Online Monitoring have been tested.
 
\paragraph{Methodology of the CALREC subproject.}

The {\bf gain and noise of the PMTs and SiPMs} will be calibrated using two sets of 400 nm LEDs located on the tracking and energy planes. The gain and noise of SiPMs and PMTs will be calculated for each sensor fitting their single photo-electron response. This procedure has been demonstrated in DEMO for the calibration of the energy plane\footcite{Lorca:2014sra}. In NEW, we will apply it also to the calibration of the last generation SiPMs deployed in the tracking plane, which have a much lower dark current than those deployed in DEMO (100 kHz per device to be compared with several MHz) and are, therefore, capable of counting single electrons.

The CALREC subproject is also responsible for the {\bf energy calibration of the NEW and NEXT-100 detectors}. The energy scale and energy resolution has been measured in DEMO using \NA\ and \CS\ radioactive sources\footcite{Alvarez:2012nd,Alvarez:2013gxa}. A fit to the photoelectric spectrum of the gammas interacting in the gas yields a gaussian distribution whose mean value sets the energy scale, while the RMS (or, as is customary in the field, the full width half maximum, FWHM) defines the resolution (see Figure \ref{fig.ERES}). In DEMO, the energy resolution has been measured at relatively low energies (511 and 662 keV), and the resolution in the region of \Qbb\ has been extrapolated using the scaling law $\delta E \propto 1/\sqrt{E}$, which has been demonstrated to hold well in the range between X-rays (35 keV) and \CS\ interactions (662 keV). 

The energy calibration in NEW and NEXT-100 will use the photo-peak of the 2.6 MeV gamma from a \Tl~  source to measure the energy resolution very near \Qbb. In addition, \NA\ and \CS~ sources will be used to calibrate the energy scale at  511 keV, 662 keV and 1.3 MeV. Adding the low energy point obtained with xenon X-rays (35 keV), we will be able to calibrate the energy in the full scale relevant for the experiment. The procedure is well understood from DEMO. The main novelty in NEW and NEXT-100 is the mechanical system, currently being designed, that will allow us to introduce and withdraw the radioactive sources from the chamber using special access ports. The calibration system will be built at US in Q1-Q2'15 and installed at the LSC in Q3'15.

The \TL\ source also produces double electrons (double escape peak of \TL) with an energy of 1.6 MeV, not very far from \Qbb. Such double electrons, together with single electrons at 1.3 MeV from a \NA\ source can be used to measure the single and double electron selection efficiency from the data themselves (recall that the selection of \bb\ events require a positive double-electron identification).

Calibration runs will be taken periodically. Typically a long run per month should be enough, but the exact procedure will be defined as part of the commissioning process of NEW and NEXT-100. 

The third objective of the CALREC subproject is the design, construction, commissioning and operation of a {\bf position calibration system}. Position calibration is needed to correct for the bias on the energy introduced by the geometry of the chamber. As demonstrated using NEXT-DEMO data\footcite{Lorca:2014sra}, PMTs detect less light for events outside the central region of the detector. In DEMO, X-rays from xenon (35 keV) have been used to correct the position dependence of the energy. The procedure takes advantage of the fact that X-rays behave like point-like sources. One can, therefore, reconstruct their position in the detector (using the tracking plane) and measure their apparent energy (using the energy plane). The apparent energy differs from point to point, and thus one forms a weight matrix  (obtained normalising to the energy measured in the center of the camber) that is used to correct point-by-point the energy measured in the chamber.

Computing the energy weight matrix requires many X-ray interactions {\em in all the chamber}. In DEMO, those X-rays were produced by de-excitation of the gas when using conventional (\NA, \CS) radioactive sources. The statistics that can be collected with such a procedures is limited, however, and both NEW and NEXT-100 require much larger statistical samples.

We have proposed, therefore, the use of X-rays from Krypton, a technology demonstrated in liquid xenon detectors\footcite{Kastens:2009rt}. A source of $^{83}$Rb
decays to the metastable state  $^{83}$Kr$^m$~ with a half-life of 86.2 days. The
$^{83}$Kr$^m$~subsequently decays via emission of 3.1 keV and 9.4 keV conversion electrons with a half-life of 1.83 hours. Thus, no long-lived isotopes are introduced in the xenon gas, and the fact that krypton is itself a noble gas guarantees no problems of chemical purity.  

The $^{83}$Rb~source is infused in a zeolite which is introduced in the gas system by adding a simple VCR cross (the zeolite itself is a small piece of 2 grams mass). The side arms of the the cross allow gas to flow through the chamber and a filter prevents the introduction of the zeolite into the gas system. The top arm of the cross allows the injection of the $^{83}$Rb~in the form of aqueous solution. A typical injection of 10 $\mu$l, discharged into the zeolite via a syringe yields 700 nCi of $^{83}$Rb. The rubidium itself stays attached to the zeolite, while the $^{83}$Kr$^m$~reaches an equilibrium rate in the range of several tens of kBq, enough to produce a very large statistical sample for calibration. Xenon is then circulated through the cross during several minutes, carrying the radioactive krypton gas with it.  

The rubidium source will be procured by the US group in 2015. The system will be commissioned in Q4'15 or Q1'16, depending of the progress of the energy calibration system. Extensive analysis of rubidium data during 2016 will yield the weight matrix for NEW. The same procedure will then be repeated for NEXT-100. 

\begin{figure}
\begin{center}
\includegraphics[width=0.99\textwidth]{img/TMA.png}
%\includegraphics{img/CALIB_LSC_sources.jpg}
\caption{\small Left: a small experimental setup to test gas mixtures. Right: initial results on scintillation yield obtained with the setup.}
\label{fig:additives}
\end{center}
\end{figure}

Finally, the ENG subproject will also be responsible for developing a dedicated testing system to evaluate {\bf additives capable to improve the performance of pure xenon}. The US will lead the R\&D focused in the studies of additives (for example TMA, trimethylamine) mixed with xenon, which will be carried out in collaboration with the Portuguese groups of Coimbra and Aveiro. These additives reduce the electron diffusion, which translate into a more confined trajectory.  In addition the scintillator light has  a 300 nm wavelength, easier to detect than the one of xenon (170 nm). Furthermore, some of the additives (including TMA) are capable of transferring the charge from the Ba$^{++}$~ion produced in a \bb\ xenon decay to
the Ba$^{+}$~ion which can be tagged (e.g., $Ba^{++} + A \rightarrow Ba^{+} + A^{+}$) where $A$~is the additive. 

Figure \ref{fig:additives} (left) shows a small setup to measure the effect of additives in the gas. A preliminary result using this setup shows that very small concentrations of TMA result in a very large increase of the scintillation yield. Larger concentrations, on the other hand, quench the yield. We plan a systematic campaign of measurements using several additives (TMA, TEA,CH$_4$~CF$_4$~and others) that will study at depth the effect on scintillation yield, attachment and resolution of different mixtures (including very small concentrations), and as a function of pressure. This campaign requires a modification of the setup to allow operation at higher pressure, as well as monitoring systems to control very small concentrations of the additive. 

The initial R\&D will be done with the simple setup already available, at pressures of up to 2-3 bar. During 2015, we will characterise different additives and choose the most suitable ones for NEXT. In 2016, the setup will be upgraded to take higher pressures and further studies will be performed up to operating pressures of 10-15 bar. In 2017 we plan to operate the DEMO detector with the chosen additive, to characterise the performance of the mixture in a large system. We will use our calibration protocol, described above, to measure electron tracks and energy resolution at different energies, to quantify the potential improvements of the mixture. 

In addition, we will carry an experiment, in coordination with the \BATA\ program  to measure the charge transfer $Ba^{++} + A \rightarrow Ba^{+} + A^{+}$, where $A$~is the chosen additive. The exact schedule of such an experiment will depend on the overall \BATA\ program, but we expect that it will be performed in 2016. 
\paragraph{Methodology of the \BATA\ subproject.}

In a first round of experiments we will excite resonantly the S$\leftrightarrow$P transition of {\bf Ba$^+$ ions generated by an electrical discharge} between two barium electrodes and will collect the fluorescence signal of the P$\rightarrow$D transition. Although this generation method is not ideal because several different species different from Ba ions will be generated (e.g., molecules like BaO or clusters), it does not need a major technological development. 

The laser source needed to resonantly excite the $Ba^{+}$ must have a wavelength of 493.5\,nm, not available in commercial lasers. {em The laser source will, therefore be produced at the CLPU}, optimising a tuneable Ti:sapphire laser at 987\,nm to obtain second harmonic generation (SHG) at 493.5\,nm. This setup allows the tuning of the wavelength and control the bandwidth of the laser which is necessary to precisely tune it to the transition frequency (e.g. to correct for pressure broadening and other effects). The IFIC group, on the other hand, has fabricated the test chamber needed for the experiments 
(see Figure \ref{fig:chamber}). 

It is expected that these initial set of experiments will provide valuable information about the population dynamics in Ba$^+$ ions, and the influence of the different homogenous and in-homogenous broadening mechanisms. 

The second objective of the \BATA\ subproject is to {\bf generate Ba ions by an ion source}. For this objective, in order to get a better approximation of the final conditions of NEXT experiment a source of ions will be designed and constructed at the CLPU. This ion source will be based on selective ionisation and mass spectrometry techniques, and it will allow an efficient selection of the desired target species (e.g, $Ba^{+}$~and $Ba^{++}$). With this setup we will be able to study the recombination process Ba$^{++}\rightarrow$Ba$^{+}$ and decide whether it can be induced by collisions with xenon atoms or requires an additive (see discussion in the objectives of the CALREC subproject). Depending on the results of the experiment, a magnetic trap can bee added to improve the experimental conditions. 

The third objective of the \BATA\ subproject is to perform a proof of principle experiment with an {\bf additional laser for deshelving the D state}. Our approach will be to use a second laser to induce a two photon transition (one photon is forbidden by selection rules, between the states D and S, see Fig.\,\ref{fig.BATA}). 

The laser needed for deshelving the D state must have a wavelength of around 4.1\,$\mu$m. The fourth objective of the \BATA\ subproject is the {\bf evelopment of a state-of-the-art 4.1\,$\mu$m laser}. While small commercial laser system exists in this range, (we will purchase one of them for the experiment described in the previous paragraph), no commercial system will satisfy the conditions of power and stability needed for a real \BATA\ experiment. Furthermore, such a laser, already well in the infrared region, has many potential applications.	

There are only a few laser systems that generate laser emission in the mid-infrared (MIR, from 2 to 10\,$\mu$m). There are lasers that emit in discrete wavelengths as gas lasers ($CO_2$, Xe-HE, He-Ne), chemical lasers (Hidrogen Fluoride, Deuterium Fluoride) and Dye lasers by Raman Shift. All this systems are big and complex, emit in relative low power and are involve the use of dangerous materials (chemicals, flammable gases and/or carcinogenic powders). 

A much more interesting alternative to reach the desired MIR wavelength is the use of an optically pumped solid state laser (OPSSL) system, by means of a specific doping of crystals with metal transition ions. Wavelengths of 2 and 2.9\,$\mu$m are available with $Tm^{3+}$, $Tm^{3+}$-$Ho^{3+}$ and $Er^{3+}$ active ions in crystalline matrices. Recent developments involve doping with $Cr^{2+}$ and $Fe^{2+}$ ions. This should allow to develop a laser system that emits in a broad range: 2.1 to 3\,$\mu$m and 3.7 to 5\,$\mu$m, respectively. Another option is to work with quantum cascade semiconductor systems potentially available from 3.8 to 9.5\,$\mu$m in a discrete range, i. e.  not only continuously. 

To develop a laser system around 4.1\,$\mu$m we need to study and evaluate the best optical parameters of some materials (crystalline matrices or semiconductors) that can be used as active laser materials.  This will allow us to design a laser cavity with the appropriate  optical components and devices (these should work in the MIR region). It will also allow us to maximise the efficiency of the laser system. The cavities are different for systems optically pumped (the case of crystals doped with $Cr^{2+}$ o $Fe^{2+}$) or electrically pumped (as the quantum cascade semiconductors). These cavities can be 2-, 3- or 4- folded mirror configurations depending on the advantages observed during the design using optical and numerical software. The characterisation of the laser emission and other related parameters will allow us  to improve the laser system to use in the NEXT experiment.  

\subsubsection*{Infraestructuras, equipamientos y propuesta de co-financiaci\'on (equpamientos y fungible) / Infrastructures, equipment and co-funding request (equipment and fungible)}

%5. La descripción de los medios materiales, infraestructuras y equipamientos singulares a disposición de los participantes que permitan abordar la metodología propuesta.
% Los objetivos específicos de cada uno de los subproyectos participantes, enumerándolos brevemente, con claridad, precisión y de manera realista (acorde con la duración prevista del proyecto).
%
% En los subproyectos con dos investigadores principales, deberá indicarse expresamente de qué objetivos específicos se hará responsable cada uno de ellos.
%

\subsubsection*{Equipment and Infrastructures: COORD, ENG and CALREC subprojects}

\begin{enumerate}
\item {\bf A state-of-the-art laboratory at IFIC}, developed and financed with funds from CUP. The DEMO detector is operating here. The laboratory includes a full gas system, infrastructures for vacuum and high pressure operation; high voltage and slow control systems; a full DAQ and computing system. 
\item {\bf A state-of-the-art laboratory electronics laboratory at IFIC}, which has made possible the very fast development of the tracking plane. 
\item {\bf A state-of-the-art electronics laboratory at UPV}, which has made possible the very fast development of the FEE and DAQ. Of particular interest is the collaboration of UPV with CERN, in the context of RD51 collaboration, one of the main reasons why NEXT has been declared CERN recognised experiment.
\item {\bf Computation resources at the US}, where there is a large cluster (Tier2 class) 
for distributed LHCb analysis, with about 1500 processor cores. The NEXT experiment, through the PI of the CALREC subproject will have access to this cluster for Monte Carlo simulation and data reconstruction. 
\item {\bf Experimental laboratory at the US}, which 
includes a clean room (30 m$^2$ class 100000) with an automatic ultrasonic wedge bonding machine. This laboratory is ideal to set the gas-additive program to be carried out by US. 
item {\bf Infrastructures at LSC}, which include: a) working platform and seismic pedestal; b) lead castle; c) gas system; d) clean tent; e) radon suppression system. In addition, the LSC provided general support for the experiments. 
\item {\bf A state-of-the-art facility for radio purity measurements}, located at the LSC. 
\end{enumerate}

\subsubsection*{Equipment and Infrastructures: BATA subproject}

For the successful development of this subproject, CLPU will provide the required human and technological resources. CLPU is the centre of reference in Spain regarding laser technology, and takes active part in several international and national projects.  Moreover, {\bf CLPU considers this project of high priority and consequently will offer the collaboration of all the scientific department} consisting of a multidisciplinar team with broad experience in laser technology and development, and laser-matter interaction. 

The CLPU will provide a substantial part of the required infrastructure for the \BATA\ subproject, including.
 
 \begin{itemize}
    \item \textbf{Laboratory room for all the experiments:} The experiments will be fully performed at the CLPU laboratories.
    
\item \textbf{Workshop service:} The CLPU has a mechanical workshop capable of manufacturing high level designs. 

\item \textbf{Electronics workshop:} Offers general electronics service. 

\item \textbf{Auxiliary services:} KHZ amplified Ti:Saphire laser (Spitfire, Spectraphysics, 7 mJ 100 fs), laser micro-machining station, Optical and Scanning electron microscopes (ZEISS), plist DAQ services, vacuum equipment and optical equipment.   

%\item \textbf{Data acquisition equipment:} Three oscilloscopes up to 1 GHZ (Tektronix), data acquisition computer, 3 GHZ RF spectrum analyzer (Tektronix). 
%
%\item \textbf{Vacuum equipment:} Primary pumps, turbo-molecular pumps, general vacuum hardware.
%
%\item \textbf{Temperature control equipment:} Thermoelectrical chiller, 
%
%\item \textbf{Optical equipment (general, all for vis/nir):} Laser spectrum analyser (500-1500 nm), Laser spectrum analyser (1000-2600 nm), photodiodes (Ge, Si), ultra-fast photodiodes (300 ps), CCD cameras, two Laser Beam profilers (Gentec), optical autocorrelator (up to 50 fs, sweep), home-made optical autocorrelator (single shot), alignment lasers (HeNe), shutters, chopper (Stanford), filters, lenses, mirrors, general optomechanics (posts, post holders, clamping forks, waveplates, diaphragms, etc), two infrared viewers (up to 1350 nm), optics and optomechanics (mirrors, lenses, LBO crystals for 800 and 1030 nm, posts, clamping forks, mounts, etc.)

\item \textbf{Lasers:} a 493.54 nm CW LASER: Pump laser (Coherent, Verdi G 20 W, 532 nm, CW); a tunable Ti:Saphire laser CW (750-1020 nm,  more than 4W).
%
%\item \textbf{Spectrometer (UV-VIS-NIR, MIGHTEX)}
%
%\item \textbf{Power and energy meters:} thermal and pyroelectrical. 

\end{itemize}



\subsubsection*{Budget requested for equipment and fungibles for NEXT: COORD, ENG and CALREC subprojects}

The bulk of equipment and fungible requested by this co-ordinated project is devoted to the second phase of the experiment (the construction, commissioning and calibration of the NEXT-100 detector). 
 Infrastructures for the experiment are co-funded by CUP, LSC and the AdG/ERC, the costs of NEW are fully funded by the AdG/ERC grant, and the contribution of the USA groups also co-funds the construction of NEXT0100. 

\begin{table}[h!]
\begin{center}
\begin{tabular}{|l|c|c|c|c|c|c|}
\hline
 Item & Total Cost \euro & CUP	&USA &	LSC & AdG &	FIS2014 \\
 \hline
Infrastructures 	& 1.875.487 & 	131.702 & 	0 &	1.306.600 &	437.185 &	0 \\
NEW &	 689.189 & 	298.678 & 	0 &	0 &	390.511 &	0 \\	
NEXT0100	 &1.982.427 & 	368.203 &	413.457 &	0 &	0 &	1.200.767 \\
Computing (Online) & 0 & 0 & 	0 &	0 &	69.522 &	0 	\\
Slow Control & 18.271 & 	0 &	0 &	 &	0 & 18.271	\\
Calibration &	 0 & 	0 &	0 &	0 &	0  & 60.700	\\
 \hline
Total  NEXT &	 4.695.595 & 	798.583 &	413.457 &	1.306.600 &	897.218 &	1.279.738 \\
 \hline\hline
\end{tabular}  
\caption{Total costs of the NEXT project.}
\label{tab.TCOSTS}
\end{center}
\end{table} 

%\begin{table}[h!]
%\begin{center}
%\begin{tabular}{|l|c|c|c|c|c|c|}
%\hline
% Item & Total Cost \euro & CUP	&USA &	LSC & AdG &	FIS2014 \\
% \hline
%Gas System &	 434.177 & 	68.970 &	365.207 &	0 &	0 &	0 \\
%Platform and Castle	& 250.600 & 	0 &	0 &	0 &	250.600 &	0 \\
%Xenon (100 kg + 100 kg) &	 1.056.000 & 	0 &	0 &	0 &	1.056.000 &	0 \\
%Lead cleaning	& 44.582 & 	44.582 &	0 &	0 &	0 &	0 \\
%Cleaning equipment	& 18.150 & 	18.150 &	0 &	0 &	0 &	0 \\
%Clean tent	 59.878 & 	0 &	59.878 &	0 &	0 &	0 \\
%Radon suppression &	 5.400 & 	0 &	5.400 &	0 &	0 &	0 \\
%Radon monitoring &	 6.700 & 	0 &	6.700 &	0 &	0 &	0 \\
%Total infrastructures 	& 1.875.487 & 	131.702 &	437.185 &	0 &	1.306.600 &	0 \\
% \hline\hline
%\end{tabular}  
%\caption{Total costs of the infrastructures.}
%\label{tab.TINFRA}
%\end{center}
%\end{table} 
%
%\begin{table}[h!]
%\begin{center}
%\begin{tabular}{|l|c|c|c|c|c|c|}
%\hline
% Item & Total Cost \euro & CUP	&AdG &	LSC & USA &	FIS2014 \\
% \hline
%Pressure vessel &	 186.669 & 	 126.445 & 	 60.224 & 	 0 & 	 0& 0 \\ 
%Inner copper shield	& 32.670 & 	 0 & 	 32.670 & 	 0 & 	 0 & 	 0 \\ 
%Energy plane	& 131.271 & 	 88.572 & 	 42.699 & 	 0 & 	 0 & 	 0 \\ 
%Tracking plane	& 106.518 & 	 0& 	 106.518 & 0 & 	 0 & 	 0\\ 
%field cage	& 78.009 & 	 0 & 	 78.009 & 	 0 & 	 0 & 	 0 \\ 
%FE electronics	& 83.661 & 	 83.661 & 	 0 & 	 0 & 	 0 & 	0 \\ 
%DAQ and online &	 70.392 & 	 0 & 	 70.392 & 	 0 & 	 0 & 	 0 \\ 
%Total NEW&	 689.189 & 	 298.678 & 	 390.511 & 0 & 	0 & 0 \\ 
% \hline\hline
%\end{tabular}  
%\caption{Total costs of the NEW detector.}
%\label{tab.TNEW}
%\end{center}
%\end{table} 
%
\begin{table}[h!]
\begin{center}
\begin{tabular}{|l|c|c|c|c|c|c|}
\hline
 Item & Total Cost \euro & CUP	&AdG &	LSC & USA &	FIS2014 \\
 \hline
 Pressure vessel &	 277.332 & 	 102.850 & 	 0 & 	 0 & 	 0 & 	 174.482 \\ 
Inner copper shield &	 187.550 & 	 0 & 	 0 & 	 187.550 & 	 0 & 	 0 \\ 
Energy plane	& 625.318 & 	 265.353 & 	 0 & 	 123.420 & 	 0 & 	 236.545 \\ 
Tracking plane	&  287.237 & 	 0 & 	 0 & 	 102.487 & 	 0 & 	 184.750 \\ 
field cage	 & 184.344 & 	 0 & 	 0 & 	 0 & 	 0 & 	 184.344 \\ 
FE electronics	& 277.871 & 	 0 & 	 0 & 	 0 & 	 0 & 	 277.871 \\
DAQ and online &	 142.775 & 	 0 & 	 0 & 	 0 & 	 0 & 	 142.775 \\ 
Total NEXT-100	 & 1.982.427 & 	 368.203 & 	 0 & 	 413.457 & 	 0 & 	 1.200.767 \\ 
  \hline\hline
\end{tabular}  
\caption{Total costs of the NEXT-100 detector.}
\label{tab.TN100}
\end{center}
\end{table} 
 
Table \ref{tab.TCOSTS} summarises the total costs (equipment and fungible) of the project and details the funds requested. The amount is 1.279.738 \euro, which corresponds to 27\% of the equipment costs of the project and matches the external funds, profiled by AdG/ERC and USA contributions. 
Table \ref{tab.TN100} details the costs of NEXT-100 and the sharing of co-funding. 

\subsubsection*{Budget requested for equipment and fungibles for \BATA: BATA\& CALREC subprojects}

\begin{table}[h!]
\begin{center}
\begin{tabular}{|l|c|}
\hline
 Item & Cost \euro  \\
 \hline
Laser and nonlinear crystals for MIR range &  9.000 \\
Optical pumping systems & 30.000 \\
Laser diode driver & 10.000  \\
Optical components &14.000 \\
Opto-mechanics: &10.000 \\
MIR detectors & 3.500 \\
Electronic components & 1.500\\
Fungible & 4.000 \\
  \hline
 Total &  82.000 \\
 \hline \hline
\end{tabular}  
\caption{Costs of the MIR laser.}
\label{tab.MIR}
\end{center}
\end{table} 

The BATA subproject requests co-funding to purchase the optical and opto-mechanical components needed to tune up the blue laser that will be devoted to the experiment (20.000 \euro) , and to develop the solid state MIR laser (82.000 \euro), see table \ref{tab.MIR}.  
The total requested in equipment and fungibles is 102,000 \euro.

The CALREC subproject will use existing infrastructures for the study of additives. Only fungible (gauges, filters, getters, piping) adding to 10,000 \euro\ and funding to purchase gas additives (10,000) \euro\ are requested to this project. 

\subsubsection*{Cronograma / Timetable}



\subsubsection*{Schedule/Cronogram of NEXT activities}

Table \ref{tab:schedule_new} and Table \ref{tab:schedule_next} show the cronogram for the construction, commissioning and operation of the NEW and NEXT-100 detectors. The activities involve 3 of the 4 coordinated projects. 

\begin{table}[h!]
\begin{center}
\begin{tabular}{| l | c | c | c | c |}
\hline
tasks & 2015 & 2016 & 2017 & 2018 \\
\hline
Pressure Vessel & & & &   \\
\hline
Installation at LSC & Q1 & & & \\
Commissioning at LSC & Q2 & & & \\
\hline
Field Cage, Energy plane \& Tracking plane & & & &   \\
\hline
Tests at IFIC & Q1 & & & \\
Installation at LSC & Q2 & & & \\
Commissioning at LSC & Q3 Q4 & & & \\
\hline
Electronics and DAQ & & & &   \\
\hline
Tests at UPV & Q1 & & & \\
Installation at LSC & Q2 & & & \\
Commissioning at LSC & Q3 Q4 & & & \\
\hline
Calibration & & & &   \\
\hline
Design calibration setup at US & Q1 & & & \\
Commissioning at LSC & Q2 & & & \\
Operation at LSC & Q3 Q4 & & & \\
\hline
Physics run & & Q1-Q4& &   \\
\hline
\hline
\end{tabular}
\caption{Chronogram for the NEW detector.}
\label{tab:schedule_new}
\end{center}
\end{table} 

\begin{table}[h!]
\begin{center}
\begin{tabular}{| l | c | c | c | c |}
\hline
tasks & 2015 & 2016 & 2017 & 2018 \\
\hline
Pressure Vessel & & & &   \\
\hline
Installation at LSC &  & & Q1& \\
Commissioning at LSC & & &Q1 & \\
\hline
Field Cage, Energy plane \& Tracking plane & & & &  \\
\hline
Fabrication at IFIC \& Texas & & Q1-Q4 & & \\
Installation at LSC &  & &Q1 & \\
Commissioning at LSC &  & & Q3-Q4& \\
\hline
Electronics and DAQ & & & &   \\
\hline
Upgrade & & Q1 & & \\
Fabrication & & Q2-Q4 & & \\
Installation at LSC & & & Q1& \\
Commissioning at LSC & & & Q3-Q4 & \\
\hline
Calibration & & & &   \\
\hline
Upgrade & & Q4& & \\
Commissioning & & & Q1 & \\
Operatoin & & & Q2-Q4 & \\
\hline
Physics run & & & &  Q1-Q4 \\
\hline
\end{tabular}
\caption{Chronogram for the NEXT-100 detector.}
\label{tab:schedule_next}
\end{center}
\end{table} 



\subsubsection*{Schedule}

Table \ref{tab:schedule_calibration} shows the cronogram for the tasks of the different objectives presented on the  BATA subproject.

\begin{center}
\begin{tabular}{| l | c | c | c | c |}
\hline
tasks & 2015 & 2016 & 2017 & 2018 \\
\hline
\hline
\multicolumn{5}{|l|}{Blue laser}  \\
\hline
\hline
design & already done & & &  \\
construction & Q1 & & & \\
commissioning & Q2-Q3& & & \\
operation &  Q1-Q4 & Q1-Q3 & & \\
\hline
\hline
\multicolumn{5}{|l|}{Proof of principle experiment with Ba ions generated by means of an electrical discharge}  \\
\hline
\hline
construction  &  Q1 & & & \\
operation &  Q1-Q2 & & & \\
analysis &  Q1-Q3 & & & \\
\hline
\hline
\multicolumn{5}{|l|}{Proof of principle experiment with Ba ions generated by an ion source}  \\
\hline
\hline
design & Q3-Q4 & & &  \\
installation  &  Q4 & & & \\
operation &  Q4 & Q1 & & \\
analysis &  Q4 & Q1-Q2 & & \\
\hline
\hline
%\multicolumn{5}{|l|}{Proof of principle experiment with Ba ions generated by the ion source plus a magneto trap}  \\
%\hline
%\hline
%design & Q3-Q4 & & &  \\
%installation  &  & Q1 & & \\
%operation &  & Q1-Q3 & & \\
%analysis &  & Q1-Q4 & & \\
%\hline
%\hline
\multicolumn{5}{|l|}{Proof of principle experiment with an additional laser for deshelving the D state}  \\
\hline
\hline
installation  &  & Q1 & & \\
operation &  & Q1-Q3 & & \\
analysis &  & Q1-Q4 & & \\
\hline
\hline
\multicolumn{5}{|l|}{Infrared laser}  \\
\hline
\hline
acquisition and characterisation of components  & Q1-Q2 & & &  \\
design  & Q2-Q4  & & & \\
construction &  & Q1-Q4 & & \\
test and improvement &  & & Q1-Q4 & \\
test for NEXT & & &  & Q1-Q4\\
\hline
\end{tabular}
\label{tab:schedule_calibration}
\end{center}


\subsubsection*{Personal / Personnel}

%7. Si se solicita ayuda para la contratación de personal, justificación de su necesidad y descripción de las tareas que vaya a desarrollar.
The COORD subproject is lead by the IFIC, the largest group in NEXT. The IFIC group includes: The PI (a CSIC professor, and spokesperson of NEXT). The analysis co-coordinator (Dr. Sorel, R\&C). Dr. P. Novella (R\&C); the technical coordinator (Dr. I. Liubarsky); the collaboration expert in coating techniques (Dr. N. Yhalali), the project leader (PL) of the construction of the energy plane (Dr. Renner); the PL of tracking plane (senior engineer J. Rodriguez); the PL of field cage (Dr. March); the PL of mechanics (pressure vessel, infrastructures), senior engineer S. Carcel. The run coordinator (Dr. Laing); the reconstruction and Monte Carlo coordinator (Dr. Ferrario) and 6 graduate students. In addition the group has two technical electronics engineering developing essential components of the tracking and energy plane, and one technical mechanical engineering, with expertise in mechanical design. 

The PI and the two R\&C positions are covered by CSIC. The positions of one senior physicist (Liubarsky), two post-docs (Renner, March), and two of the three technical engineers positions, will be covered by the AdG. Dr. Laing and Dr. Ferrario are applying to independent grants (young researcher modality). 

COORD requests funding for three {\bf essential} positions. Senior physicist Yhalali, in charge of the coating laboratory (4 years). Senior electronics engineer J. Rodríguez, the PL of the NEXT tracking plane construction for NEW and NEXT-100, developer of the NEXT KDBs, and responsible of the NEXT-IFIC electronics laboratory (4 years). And senior mechanical engineer S. Cárcel, the PL of the pressure vessel and NEXT infrastructures and integration coordinator (4 years). In addition we request funding for 3 years for one technical engineer (a mechanical designer).  

For the ENG subproject we request two positions.

A technical electronics engineer (3 years), whose tasks will be: a) carry out installation, maintenance, repairing and support of the front-end and DAQ interface electronics for NEW and NEXT-100, as well as the power supplies for sensors and electronics; b) will also take part in the design and construction of small circuits (like filters or cabling); c) will help in the commissioning phases and in the coordination of the electronic modules purchases and production and will carry out functional tests of the new units. 

A computer scientist/engineer, expert in Linux systems administration and Ethernet computer networks (4 years). Will configure, administer, maintain and support the different PC clusters (mostly running CERN Scientific Linux) in the DAQ, Online and Offline systems. Will configure the DATE environment for the DAQ system. Will scale these systems according to specific needs (like the upgrade from NEW to NEXT-100). Will manage the data storage. Will carry out R\&D programs to enhance performance. Construction and commissioning. 

For the CALREC subproject we request one post-doc position (4 years) and one technical engineer position (3 years). The post-doc will assist the PI in the essential tasks of setting up the calibration system with sources and with krypton; carrying out the calibration runs for the sensors, energy and position, including data taking, analysis, and maintenance of calibration data base. Develop energy and tracking reconstruction algorithms, using calibration data to assess and improve the performance of the detector. Last but not least, the post-doc will assist the PI in the R\&D program for gas additives. The role of the technical engineer is to help in the mechanical design of the calibration sources, the modifications to the NEXT gas system needed to introduce the rubidium source, and the setting up of the gas additives laboratory. 

For the BATA subproject we request one post-doc position (4 years) to assist the PI and the rest of the CLPU team in the experimental program (setting up the experimental lab, tuning up the blue laser, assist in the experiments, analysis of data, and participate in the development of the IR laser), and one technical electronics engineer (3 years) to assist in the electronics tasks associated to the development of the IR laser.  


\vspace{12pt}

\noindent\textbf{C.3. IMPACTO ESPERADO DE LOS RESULTADOS / EXPECTED IMPACT}

\subsubsection{Scientific and technological impact}

This project involves the construction of detectors which are unique in the World, implementing the HPXe technology with EL readout, widely considered as one of the most promising ones in the field of \bbonu\ searches. Therefore: 
\begin{enumerate}
\item {\bf The NEXT experiment has the potential of being in the forefront of a major scientific discovery}. The sensitivity of NEXT-100 will reach that achieved by GERDA, EXO and KamLAND-Zen in 2018, without being yet background-saturated. Consequently, from 2018 onwards, NEXT will probe a region of \mbb\ yet unexplored. If the NME is sufficiently large, NEXT could register \bbonu\ events, thus making (or participating in the making, together with other leading experiments of the field) a major discovery.
\item {\bf The NEXT will bring innovation to the industry-science relation in Spain}. The experiment uses and develops high technology, involving national and international firms. We have written statements of interest by companies in the mechanical sector (AIMPLAS, ACEX), gas and pressure (Swagelock, SERA), and light sensors (SENSL, Hamamatsu), among others. We are starting join R\&D  projects with a number of them (radio-pure SiPMs with SENSL, UV sensitive SiPMs with Hamamatsu, new plastic materials with AIMPLAS).  
\item {\bf Our approach to \BATA\ is opening a new inter-disciplinary field}, which incorporates elements of atomic, nuclear, particle physics, laser-matter interactions and photonics. It involves the technology of \HPXE\ chambers, ion sources, magnetic-traps and visible as well as IR lasers. There is a clear potential to develop experimental techniques associated to our experiments that go beyond our particular needs and have wide applications.
\end{enumerate}

\subsubsection{Dissemination of results}

The results of the experiment will be amply advertised. NEXT and CLPU keep modern web pages, and a firm presence in social media. In addition, the PI of NEXT is the science-advisor of the well known JotDown magazine\footcite{JotDown}, and develops an intense outreach activity which involves interviews to scientific personalities\footcite{JotDownNygrenBettini, JotDownCattaiGonzalez, JotDownHalzen} as well as a well read scientific blog\footcite{JotDownBlog}.

\subsubsection{Technology transfer}

Among the potential applications, scientific and industrial returns, we will mention: a) the development of solid-state MIR laser technology, and b) the development of matrices of sensors (SiPMs) and the associated software suitable for medical imaging. 
\begin{enumerate}
\item Lasers in the MIR spectrum are invisible to the human eye and not absorbed by the atmosphere, having them obvious applications in the surveillance industry. At the same time, IR also enhances night vision equipment capabilities by flooding forward positions with invisible light that enhances light gathering performance and increasing the ability to detect objects at greater lengths. 
\item The dense SiPM arrays used in the NEXT tracking plane, together with the associated electronics and software, may have direct applications to novel imaging systems in medicine capable of better resolution and/or operating with lower doses. This is due to the capabilities of SIPMs of reconstructing 3D positions for dim signals. 
\end{enumerate}
%C.3. IMPACTO ESPERADO DE LOS RESULTADOS 
%El contenido de este apartado se solicitará para cada uno de los subproyectos en la aplicación informática de solicitud (con un máximo de 3500 caracteres) y su contenido podrá ser publicado a efectos de difusión si el proyecto resultara financiado en esta convocatoria.
%En este apartado deberá detallar los siguientes aspectos para el proyecto coordinado en su conjunto.
%
%Se recomienda incluir:
%
%1. Descripción del impacto científico-técnico social y/o económico que se espera de los resultados del proyecto coordinado, tanto a nivel nacional como internacional.
%
%2. El plan de difusión e internacionalización en su caso de los resultados del proyecto coordinado.
%
%3. Si se considera que puede haber transferencia de resultados, se deberán identificar los resultados potencialmente transferibles y detallar el plan previsto para la transferencia de los mismos.

\vspace{12pt}

\noindent\textbf{C.4. CAPACIDAD FORMATIVA DEL EQUIPO / TRAINING CAPABILITIES OF THE GROUP}
\subsubsection{\label{subsubsec:training}Training plan}

This coordinated project requests 2 Ph.D. fellowships, one for the COORD subproject and one for the CALREC subproject. The students will be enrolled in the Nuclear and Particle Physics doctorate programs at University of Valencia (COORD) and University of Santiago de Compostela (CALREC). As part of their training, students will attend at least one international and one national school. Students will also spend a fraction of their time in international institutions participating in NEXT, in particular Coimbra, LBNL and Texas University. 

This coordinated project is highly multi-disciplinary, offering the students the capability of cross-training and of developing skills in a variety of areas, including laser-matter interactions, sophisticated data reconstruction with algorithms suitable for medical physics, and state-of-the-art instrumentation. Students in this coordinated project will rotate between different areas of work (and groups) to acquire a wide background, before settling onto a specific topic. 

The NEXT collaboration has recently approved a publication model in which the main authors of the analysis are the first authors (as opposed to the general trend in the field of particle physics, which uses alphabetical order). The goal of this policy is to facilitate the visibility of graduate students and young postdocs, and to encourage them to take leading roles in the development of the experiment and the analysis of the data. As an example, in \footcite{Lorca:2014sra}, the first two authors are graduate students and the other two post-docs.

Finally, the construction, commissioning and operation of NEXT is a great opportunity for graduate students, who typically take leading roles in important parts of the experiment.

\subsubsection{Supervision of Ph.D. theses and career development of former Ph.D. students}

\paragraph{COORD subproject.}
The IFIC group has currently two Ph.D. students about to finish in 2014 (J. Martin-Albo and F. Monrabal), two students to defend their Ph.D. thesis in 2015 (Lorca and Serra), and two to finish in 2016 (Nebot and Simo). 

Prof.~G\'omez-Cadenas (COORD co-PI) has advised a total of 11 students: De Fez (94), Lozano (95), Hernando (98), Cervera (02), Vidal (03), Burguet (08), Tornero (08), Novella (09), Catala (14) Martin-Albo (14) and Monrabal (14)\footnote{Medida de las fracciones de desintegración topologicas del leptron Tau; Estudio del canal electrónico de desintegración del Lepton TAU en LEP ; Busqueda de oscilaciones numu-nutau en el experimento Nomad del CERN; Diffractive Production de A1+Mesons by neutrinos in Nomad; The HARP time projection chamber; Study of neutral pion production via neutrino-induced, charged-current interactions in the K2K scibar detector; Physics of the Nutrino Factory and related long baseline designs;Experimental studies of neutrino nature: from K2K to Supernemo; Measurement of Neutrino Induced charged current neutral pion production cross section at Sciboone; The NEXT experiment for neutrinoless double beta decay searches; Demonstration of electroluminescent TPC technology for neutrinoless double beta searches using the NEXT-DEMO detector.}. Dr. Sorel (COORD co-PI) has co-advised Tornero and Catala. De Fez, Lozano, Hernando and Cervera have obtained academic positions (Emirates, Granada, U. of Santiago and Valencia). Novella has been a Marie Curie fellow and has just re-joined the group with a RyC position. Burguet works in a software company developing Internet products and Tornero has a permanent position in medical physics (radiotherapy). Catala and Vidal have continued their careers as teachers. Martin-Albo and Monrabal have already offers for post-doc position in the USA.

For the future, we plan to enrol 2-3 students in the next few years, replacing those that graduate. One Ph.D. fellowship is requested for this subproject (see Sec.~\ref{subsubsec:training}) and two others will be made available through other national and international grants. The group includes several experienced senior physicists, such as Prof. G\'omez-Cadenas, Dr. Sorel, Dr. Liubarsky, Dr. Yahlali and Dr. Novella. 

\paragraph{CALREC subproject.}

The CALREC PI has advised two high-impact Ph.D. thesis in the LHCb. Martínez Santos
(2010) and Cid Vidal (2012)\footnote{Search for the rare decays $B_s \to \mu^+\mu^-$
and $K_S \to \mu^+\mu^-$ in the LHCb with 1 fb$^{-1}$ integrated luminosity; Search for the very rare decay $B_s \to \mu^+\mu^-$ in the LHCb experiment}. Both students continue in research were they are following bright careers.   
Dr. Mart\'inez Santos was research fellow at CERN and is now a postdoctoral associate at NIKHEF institute, Amsterdam, as well as CERN Corresponding Associate. He is the coordinator of the $B_s$ mixing phase, $\phi_s$, (the second main LHCb result after the $B_s \to \mu^+\mu^-$ search). He was awarded in 2013 with the Young Experimental Physicist Prize by the European Physical Society (EPS) for his work at the LHCb trigger and the search of the $B_s \to \mu^+ \mu^-$ decay. Dr. Cid Vidal is currently a CERN fellow. He is currently working on the identification of the Higgs boson to b,b-bar jets at LHCb and leading the strange mesons physics at LHCb. He has recently presented a Marie Curie ITN  (International Training Network) proposal to extend kaon physics, and to apply multivariate methods used in HEP into other fields, for example to study the evolution of financial markets.
 



%Este apartado solo se rellenará si alguno de los subproyectos participantes solicita la inclusión del proyecto en la convocatoria de “Contratos predoctorales para la formación de doctores”. Dicha inclusión solo será posible en un número limitado de los proyectos aprobados.
%
%Para evaluar la capacidad formativa del equipo solicitante, se recomienda incluir:
%
%1. El plan de formación previsto.
%
%2. Relación de tesis realizadas o en curso (últimos 10 años) con indicación del subproyecto, nombre del doctorando, el título de tesis y la fecha de obtención del grado de doctor o de la fecha prevista de lectura de tesis.
%
%3. Breve descripción del desarrollo científico o profesional de los doctores egresados de los equipos de investigación de los subproyectos participantes.

%%%%%%%%%%%%%%%%%%%%%%%%%%%%%%%%%%%%%%%%%%%%%%%%%%%%%%%%%%%%%%%%%%%%%%%%%%%%%

\vspace{12pt}

\noindent\textbf{C.5. IMPLICACIONES ÉTICAS Y/O DE BIOSEGURIDAD/ETHICS AND SAFETY IMPLICATIONS}

None

\noindent\textbf{APPENDIX: DETAILED COSTS OF THE NEXT PROJECT}

\section{\bf Costs of the NEXT project}
\label{next.costs}

In this appendix we provide a detailed account of the costs of the project, including details about the funding source. The appendix presents the global costs of the project. Notice that the coordinated project distributes the costs associated with NEXT between three sub-projects, COORD, ENG and CALREC. The subproject COOR (which is also the coordinator) includes the bulk of  the costs of detector construction. The costs of front end electronics, data acquisition and slow controls are included in ENG, and the costs of calibration in CALREC. 

The costs associated with the R\&D for Barium Tagging, a sub-project (BATA) led by the CLPU are described in the main part of the project. 

\subsection{Costs of the NEW detector}
Table \ref{tab.new:DET} summarises the total costs of the NEW detector as well as the funding sources. {\bf AdG} refers to the Advanced Grant ERC granted to the PI of this proposal. {\bf CUP} refers to the CONSOLIDER INGENIO grant of which the PI of this proposal is co-coordinator.

Each subsystem is costed in the subsequent tables. Table \ref{tab.new:PV} summarises the costs of the pressure vessel, 
table \ref{tab.new:ICS} the costs of the inner copper shielding,
table \ref{tab.new:EP} the costs of the energy plane,
table \ref{tab.new:TP} the costs of the tracking plane,
table \ref{tab.new:FC} the costs of the field cage,
table \ref{tab.new:FEE} the costs of the front-end electronics and
table \ref{tab.new:DAQ} the costs of the online and data acquisition. The NEW detector has
been fully payed by CUP and AdG grants.

The costs detailed in the tables are very accurate, since most of the components have already been acquired. 
  
\begin{table}[h!]
\begin{center}
\begin{tabular}{|l|c|c|c|}
\hline
 System & Total \euro & CUP & AdG  \\
 \hline
 Pressure vessel 	& 186,668 &	12,6445 &	60,223 \\
Inner copper shield	& 32,670	& 0 &	32,670 \\
Energy plane	& 131,270 &	88,572 &	42,698 \\
Tracking plane	& 82,318 &	0 &	82,318 \\
field cage	& 78,009 &	0 &	78009 \\
FE electronics &	83,661 &	83,661 &	0\\
DAQ and online &	70,391 &	0 &	70,391 \\
 \hline
{\bf Total NEW} &	{\bf 664,989 }& 	298,678 & 	366,311 \\	
 \hline\hline
\end{tabular}  
\caption{Costs of the NEW detector.}
\label{tab.new:DET}
\end{center}
\end{table} 

\begin{table}[h!]
\begin{center}
\begin{tabular}{|l|c|c|}
\hline
 Concept & \euro & Funding Source \\
 \hline
 Tools &	2,144 &	AdG \\
Gaskets & 24,200 &	AdG \\
Carts	& 12,100 &	AdG\\
Machining	 & 21,780 & 	AdG\\
Vessel	& 96,195 &	CUP \\
End-cups	& 15,730 &	CUP \\
Vacuum Pump	& 14,520 & CUP \\
 \hline
{\bf Total}	& {\bf186, 669 }& \\	
Total CUP	& 126,445 & \\	
Total AdG	& 60.223 & \\	
 \hline\hline
\end{tabular}  
\caption{Costs of the NEW pressure vessel.}
\label{tab.new:PV}
\end{center}
\end{table} 

\begin{table}[h!]
\begin{center}
\begin{tabular}{|l|c|c|}
\hline
 Concept & \euro & Funding Source \\
 \hline
 Copper stock &	18,150 &	AdG \\
Machining & 14,520 &	AdG \\
 \hline
{\bf Total} &	{\bf 32, 670} & \\		
Total AdG	& 32,670 & \\	
 \hline\hline
\end{tabular}  
\caption{Costs of the NEW inner copper shield.}
\label{tab.new:ICS}
\end{center}
\end{table} 

\begin{table}[h!]
\begin{center}
\begin{tabular}{|l|c|c|}
\hline
 Concept & \euro & Funding Source \\
 \hline
Support plate	&	14,520 &	CUP \\
PMT cans &	25,250 &	AdG\\
Feedtrhoughs &	14,520 &	AdG \\
R11410-10 (12) &	74,052 &	CUP \\
PMT Bases &		2,928 &	AdG\\
  \hline
{\bf Total}	&	{\bf 131,271}	& \\
Total CUP	&	88,572	&\\
Total AdG	&	42,699 & \\	
 \hline\hline
\end{tabular}  
\caption{Costs of the NEW energy plane.}
\label{tab.new:EP}
\end{center}
\end{table}

\begin{table}[h!]
\begin{center}
\begin{tabular}{|l|c|c|}
\hline
 Concept & \euro & Funding Source \\
 \hline
SiPMs MicroFC-10035-SMT-GP &	29,814 & AdG \\
DICE-Boards SLK-1	&	4,356 & AdG \\
LEDs \& sensors	&	24 & AdG \\
Connectors FX11	&	871 & AdG \\
Inner Cables	&	4,840 & AdG \\
Screws	&	3,630 & AdG \\
Adapter Boards &	4,840 & AdG \\
External Cables &	5,505 & AdG \\
External cables shielding	&	847,00 & AdG \\
SiPM Power Supply Components	& 2,420 & AdG \\
SiPM Power Supply Cables	& 7,260 & AdG \\
Support plate  & 12,100 &  AdG \\
 Feedthrough PCB	&	2,178 & AdG \\
Feedthrough mechanics &	 24,200 & AdG \\
  \hline
{\bf Total}	&	{\bf 82.318,24 }	& \\
  Total AdG	&	82.318,24 	& \\
 \hline\hline
\end{tabular}  
\caption{Costs of the NEW tracking plane.}
\label{tab.new:TP}
\end{center}
\end{table} 

\begin{table}[h!]
\begin{center}
\begin{tabular}{|l|c|c|}
\hline
 Concept & \euro & Funding Source \\
 \hline
 Light tube & 12,877 & AdG \\
 Drift resistor chain & 8,258 & AdG \\
 Buffer resistor chain & 4,386 & AdG\\
 Poly body & 19,844 & AdG \\
 Field shaping rings & 4,451 & AdG \\
 Gate prototype & 4,477 & AdG \\
 Gate & 12,705 & AdG \\
 HVFT \& meshes & 11,011 & AdG \\
  \hline
{\bf Total} &	{\bf 78,009}	& \\
  Total AdG	&	78,009	& \\
 \hline\hline
\end{tabular}  
\caption{Costs of the NEW field cage.}
\label{tab.new:FC}
\end{center}
\end{table} 

\begin{table}[h!]
\begin{center}
\begin{tabular}{|l|c|c|}
\hline
 Concept & \euro & Funding Source \\
 \hline
 PMT FEE & 990 & CUP \\
 SiPM FE boards:  components	&	43,337 & CUP \\
SiPM FE boards: PCB manufacturing &	4,374 & CUP \\
SiPM FE boards: prototypes &	2,815 & CUP \\
SiPM FE boards: component mounting &	4,704 & CUP \\
Cables from FE to DAQ interface &	493 & CUP \\
SiPM FE power supplies & 19,766 & CUP \\
19" crates + fan cooling units	& 1,133 & CUP \\
100 ft power supply cable AWG14 &	783 & CUP \\
SiPM FE board design &	5.263 & CUP \\
  \hline
{\bf Total}	&	{\bf 83,6661}	& \\
 Total CUP	&	83,661	& \\
 \hline\hline
\end{tabular}  
\caption{Costs of the NEW front end electronics.}
\label{tab.new:FEE}
\end{center}
\end{table} 

\begin{table}[h!]
\begin{center}
\begin{tabular}{|l|c|c|}
\hline
 Concept & \euro & Funding Source \\
 \hline
FECs + rear module vATCA &	15,730 & AdG \\
FEC v6 TRG module		&	2,420 & AdG \\
ADC Cards vATCA & 	1,542 & AdG \\
Digital Mezzanine vATCA & 1,452 & AdG \\
Chassis vATCA 6-slot	&	10,527 & AdG \\
GbE CAT6 cables		& 145 & AdG \\
Optic SFP Modules - GbE	& 1.319 & AdG \\
GbE CAT6 cables &	1,815 & AdG \\
PCs (LDC/GDC) & 16,940 & AdG \\
Double port 10Gb &	2,831 & AdG \\
SAI	APC	&	9,680 & AdG \\
Swicth 1GbE	& 4,356& AdG \\
Rack SX 24U & 1, 633,50 & AdG \\
  \hline
{\bf Total} &	{\bf 70,391}	& \\
 Total AdG	&	70,391	& \\
 \hline\hline
\end{tabular}  
\caption{Costs of the NEW DAQ.}
\label{tab.new:DAQ}
\end{center}
\end{table} 




\subsection{Costs of the NEXT-100 detector}
Table \ref{tab.n100:DET} summarises the total costs of the Next-100 detector as well as the funding sources. {\bf FIS2014} refers to our proposal of co-funding submitted to the ``Retos de la Sociedad" I+D+i program. {\bf AdG} refers to the Advanced Grant ERC granted to the PI of this proposal. {\bf CUP} refers to the CONSOLIDER INGENIO grant of which the PI of this proposal is co-coordinator. {\bf USA} refers to funds committed by the USA groups.

Each subsystem is costed in the subsequent tables. Table \ref{tab.n100:PV} summarises the costs of the pressure vessel, 
table \ref{tab.n100:ICS} the costs of the inner copper shielding,
table \ref{tab.n100:EP} the costs of the energy plane,
table \ref{tab.n100:TP} the costs of the tracking plane,
table \ref{tab.n100:FC} the costs of the field cage,
table \ref{tab.n100:FEE} the costs of the front-end electronics and
table \ref{tab.n100:DAQ} the costs of the online and data acquisition. The NEW detector has
been fully payed by CUP and AdG grants.

The costs detailed in the tables are  accurate, since the price of most of the components are estimated directly from the costs of the components already purchased for NEW. 
  
\begin{table}[h!]
\begin{center}
\begin{tabular}{|l|c|c|c|c|}
\hline
 Cost &	Total& 	CUP & USA & FIS2014 \\
 \hline
Pressure vessel &	277,332 & 102,850  &	174,482 & 0 \\
Inner copper shield &	187,550 &	0 & 187,550 & 0 \\
Energy plane	& 625,317 &	265,353	&	123,420	&	236,544 \\
Tracking plane	& 287,237 &	0 & 102,487	& 184,750 \\
field cage	& 184,343 &	0	& 0	&	184,343 \\
FE electronics	& 277,870 &	0 &	0 &	277,870 \\
DAQ and online &	142,775 & 	0	& 0	& 142,775 \\
 \hline
{\bf Total NEXT-100} &	{\bf1,982,426 }& 368,203& 	413,457 & 	1,200,766 \\	
 \hline\hline
\end{tabular}  
\caption{Costs of the NEXT-100 detector.}
\label{tab.n100:DET}
\end{center}
\end{table} 

\begin{table}[h!]
\begin{center}
\begin{tabular}{|l|c|c|}
\hline
 Concept & \euro & Funding Source \\
 \hline
Tools &	24,200 & FIS2014 \\
Carts &	 36,300 & FIS2014 \\
Vessel	& 102,850 & CUP \\
Gaskets 	& 27,104 & FIS2014 \\
Main flange &	24,200 & FIS2014 \\
Bolts & 8,470 & FIS2014 \\
Adaptor CF-DN	 &	16,940 & FIS2014 \\
Connexion Gas System	& 18,150 & FIS2014 \\ 
VCR gaskets	& 968 & FIS2014 \\ 
End-Cup & 18,150 & FIS2014 \\ 
\hline
{\bf Total}	& {\bf277,332 } & \\	
Total CUP	& 102,850 & \\	
Total FIS2014	& 174,482 & \\	
 \hline\hline
\end{tabular}  
\caption{Costs of the NEXT-100 pressure vessel.}
\label{tab.n100:PV}
\end{center}
\end{table} 

\begin{table}[h!]
\begin{center}
\begin{tabular}{|l|c|c|}
\hline
 Concept & \euro & Funding Source \\
 \hline
 Copper stock &	158,510 &	USA \\
Machining & 29,040 &	USA \\
 \hline
{\bf Total} &	{\bf 187,550} & \\		
Total USA	& 187,550 & \\
Total FIS2014	& 0 & \\	
 \hline\hline
\end{tabular}  
\caption{Costs of the NEXT-100 inner copper shield.}
\label{tab.n100:ICS}
\end{center}
\end{table} 

\begin{table}[h!]
\begin{center}
\begin{tabular}{|l|c|c|}
\hline
 Concept & \euro & Funding Source \\
 \hline
Support plate	&	58,080 &	FIS2014 \\
PMT cans &	108,216 &	FIS2014 \\
Feedtrhoughs &	60,000 & FIS2014 \\
R11410-10 (12) &	388,773	& CUP+USA \\
PMT Bases &		10,248 &	FIS2014 \\
  \hline
{\bf Total}	&	{\bf 625,317}	& \\
Total CUP	&	265,353	&\\
Total USA	&	123,420 & \\
Total FIS2014	&	236,544 & \\	
 \hline\hline
\end{tabular}  
\caption{Costs of the NEXT-100 energy plane.}
\label{tab.n100:EP}
\end{center}
\end{table}

\begin{table}[h!]
\begin{center}
\begin{tabular}{|l|c|c|}
\hline
 Concept & \euro & Funding Source \\
 \hline
SiPMs MicroFC-10035-SMT-GP & 102,487 & USA \\
DICE-Boards &15,246 & FIS2014 \\
LEDs \& sensors &	85 & FIS2014 \\
Connectors FX11 & 2,439 & FIS2014 \\
Inner Cables & 16,940 & FIS2014 \\
Screws	& 14,520 & FIS2014 \\
Adapter Boards	 &	16,940 & FIS2014 \\
External Cables &	22,022 & FIS2014 \\
External cables & 3,388 & FIS2014 \\
SiPM Power Supply Components & 9,680 & FIS2014 \\
SiPM Power Supply Cables &	14,520 & FIS2014 \\
Plate  TP:  copper stock &  27,830 & FIS2014 \\
Plate  TP:  manufacturing & 18,150 & FIS2014 \\
Plate  TP:  bolts & 	18,150 & FIS2014 \\
Feedthroughs & 4,840 & FIS2014 \\
  \hline
{\bf Total}	&	{\bf 287,237 }	& \\
  Total USA	&	102,487 	& \\
   Total FIS2014	&	184,750 	& \\
 \hline\hline
\end{tabular}  
\caption{Costs of the NEXT-100 tracking plane.}
\label{tab.n100:TP}
\end{center}
\end{table} 

\begin{table}[h!]
\begin{center}
\begin{tabular}{|l|c|c|}
\hline
 Concept & \euro & Funding Source \\
 \hline
 Light tube & 20,570 & FIS2014 \\
 Drift resistor chain & 8,621 & FIS2014 \\
 Buffer resistor chain & 12,856 & FIS2014\\
 Poly body & 44,700 & FIS2014 \\
 Field shaping rings & 17,182 & FIS2014 \\
 Gate &45,980 & FIS2014 \\
 HVFT \& meshes & 34,364 & FIS2014 \\
  \hline
{\bf Total} &	{\bf 184,343}	& \\
  Total FIS2014	&	184,343	& \\
 \hline\hline
\end{tabular}  
\caption{Costs of the NEXT-100 field cage.}
\label{tab.n100:FC}
\end{center}
\end{table} 

\begin{table}[h!]
\begin{center}
\begin{tabular}{|l|c|c|}
\hline
 Concept & \euro & Funding Source \\
 \hline
 PMT FEE & 5130 & FIS2014 \\
 SiPM FE boards: front panels	& 1,332 & FIS2014 \\
SiPM FE boards: components	&	152,266 & FIS2014 \\
SiPM FE boards: PCB manufacturing	&	12,942  & FIS2014 \\
SiPM FE boards:  mounting	&	18,728 & FIS2014 \\
Cat-6 RJ45 cables &	1,731 & FIS2014 \\
SiPM FE power supplies & 73,416 & FIS2014 \\
19" crates  &	5,666 & FIS2014 \\
100 ft power supply cable &	2,663 & FIS2014 \\
Rack19" 42U height x 600 mm deep &	3,993 & FIS2014 \\

  \hline
{\bf Total}	&	{\bf 277,870}	& \\
 Total CUP	&	277,870	& \\
 \hline\hline
\end{tabular}  
\caption{Costs of the NEXT-100 front end electronics.}
\label{tab.n100:FEE}
\end{center}
\end{table} 

\begin{table}[h!]
\begin{center}
\begin{tabular}{|l|c|c|}
\hline
 Concept & \euro & Funding Source \\
 \hline
FECs v6	&	41,140 & FIS2014 \\
ADC Cards	&	5,082 & FIS2014 \\
CDTC16 v2	&	11,858 & FIS2014 \\
Crate Eurocard 19" 	&	363 & FIS2014 \\
Cat-6 RJ45 cables & 266 & FIS2014 \\
Fan cooling units	&	847 & FIS2014 \\
Power supply & 	16,456 & FIS2014 \\
Power supply connectors &	242 & FIS2014 \\
Rack for Eurocard Crate	&	1,210 & FIS2014 \\
Optic SFP Modules &	4,484 & FIS2014 \\
cables from FEC to PC &	617 & FIS2014 \\
PCs (LDC/GDC)	&	33,880 & FIS2014 \\
Double port 10Gb DA/SFP & 6,606 & FIS2014 \\
SAI	APC	 &	12,100 & FIS2014 \\
Swicth 1GbE	& 4,356 & FIS2014 \\
Rack  SX 24U 	&	3,267 & FIS2014 \\						
  \hline
{\bf Total} &	{\bf 142.775,16 €}	& \\
 Total FIS2014	&	142.775,16 €	& \\
 \hline\hline
\end{tabular}  
\caption{Costs of the NEXT-100 DAQ.}
\label{tab.n100:DAQ}
\end{center}
\end{table} 




\subsection{Costs of the NEXT infrastructures}
The operation of NEXT at the LSC requires extensive infrastructures. In addition to the xenon gas, owned by the LSC (100 kg enriched and 100 kg natural), the laboratory has built the working platform, seismic pedestal and lead castle needed to host the experiment. The NEXT experiment provides the gas system needed to recirculate and clean the gas. Such system is expensive, given the safety requirements, and has been purchased with AdG funds. The AdG also provides funds to buy the clean tent, radon suppression and monitoring system and miscellaneous expenses for a total of some
437 k\euro. Importantly, the infrastructures are fully funded with the contributions of the LSC, CUP and AdG. 
Table \ref{tab.n100:INFRA} summarises the total costs of the infrastructures. The costs detailed in the tables are very accurate, since most of the components have already been acquired. 


  
\begin{table}[h!]
\begin{center}
\begin{tabular}{|l|c|c|c|c|}
\hline
 Cost &	Total& 	CUP & AdG &  LSC \\
 \vline
Gas System &	434,177 &	68,970 &	365,207 &	0 \\
Platform and Castle	& 250,600 & 	0	&0 &	250,600 \\
Xenon (100 kg + 100 kg)	&1,056,000	& 0 & 0 &	1,056,000 \\
Lead cleaning	& 44,581 &44,581 &	0 & 0 \\
Cleaning equipment	18,150	& 18,150	& 0	& 0	\\
Clean tent	 & 59878	&	0& 59,878 &	0	\\
Radon suppression 	& 5,400 &	0 &	5,400 &	0	\\
Radon monitoring	6,700 &	0	& 6,700	& 0	\\
 \hline
{\bf Total Infrastructures} &	{\bf1,875,486}& 131,701& 437,185 & 1,306,600 \\	
 \hline\hline
\end{tabular}  
\caption{Costs of the NEXT infrastructures at the LSC.}
\label{tab.n100:INFRA}
\end{center}
\end{table} 



\subsection{Costs of computing, calibration and slow controls}
Computing is estimated in 69,521 \euro\ that will be covered by the AdG grant. The costs of the slow control (18,271 \euro) and the calibration (60,700 \euro, cost dominated by the need to purchase special radioactive sources for calibration) are assigned to FIS2014. 

\subsection{Total costs equipment, NEXT construction}
The total costs in equipment of NEXT construction are summarised in table  \ref{tab.TotalE}. They totalise about 4.6 million \euro\ including the xenon gas (1.1 million \euro). The {\bf external} contributions to the project (e.g, the money that does not come from the spanish science system)
add to 1,286,475 \euro\ ,  a figure slightly higher than the co-funding of 1,279,737  \euro\ 
requested in this project. 

\begin{table}[h!]
\begin{center}
\begin{tabular}{|l|c|c|c|c|c|c|}
\hline
Cost	& Total &	CUP &	AdG	& USA &	LSC &	FIS2014 \\
 \hline
& {\bf 4,626,074} &	798,583 & 	897,218 & 	413,457&	1,306,600 & 1,279,738 \\	
 \hline\hline
\end{tabular}  
\caption{Total costs of the NEXT construction (equipment).}
\label{tab.TotalE}
\end{center}
\end{table} 

\subsection{Costs of personnel for NEXT construction}

The construction, commissioning and operation of the NEXT detectors require of a team of specialised physicists and engineers. This team comes from both national and international universities and research institutions. The contributions of the international collaboration, in particular of the USA groups during the period of R\&D and design of NEXT have been very important for the development of the project. Currently, the spanish groups, in particular those participating in this co-ordinated project have absorbed the know-how brought to the collaboration by the crucial contributions of the Berkeley group (prof. David Nygren, the inventor of the TPC technology) and Texas group (the late prof. James White, who was the leading World expert in high pressure gas chambers). 

The CUP grant have made possible the creation of a world class group at IFIC, which includes the PI, the technical coordinator (Dr. Igor Liubarsky, a renewed expert in the field), two R\&C fellows, one of them senior (Dr. Sorel) and one of them junior (Dr. Novella, who starts this year in the group), four post-docs  (Laing, Ferrario, L\'opez-March,  Renner) and 6 Ph.D. students, 2 of whom will present their Ph.D. thesis in 2014. Last but not least, the group has formed several engineers. S. C\'arcel is leading the development of mechanics (with the help of technical mechanics engineer A. Mart\'inez) and J. Rodr\'iguez leads the development of electronics (with the help of technical electronics engineer V. Alvarez).

The group at the UPV brings the essential expertise in front-end electronics and data acquisition. The group includes three experienced engineers, all of them professors at the UPV. Last, but not least, prof. J.A. Hernando, from the University of Santiago has taken the important role of calibration and reconstruction coordinator in NEXT. 

We require co-funding to keep essential personnel for the project. A substantial contribution to personnel at IFIC will come from the  AdG grant, which will provide funds for amount of 1,097,258 \euro\ over the period requested for this grant. This will cover the salary of the technical coordinator (Liubarsky) and 2 post-docs (Renner, López-March). Two post-docs (Ferrario and Laing) have applied to the ``young research program''.  Consequently, the IFIC group does not require support for post-docs to this grant.  

IFIC  requests funding to keep our 2 senior engineers (C\'arcel and Rodr\'iguez), who are in charge of essential parts of the project and one technical engineer. The other two technical engineers in the group will be supported by the AdG grant.  

The UPV is in charge of the full development of the electronics and brings in essential man power (with permanent positions). The personnel needed is: a technical engineer to help with the development of the front-end electronics, lead by the electronics coordinator of NEXT (prof J. Toledo), and a senior enginner/computer scientist to help with the dual task of DAQ development (task lead for the DAQ coordinator of NEXT, professor R. Esteve) and online computing.  

Last but not least, we request a post-doc to reinforce the group at the University of Santiago. The group is led by prof. J.A. Hernando, a renowned physicist who has made major contributions to neutrino physics and to flavour physics. Hernando is now full time in NEXT and has taken the role of calibration and reconstruction coordinator. A post-doc to help in these tasks, essential for the performance of NEXT is requested. 

The NEXT project is extremely well suited as a training ground for students and post-docs. The project involves the construction, commissioning, operation and data analysis of the most advanced HPXe in the World, and the possibility to participate in a major discovery. The teams are very experienced and well organised. At IFIC, four senior physicists (the PI, Dr. Liubarsky, Dr. Sorel and Dr. Novella). At the US, prof. Hernando is already working with two students in calibration and reconstruction.

At the same time, graduate students are very important for the future of the project and for its impact in science and society. Consequently, the groups in this coordinated project require 2 FPI grants, one at IFIC and one at US. The group of US can absorb a second graduate student, and the group at IFIC can absorb at least 2 in 2015, but we will seek for additional funding both at national and international level to support them. 

%Table \ref{tab.P} summarizes the personnel requested. Table \ref{tab.new:PT} details the standard salaries
%payed to post-docs and engineers. Finally, table \ref{tab.new:PC} describes the personnel costs required to this project.
%
%\begin{table}[h!]
%\begin{center}
%\begin{tabular}{|l|c|c|c|c|}
%\hline
%Group &	post-docs	& engineers &	technical engineers & FPI\\
% \hline
%IFIC &	1 &	2	&1 (3 yr) &	1\\			
%UPV	  & 0	&1 &	1 (3 yr)  &	1 \\			
%US	& 1 &	0 &	1 (3 yr)  &	 1\\	
%CLPU	& 0 &	0 &	1 (3 yr)  &	 1\\			
% \hline
%{\bf Total} & 2 & 3 & 4 & 4 \\
% \hline\hline
%\end{tabular}  
%\caption{Personnel requested.}
%\label{tab.P}
%\end{center}
%\end{table} 
%
%\begin{table}[h!]
%\begin{center}
%\begin{tabular}{|l|c|c|c|}
%\hline
% &	post-docs	& engineers &	technical engineers \\
% \hline
%Cost &	40,000 &	40,000	&30,000 \\					
% \hline\hline
%\end{tabular}  
%\caption{Table of costs per person per year.}
%\label{tab.new:PT}
%\end{center}
%\end{table} 
%
%\begin{table}[h!]
%\begin{center}
%\begin{tabular}{|l|c|c|c|c|c|}
%\hline
%Group &	post-docs	& engineers &	technical engineers &  Total \\
% \hline
%IFIC	&160,000 &	320,000 &	90,000 & 570,000 \\
%UPV	 &	0 & 160,000 &	90,000 &	250,000 \\
%US	& 160,000 & 0 & 90,000 &	250,000\\
%CLPU & 0 & 0 & 90,000 & 90,000\\
%\hline
%{\bf Total} & 320,000 & 480,000 & 360,000 & {\bf 1,160,000} \\
% \hline\hline
%\end{tabular}  
%\caption{Personnel costs.}
%\label{tab.new:PC}
%\end{center}
%\end{table} 
%
%Notice that the total costs requested in this project are slightly below those provided by external fund sources such as the AdG. 


\subsection{Travel to LSC}
The NEW detector will be installed and commissioned at the LSC in mid 2015. In 2016, NEW will operate at the LSC, while the NEXT-100 detector will be constructed at IFIC, UPV and Texas, among other laboratories. In 2017, NEXT-100 will be commissioned at the LSC. Operation will proceed from 2018 onwards.

During commissioning, we foresee the constant presence at the LSC of one of our mechanical engineers and one of our electronics engineer (4 weeks a month).  This is a must, because IFIC and UPV jointly coordinate the mechanics, electronics and computing of NEXT. In the operation periods, when the detector is stable, this presence can be reduced to one week per month. Concerning post-docs, a constant presence of 2 post-docs or students (4 weeks a month) from our groups is needed. In addition, one student or post-doc in charge of the detector shifts is needed. We also foresee the presence of the PI and technical coordinator for about one week per month during the span of the project. 

Notice that the personnel at the LSC provided by the international collaboration will amply match the personnel provided by this project. The USA groups foresee to deploy at least two physicists to the LSC (4 weeks per month). The portuguese groups will deploy at least two more. Personnel from the University of Zaragoza, which is a part of NEXT, will travel frequently to the LSC. 

To minimise costs, we foresee to rent an apartment near Canfranc (probably at Jaca), and to organise travel car-pooling the teams. The daily expenses are computed conservatively (250 \euro\ per week). 

\begin{table}[h!]
\begin{center}
\begin{tabular}{|l|c|c|c|c|}
\hline
Activities at LSC &	2015 &	2016 &	2017 &	2018\\
 \hline	
Construction &	NEW (6 months) &	NEXT-100 & - & - \\ 		
Commissioning	&NEW (6 months) &	- & NEXT-100 & - \\	
Operation	&	- &  NEW & - & NEXT-100 \\ 	
\hline\hline	
\end{tabular}  
\caption{Activities at the LSC.}
\label{tab.ActivitiesLSC}
\end{center}
\end{table}
 
\begin{table}[h!]
\begin{center}
\begin{tabular}{|l|c|c|c|c|}
\hline
Personnel at LSC &	2015 &	2016 &	2017 &	2018\\
\hline							
engineers &	2 x 4 w/m per 6 months &	2 x 1 w/m & 2 x 4 w/m per 9 months  &	2 x 1 w/m\\
post-docs/students & 	2 x 4 w/m per 6 months &	2 x 4 w/m & 2 x 4 w/m &	2 x 4 w/m \\
Technical coordinator &	1 x 1 w/m &1 x 1 w/m &1 x 1 w/m &	1 x 1 w/m \\
PI	& 1 x 1 w/m &	1 x 1 w/m &1 x 1 w/m &	1 x 1 w/m\\
Shifters &	1 x 4 w/m per 6 months &	1 x 4 w/m	& 1 x 4 w/m & 1 x 4 w/m \\
\hline\hline
\end{tabular}  
\caption{Personnel at the LSC. w/m means week per month. }
\label{tab.PersonelLSC}
\end{center}
\end{table}

\begin{table}[h!]
\begin{center}
\begin{tabular}{|l|c|c|c|c|}
\hline
Weeks at LSC &	2015 &	2016 &	2017 &	2018\\
\hline							
engineer& 	48 &	24 &	72 &	24 \\
post-docs/students & 	48	& 96 & 	96 &	96\\
Technical coordinator &	6	& 12	 & 12	 & 12\\
PI	& 6	& 12	 & 12	 & 12\\
Shifters &	24 &	48 &	48 &	48\\
\hline
Total &	132	&192 &	240 &	192\\
\hline\hline
\end{tabular}  
\caption{Weeks at the LSC}
\label{tab.WeeksLSC}
\end{center}
\end{table}

\begin{table}[h!]
\begin{center}
\begin{tabular}{|l|c|}
\hline				
Large apartment rental (month) &	1200	\\		
car-pool trip &	150 \\			
number of car-pool trips & 1 per week \\
subsistence/week &	250 \\
\hline \hline	
\end{tabular}  
\caption{Details of costs}
\label{tab.DetailsLSC}
\end{center}
\end{table}	


\begin{table}[h!]
\begin{center}
\begin{tabular}{|l|c|c|c|c|}
\hline	
Concept &	2015 &	2016 &	2017 &	2018\\
\hline						
Apartment	 & 7,200 &	14,400 &	14,400 &14,400 \\
Trips	& 3,600 &	6,000 &	6,000 &	6,000 \\
Subsistance &	33,000 &	48,000	& 60,000 & 48,000 \\
\hline
Total &	43,800 &	68,400 &	80,400 &	68,400 \\				
\hline \hline				
\end{tabular}  
\caption{Costs travel to the LSC.}
\label{tab.CostsLSC}
\end{center}
\end{table}

Tables \ref{tab.ActivitiesLSC}, \ref{tab.PersonelLSC},  
\ref{tab.WeeksLSC}, \ref{tab.DetailsLSC} and  \ref{tab.CostsLSC} detail the calculation of costs. The total travel to LSC foreseen during the span of this project is 261,000 \euro.




%\subsection{Costs of construction, personnel and travel to LSC}
%\begin{table}[h!]
%\begin{center}
%\begin{tabular}{|l|c|c|c|c|}
%\hline
%Subproject &	Construction &	Personnel  &	Travel& Total\\
%\hline
%COORD (IFIC)	& 859,091 &	600,000 &	181,000 &	1,640,091 \\
%ENG (UPV)	& 420,646 &	190,000	& 40,000	& 610,646 \\
%CALREC (US)	& 0	& 160,000	 & 40,000	& 200,000 \\
% \hline\hline
%\end{tabular}  
%\caption{Costs of NEXT construction, personnel and travel to LSC.}
%\label{tab.CostsTotal}
%\end{center}
%\end{table} 
%
%The total costs of construction, personnel and travel to the LSC amounts to 
%{\bf 2,450,737 \euro}. The costs are distributed in three subprojects, called COORD (coordination, IFIC), 
%ENG (engineering, UPV) and CALREC (calibration and reconstruction, US). The distribution of costs per subproject is detailed in table \ref{tab.CostsTotal}. Notice that the construction funds requested
%by IFIC are matched by those provided by the AdG, and the construction funds requested by the 
%UPV are matched by those provided by the USA groups. All the personnel funds are matched by 
%funds provided by the AdG. The personnel from the co-ordinated project at the LSC will be matched
%by personnel from the international collaboration. 

%\subsection{Common fund}
%
%The operation and maintenance costs of NEXT will be covered by contributions of all the groups to a common fund (CF). The CF is being implemented in 2015 and will be firm from 2016 onwards. The CF is implemented by charging an annual quota (2,500 \euro) per Ph.D in the group. Table \ref{tab.CFD} gives the distribution of PH.Ds in the collaboration (as of September 2014) and their contribution to the CF. The total is 100 k\euro\ per year, which is the right order of magnitude to cover for common expenses (including supplies, cleaning material, maintenance of subsystems, nitrogen and argon gas, and replacements of expensive supplies such as gaskets) and to afford modest improvements to the systems. 
%
%\begin{table}[h!]
%\begin{center}
%\begin{tabular}{|l|c|c|c|}
%\hline
%Group &	number of Ph.D & 	Contribution \\
%\hline
%IFIC &	10	& 25,000 \\
%UPV	& 3 &	7,500 \\
%US	& 2	& 5,000 \\
%UAM	  & 2 & 	5,000 \\
%UNIZAR	& 6	& 15,000 \\
%Portugal 	& 6	& 15,000 \\
%Russia	& 3	& 7,500 \\
%Colombia	& 3	& 7,500 \\
%USA	& 5 & 	12,500 \\
%\hline
%Total & 40& 100,000 \\
% \hline\hline
%\end{tabular}  
%\caption{Contributions to NEXT common fund.}
%\label{tab.CFD}
%\end{center}
%\end{table} 
%
%In this project we require funds to contribute to the NEXT CF, corresponding to 3 years (2016, 2017 and 2018). The contributions to 2015 can be covered with remaining funds from CUP. The contribution is
%37,500 \euro\ per year, thus a total of 112,500 \euro. 
%
\end{document}
%
