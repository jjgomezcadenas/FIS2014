NEXT is organised as an international collaboration, which includes groups from Spain, Portugal, Russia, US, and Colombia. The spanish groups participating in NEXT are: Instituto de Física Corpuscular (IFIC), a joined center of the University of Valencia (UV) and the Spanish Council for Research (CSIC).The  Polytechnic University of Valencia (UPV). University of Santiago de Compostela (US). Autonomic University of Madrid (UAM); and University of Zaragoza (UZ). 

The leading groups participating in this coordinated project form the core of the collaboration. The spokesperson (and PI of this coordinated project), the technical coordinator, the software coordinator and the leaders of several working  packages are members of IFIC. The coordinators of the electronics, DAQ, and risk management are members of the UPV. The coordinator of calibration and reconstruction is a member of US. The groups participating in this coordinated project invest 100\% of their research time and resources in the NEXT project. 

Another key task for the experiment is that of radio purity, coordinated by the UAM (prof. Luis Labarga) who also presents an project to this call. The project of prof. Labarga includes a proposal to participate in the Super Kamiokande experiment, and for this reason we have considered more appropriated to present separated proposals. The task of radio purity and the corresponding request for resources is described in his project.

%The IFIC, UPV, UV and US groups work in a fully co-ordinated way. The NEXT collaboration is organised in terms of Working Packages (WP) which include members of the different groups. In addition of the WP structure, a general ``hardware coordination meeting'' and a ``software coordination meeting'' which involves members of all the groups is organised on a weekly-basis. Last, but not least, the coordination of the groups is essential to construct, commission and operate the NEW and NEXT-100 detectors at the LSC. The management of the laboratory and its Scientific Committee (SC), require a formal Project Management Plan (PMP) from the NEXT collaboration, and reviews its progress every six months. The PMP integrates the activities of all the collaboration groups and in particular of the groups participating in this project. 

Furthermore, a strong collaboration is currently being formed between IFIC and the Center for Pulsed Lasers (CLPU after the initials of the center in spanish), to develop the laser technology which could be used to tag the barium ion emitted in the \bb\ decays, resulting (when combined with the excellent energy resolution of NEXT and its topological signature) in a virtually background-free experiment. We are in the process of preparing a ``white paper'' detailing the theoretical grounds and the experimental procedures to address a successful  \BATA\ program. 
