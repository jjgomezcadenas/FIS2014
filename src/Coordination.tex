NEXT is organised as an international collaboration, which includes groups from Spain, Portugal, Russia, US, and Colombia. The spanish groups participating in NEXT are: Instituto de Física Corpuscular (IFIC), a joined center of the University of Valencia (UV) and the Spanish Council for Research (CSIC).The  Polytechnic University of Valencia (UPV). University of Santiago de Compostela (US). Autonomic University of Madrid (UAM); and University of Zaragoza (UZ). 

The leading groups participating in this coordinated project form the core of the collaboration. The spokesperson (and PI of this coordinated project), the technical coordinator, the software coordinator and the leaders of the detector construction, are from  IFIC. The coordinators of the electronics, DAQ, and slow controls are members of the UPV. The coordinator of calibration and reconstruction is a member of US. The groups participating in this coordinated project invest 100\% of their research time and resources in the NEXT project. 

Another key task for the experiment is that of radio purity, coordinated by the UAM (prof. Luis Labarga) who also presents a research project to this call. The project of prof. Labarga includes a proposal to participate in the Super Kamiokande experiment, and for this reason we have considered more appropriated to present separated proposals. The task of radio purity and the corresponding request for resources is described in his project.


Furthermore, a strong collaboration has been  formed between NEXT (and in particular the IFIC group) and the Center for Pulsed Lasers (CLPU after the initials of the center in spanish), to develop the laser technology which could be used to tag the barium ion emitted in the \bb\ decays, resulting (when combined with the excellent energy resolution of NEXT and its topological signature) in a virtually background-free experiment. The ambitious R\&D program that could produce a viable scheme for barium tagging (\BATA) is also described in this proposal.  
