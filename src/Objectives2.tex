\subsection*{Objectives}

%2. La hipótesis de partida y los objetivos generales perseguidos con el proyecto coordinado en su conjunto, así como la adecuación del proyecto a la Estrategia Española de Ciencia y Tecnología y de Innovación y, en su caso, a Horizonte 2020 o a cualquier otra estrategia nacional  o internacional de 

The overall goals of this research proposal are:

\begin{enumerate}
\item Construction, commissioning and operation of the NEW and NEXT-100 detectors, during a period of 4 years, from 2015 to 2019.
\item Demonstrate the feasibility of barium tagging in an HPXe, performing a systematic set of small, focused, prove-of-concept experiments. 
\end{enumerate}
  
The COORD subproject leads the construction of the NEW and NEXT-100 detectors, while the ENG subproject leads the deployment of the electronics, DAQ and slow controls. The CALREC subproject leads the calibration of the detector. The reconstruction and analysis of the data is shared between IFIC and US. The R\&D for barium tagging is lead by the BATA subproject (CLPU), with the participation of all the groups.   

The above objectives are very well aligned to the spanish program for science, as demonstrated by the fact that NEXT has been supported by the CONSOLIDER-INGENIO project CUP. The support of the AdG/ERC makes it clear that the projects suits perfectly well the goals of H2020. NEXT is a CERN recognised experiment and has been listed by NSA\footnote{http://science.energy.gov/~/media/np/nsac/pdf/docs/2014/NLDBD\_Report\_2014\_Final.pdf} as one of the key \bbonu\ experiments in the field, and the one with best future prospects.
