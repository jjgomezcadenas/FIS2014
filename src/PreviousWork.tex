
%Si el proyecto es continuación de otro previamente financiado, individual o coordinado, deben indicarse con claridad los objetivos y los resultados ya alcanzados de manera que sea posible evaluar el avance real que se propone en el nuevo proyecto. Si el proyecto aborda un tema nuevo, deben indicarse los antecedentes y contribuciones previas de los equipos de investigación que justifiquen su capacidad para llevarlo a cabo.
%

This research project is the continuation of the CONSOLIDER-INGENIO project CUP (2010-2014), and the SEIDI project FIS-XXX--- (2012-2014). 

To assess the achievements of the collaboration so far it is useful to review the historical development of the project. 

The largest HPXe chamber ever operated in the world before NEXT was the so-called St.~Gotthard TPC \cite{Luscher:1998sd}. It had a total mass of 5 kg of xenon, and was, therefore, of a similar size as our NEXT-DEMO prototype, although it operated at a considerably lower pressure (5 bar). Furthermore, the St.~Gotthard TPC amplified the ionisation charge ---needed to measure the event energy--- with a plane of wires at high voltage. This classical technology, the only one mature enough in the mid 90's, when the experiment operated, required the addition of a quencher ($CH_4$) to the xenon, to avoid sparks. Unfortunately, the methane destroyed the scintillation signal in xenon, with two undesirable consequences: a) the possibility to measure the start-of-the-event or $t_0$, defined by the prompt scintillation signal was lost; and b) the measurement of the energy resolution degraded. The energy resolution of the St.~Gotthard TPC was 7\% at \Qbb\ and the background was dominated by events coming from the detector walls that could not be vetoed due to the lack of $t_0$. The (relatively) poor results obtained with this early experiment, prompted the EXO experiment to choose liquid xenon and abandon the \HPXE\ technology. However, a LXe TPC has mediocre resolution, due to abnormal partition between scintillation and ionisation in the liquid phase (EXO-200 measures 3.6\% FWHM at \Qbb) and the topological signature of the event (the track of the two electrons) is also lost due to the high density of the liquid. 

The first problem that the NEXT project faced was to develop the technology to build high-pressure, radio-pure xenon chambers. High-pressure implies also moderately high-vacuum, needed for gas purity. The technology did not exist in Spain, and was rather underdeveloped elsewhere. All the \HPXE\ detectors built prior to 2009 (except for the St.~Gotthard TPC) where small objects, typically holding a few hundred grams of gas at moderately low pressures. Radiopure copper, as in the St.~Gotthard TPC, could not be used to build the pressure vessel, since the NEXT-100 detector was much larger and had to hold much higher pressures.  It was necessary, therefore, to find a radiopure solution for the pressure vessel, either in steel or in titanium (an alloy of steel and titanium was chosen at the end). Furthermore, it was necessary to develop a way to read the ionisation signal without killing the scintillating signal (i.e. without using quenchers) and without degrading the excellent intrinsic resolution available in the gas. 

The construction of the NEXT-100 detector faced, therefore, a large number of challenges. From the instrumental point of view, it was necessary:

\begin{enumerate}
\item {\em To acquire the technology}. This required equipping state-of-the-art laboratories, hiring specialised personnel and building prototypes.
\item {\em To study technological solutions, in order to read the ionisation and scintillation signals}. A number of possibilities were, a priori, available. In particular, the ionisation charge could be transformed in secondary VUV light and be read with optical sensors (the EL solution), or amplified with micro-pattern devices (the micromegas or MM solution). Within the EL solution, there was the choice of using photomultipliers (PMTs) and multi-pixel (SiPMs) devices or avalanche photo diodes (APDs). 
\end{enumerate}

Coupled with those instrumental challenges, the collaboration had to attack major mechanical engineering problems:

\begin{enumerate}
\item{\em Design and build a radiopure pressure vessel}, capable of holding at least 100 kg of xenon and capable of withstanding up to 15 bar with negligible losses (less than 1 gram a year). 
\item{\em Design and build a radiopure energy plane}, capable of protecting the PMTs inside the pressure vessel.
\item{\em Design and build a radiopure tracking plane}, including large custom feedthroughs, needed to extract the signals of $\sim$ 8000 SiPMs from the pressure vessel. 
\item{\em Design and build a state-of-the-art gas system}, capable of guaranteeing gas purity via continuous recirculation in a hermetic loop. The gas system had to guarantee redundancy and safety to minimise to negligible level the chances of loosing any substantial amount of enriched xenon gas. 
\item{\em Design and build the shielding of the detector}, choosing between several possible solutions (e.g., a water tank or a lead castle).
\item{\em Design and build the infrastructures} to hold the apparatus (working platform, seismic pedestal).
\end{enumerate}

Finally, one had to provide solutions for the electronics of the PMTs and the SiPMs, the  DAQ and the software.

In the years 2010-2013, the various problems were attacked in a systematic way: a brief summary can be jotted down as follows:

\begin{enumerate}
\item {\em Learning R\&D period (2010)}, needed to acquire the very innovative technology the detector is based on, and to equip state-of-the-art laboratories in some of our participating institutions. This phase of the project resulted in the construction of several prototypes, including NEXT-DEMO, capable of holding up to 4 kg of gas (thus, the same mass than  the St.~Gotthard TPC, and much higher pressure). DEMO has been operating continuously at IFIC for more than two years, demonstrating, in addition to its excellent performance the stability of the technology.  

\item {\em Selection R\&D period (2011)}, targeted to choose among the various technological solutions candidate to be implemented in NEXT-100. This period culminated in June of 2011 with the presentation of a Conceptual Design Report (CDR), where the NEXT detector was defined as an electroluminescent HPXe TPC equipped with photomulitpliers (PMTs),  to read the event energy and multipixel proportional counters (MPPCs also called SiPMs), to reconstruct the event topology. 

\item {\em R\&D targeted to produce a Technical Design Report (TDR)}, which defined the actual detector to be built. The TDR was presented in February 2012 and in final version in May 2012. The TDR defined solutions for the instrumentation, mechanical design of detector and infrastructures, electronics, DAQ and software, as well as the detector background model. 

\item {\em Demonstration of performance}: during 2012 and 2013, the results of the prototypes were analysed and published, showing the excellent performance (energy resolution, electron reconstruction) of the apparatus, as well as the robustness of the EL technology \footnote{http://next.ific.uv.es/next/talks.html}.  
\end{enumerate}

To summarise, the previous work of the NEXT collaboration has resulted in the demonstration, using large prototypes (NEXT-DEMO/DBDM ) of the excellent performance of the \HPXE\ technology. Furthermore, the full detector design has been detailed in the TDR and an exhaustive radiopurity campaign has been carried out  validating the background model assumed in the TDR.  The physics case of NEXT has therefore been clearly established, demonstrating that this technology has the potential to be one of the very few that can be extrapolated to the ton scale.\footcite{GomezCadenas:2012jv} 

