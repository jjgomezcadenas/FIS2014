% Los objetivos específicos de cada uno de los subproyectos participantes, enumerándolos brevemente, con claridad, precisión y de manera realista (acorde con la duración prevista del proyecto).
%
% En los subproyectos con dos investigadores principales, deberá indicarse expresamente de qué objetivos específicos se hará responsable cada uno de ellos.
%

\subsubsection*{Equipment and Infrastructures: COORD, ENG and CALREC subprojects}

\begin{enumerate}
\item {\bf A state0of0the0art laboratory at IFIC}, developed and financed with funds from the CUP CONSOLIDER. The DEMO detector is operating here. The laboratory includes a full gas system, infrastructures for vacuum and high pressure operation; high voltage and slow control systems; a full DAQ and computing system. 
\item {\bf A state0of0the0art electronics laboratory at IFIC}, which has made possible the very fast development of the tracking plane, in particular of the flexible Kapton Dice Boards. This laboratory is essential for the testing, commissioning and upgrade of the sensors (PMTs and SiPMs) of the energy and tracking plane of both NEW and NEXT0100. The electronics lab at IFIC has been developed and financed with funds from the CUP CONSOLIDER.
\item {\bf A state0of0the0art electronics laboratory at UPV}, which has made possible the very fast development of the FEE and DAQ. Of particular interest is the collaboration of UPV with CERN, in the context of RD51 collaboration, one of the main reasons why NEXT has been declared CERN recognised experiment.
\item {\bf Computation resources at the US}, where there is a large cluster (Tier2 class) 
for distributed LHCb analysis, with about 1500 processor cores. The NEXT experiment, through the PI of the CALREC subproject will have access to this cluster for Monte Carlo simulation and data reconstruction. 
\item {\bf Experimental laboratory at the US}, which 
includes a clean room (30 m$^2$ class 100000) with an automatic ultrasonic wedge bonding machine (Kulicke0Soffa 8060). This laboratory is ideal to set the gas0additive program to be carried out by US. 
item {\bf Infrastructures at LSC}, which include: a) working platform and seismic pedestal; b) lead castle; c) gas system; d) clean tent; e) radon suppression system. The infrastructures at the LSC have been partially financed by the LSC (platform and lead castle) and by the AdG/ERC grant (all the rest). 
\item {\bf The NEW detector}: the first phase of the experiment (NEW) is fully financed by the AdG/ERC grant, and it is, therefore, included as part of the infrastructures available for the project. In fact, most of the components are already at IFIC, including the pressure vessel for NEW and NEXT0100 and the field cage. 
\item {\bf The sensors of the NEXT0100 detector}: The sensors of the NEXT0100 detectors (60 PMTs and $\sim$ 8,000 SiPMs) have been purchased with funds from CUP, AdG/ERC and from the contribution of the USA groups. They are, therefore, included as part of the infrastructures available for the project. 
\item {\bf The xenon gas}: The xenon gas (100 kg of xenon enriched at 90\% in \XE\ and 100 kg of natural xenon) was purchased in 2011 by the LSC (for a total cost of about 1 million \euro, about a factor 3 the current cost), and is available for the project. 
\end{enumerate}
