% Los objetivos específicos de cada uno de los subproyectos participantes, enumerándolos brevemente, con claridad, precisión y de manera realista (acorde con la duración prevista del proyecto).
%
% En los subproyectos con dos investigadores principales, deberá indicarse expresamente de qué objetivos específicos se hará responsable cada uno de ellos.
%

\subsubsection*{Personal requested for the CALREC subproject}

In order to reach the objectives of the CALREC subproject we request the following staff:

\begin{enumerate}

\item {\bf A postdoct highly qualified on gas noble detectors and instrumentation}. The postdoct will {\bf be responsible for designing, building and installing the calibration system with radioactive sources for NEW and NEXT-100}. He will dedicate 60 \% of the his time to this task, and the rest to R+D efforts for BEXT and for NEXT 1 Ton. 
%The postdoct will continue the initial studies done by the NEXT collaboration  with noble gasses and additives. With this mixtures, one could improve the tracking resolution of the detector, or provide a technique to detect dark matter (see \cite{NEXT-TMVA} \cite{NEXT-DB}).  
The postdoct will closely collaborate with other NEXT institutions (mainly IFIC an UZ) and use existing prototypes and installations. The contract duration will be 4 years.

The incorporation of a high qualified postdoct to the USC group has many aventages: 1) It removes one heavy responsibility from the IFIC group, that at this moment, supports most of the NEXT construction commitments, 2)  
It ensures a modest but sizable hardware contribution from a spanish institution, reinforcing the collaboration, 
%3) It is an initial step toward a larger and most substancial contribution from the USC and Spain, for the NEXT 1 ton detector, 
3) It spreads and extends the technological know-how acquired by the IFIC group to other spanish institutions, 4) It opens the possibility to establish contacts and contracts with technological companies in the Galicia region and the Nort-West part of Spain. In summary, the incorporation of a postdoct to this subproject is crucial for the USC group to make a relevant contribution on NEXT and to reinforce the spanish presence inside the collaboration.

\item {\bf A Ph.D. student via a FPI grant} associated to this sub-project, with a thesis proposal on the ``Contribution to tracking algorithms in NEXT and measurement of the \bb ~spectrum with NEW".
The student will develop the reconstruction algorithms of NEXT. He will validate the reconstruction with data from the \NA, ~\CS~ and \Tl ~sources. For the thesis, the student will mesure the \bb ~spectrum with NEW data.
He/Shel will contribute part time to the construction, installation and operation of the NEW and NEXT-100 detectors at LSC. The IP of this subproject will be the supervisor.
%He will take the responsibility of the USC contribution on the WP13 (reconstruction and analysis). He is an expert on both subjects. In addition, he has very successful experiences with the direction of Ph.D. students.% D. Martinez and X. Cid defend their thesis on top physics topics at LHC. 

The main reason we request a Ph.D student is because we are convinced of the high quality of the thesis project, and 
that the NEXT experiment is a great place to train young researches. The IP has already demonstrated the benefits of a combination of brilliant students with good supervision. Second, the student will be a mayor contributor to the software and analysis of NEXT, and his presence, will largely enlarge and strength the USC contribution to reconstruction and analysis of NEXT data.

\item {\bf A technical staff} (with a degree on physics or engineer) to maintain and operate the calibration system. He should be an expert on radiation sources and radiation hazards. He will dedicate part of his time (~20\%) to R+D studies. The postdoct of the USC group will be his supervisor.

The main reason for the incorporation of the technician is to help the postdoct on his duties. A technician will liberate the postdoct from some of the daily basic tasks and help him in the installation, operation of the calibration system. The technician will be necessary on the periods of data taking 2016 and 2018.

%The incorporation of a Ph.D student via a FPI grant has again many benefits. It is  the best way to profit from the experience and skills of his supervisor on this subject. 
%He is an expert on reconstruction and analysis, more, Jose A. Hernando had a successful experience with previous Ph.D. students that defended high quality thesis.  It ensures that the USC will take a leading roll on the reconstruction and analysis of NEXT (WP13). And what it is more important, we offer to a young, brilliant student the possibility do research of excellence.  His/her thesis will be the main physics outcome of the NEW and NEXT-100 detectors.


%Given the knowledge of the supervisor on this subject, it almost guarantees a mayor presence of the USC on the reconstruction and analysis of NEXT.It is the way, that the USC, could perform one of its main duties: provide high-qualified training. In summary, this is an excellent thesis topic, and the USC group has demonstrated a high level training capability. 

%Jose A. Hernando will be the responsible that USC contributes and take a leading roll on WP13 (reconstruction and analysis). The affilia

\end{enumerate}


%To build and operate the calibration systems, the USC group request finance for a postdoct that will take of the design, construction, installation and commissioning of the calibration systems with radioactive sources and a technical staff (a graduate in physics or an engineer). 
%The postdoct must be an expert on gas detectors and good knowledge in underground physics. The technical staff must be an expert on handling radioactive sources and radioactive hazards. 
%The postdoct will be the final responsible of the calibration system and he should provide the calibration constants for each run. He will dedicate 60\% of his time to this task.
%His contract will have a duration of 4 years.
%The technical staff will be responsible of the maintenance and operation of the calibration system.
%He will dedicate to this 100 \% of his time. 
%We expect a 2 year contract, during the data taken periods of NEW and NEXT-100.

%In order to contribute to the R+D program for BEST and NEXT 1 ton. The postdoct, that will be an expert on gas detector, with large background in constructing and operating experiments,  will dedicate 40\% of his time to contribute on the \BATA program and to continue the R+D studies about how to use the gas noble with additives, that could help in order to reduce the trasversal diffusion of the electrons in they drift towards the tracking plane, or to increase the ratio of scintillation/ionization to detect dark matter, both R+D paths have been pursued by the NEXT collaboration (see \cite{NEXT-TMVA}, \cite{NEXT-RD}. Simple studies will take place at the laboratorios that the High Energy Group has at USC, but those one requiring the use of large equipment, they will be done at IFIC and UZ with previous existing prototypes. 

%The presence of a postdoct, with a technologist profile,  at the USC, is a necessary requirement if the USC group wants to contribute in a modest but substantial way to the construction of NEXT-100 and the R+D for NEXT 1 ton.
%It will spread the knowledge acquired during the previos R+D years, mostly at IFIC and UZ, to other institutions in Spain.
%it will preserve an expand that the technological know-how in gas detector and TPCs in Spain. It could provide some contact and possible interaction with technical enterprises in the Galicia region or north-west part of the Iberian peninsula. 
%The presence of a technical stall is desirable to liberate of high time consuming tasks to the postdoct and he can contribute in a more efficiency way to important parts of the project.

%Given the fact that the postdoct will be a high experience physicist, it is expected that he will get involved, if he wishes, on the physics analysis of the NEW and NeXT-100 data.

%In this proyect we also request a Ph.D. student grant (FPI).The student thesis will be development of tracking algorithms for NEXT and the measurement of \bb~ spectrum with NEW and. This is a topic for a high impact Ph.D. thesis. This work will be supervised and done in collaboration with the IP of this subproject, Jose A. Hernando. He is profesor at USC since the year 2000, and have a large experience in reconstruction algorithms and doing physics analysis. He was fellow and research staff at CERN from 2002-2009 and was responsible for the trigger algorithms, designed the tracking at LHCb. His previous Ph.D. student, Diego Martinez and Xabir Cid, did their thesis on the search for the very rare decay Bs->mumu with LHCb data. J.A Hernando was the coordinador of this analysis during 2011-2012. This is one of the most relevant results of the LHC Run-I. The article with most cited of those of LHCb. Both, D.  Martinez and X. Cid, were research fellows at CERN, and D. Martinez was awarded with the young physicist in 2013 prize of the European Physics Society (EPS).  

%A Ph.D student, with a FPI associated to this project, will ensure that USC can assume a leading roll on the reconstruction project of NEXT (WP13) and on the physics analysis. The student will be do the measurement of the \bb~ spectrum with NEW, the main physics result of this detector.Given the trajectory of successful training of the IP, the FPI could be a efficient way to ensure an excellent physics outcome of the NEW and NeXT-100 detector.

%The other goals of the project, the contribution of tracking reconstruction and physics analysis, and revisiting critically the NEXT detector for the 1 ton scale, will be done by the IP, Jose A. Hernando, that have a large experience on those topics. He was an expert on both at the experiment he previously worked on: LHCb. He designed the tracking and developed the trigger algorithms for LHCb, he was responsible of the later during the preparation period. This are excellent areas for a Ph.D student to to his/her thesis. The IP of the USC has an successful trajectory of training students.  











