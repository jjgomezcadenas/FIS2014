% Los objetivos específicos de cada uno de los subproyectos participantes, enumerándolos brevemente, con claridad, precisión y de manera realista (acorde con la duración prevista del proyecto).
%
% En los subproyectos con dos investigadores principales, deberá indicarse expresamente de qué objetivos específicos se hará responsable cada uno de ellos.
%

\subsubsection*{Objectives of the CALREC subproject}

The subproject CALREC, under the responsibility of the USC, will provide the calibration systems for the NEW and NEXT-100 detectors (WP14).
In addition the USC group will develop tracking reconstruction algorithms for NEXT, will
take a leading role in the analysis of the NEW and NEXT-100 data (WP13) and will collaborate  in the R+D tasks for BEXT and NEXT 1 ton. 

The different parts of this subproject are:
\begin{enumerate}
\item {\bf Design, construction, commissioning and operation of a calibration system to estimate the gain and noise of the SiPMs and PMTs sensors}. Two sets of 400 nm LEDs will be located on the tracking a energy plane. They will illuminate periodically the detector with a dim light. The gain and noise of SiPMs and PMTs will be estimated from single photo-electron signals (see \cite{NEXT-DEMO}).

\item {\bf Design, construction, commissioning and operation of an energy calibration system with external radioactive sources for NEW and NEXT-100}.
The energy scale and energy resolution in the region of \Qbb ~can be estimated using the photo-peak of the 2.6 MeV gamma from a \Tl~  source. In addition, the photo-peak electron and the two electrons of the double-scape peak (~1.6 MeV) can be used to measure the misidentification efficiency and the signal selection efficiency respectively. With additional 
\NA,  \CS~ sources we can calibrate the energy scale in the range of 500 keV.

\item {\bf Design, construction, commissioning and operation of a position calibration system with X-rays}. X-rays from \Xe~ (30 keV) and \KR ~(42 keV) are point-like sources in NEXT.
They will be used to correct the bias on the energy introduced by geometrical effects and to estimate the light collection at the SiPMs from a point-like source (PSF function). This is a fundamental element for the tracking reconstruction algorithms. R+D will be needed to understand the operation with \KR.

\item {\bf Design, implementation and validation of the tracking reconstruction algorithms of NEXT}.  
They will be validated with MC simulations and the data from the \NA, \CS~  and \Tl~  sources.

\item {\bf Take a leading roll in analysis}. Design and do the measurement of the \bb~ spectrum with NEW data, and Design and prepare the search for \bbonu~  for NEXT-100. 
Again, calibration data from the radioactive sources will be crucial to estimate the efficiencies of the analysis.

\item {\bf Contribute to the R+D for NEXT 1 ton} on noble gas detectors with additives (i.e TMVA) that can result in a smaller electron transverse diffusion and open the possibility to detect dark matter (following the preliminary results on \cite{NEXT-DM)}. This work will be done in collaboration with other NEXT institutions (IFIC and UZ) and using existing prototypes and installations.
\end{enumerate}
