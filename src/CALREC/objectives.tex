% Los objetivos específicos de cada uno de los subproyectos participantes, enumerándolos brevemente, con claridad, precisión y de manera realista (acorde con la duración prevista del proyecto).
%
% En los subproyectos con dos investigadores principales, deberá indicarse expresamente de qué objetivos específicos se hará responsable cada uno de ellos.
%

\subsubsection*{Objectives of the CALREC subproject}

The subproject CALREC, under the responsibility of the USC, will provide the calibration systems for the NEW and NEXT-100 detectors (WP15).
%In addition the USC group will develop tracking reconstruction algorithms for NEXT, will take a leading role in the analysis of the NEW and NEXT-100 data (WP13) and will collaborate  in the R+D tasks for BEXT and NEXT 1 ton. 
The USC group will coordinate, in collaboration with the IFIC, the implementation and validation of reconstruction algorithms (WP12); will take a leading role in the analysis of the NEW and NEXT-100 data and will collaborate  in the R+D tasks for BEXT and NEXT 1 ton. 

The objectives of this subproject are:

\begin{enumerate}
\item {\bf Design, construction, commissioning and operation of a calibration system to estimate the gain and noise of the SiPMs and PMTs sensors}. 
Two sets of 400 nm LEDs will be located on the tracking a energy planes. They will illuminate periodically the detector and the gain and noise of SiPMs and PMTs will be estimated from the single photo-electron signals. 
%(see \footcite{NEXT-DEMO}).

\item {\bf Design, construction, commissioning and operation of an energy calibration system with external radioactive sources for NEW and NEXT-100}.
The energy scale and energy resolution in the region of \Qbb ~can be estimated using the photo-peak of the 2.6 MeV gamma from a \Tl~  source. Furthermore the photo-peak electron and the two electrons from the double-scape peak can be used to measure the misidentification and signal efficiency respectively. 
The rest of the spectrum will be calibrated with additional  \NA ~(511 keV),  \CS~ sources (662 keV) and X-rays (30-40 keV).
A setup will be constructed and installed at LSC to store the sources and to take calibrations runs with NEW and NEXT-100.

\item {\bf Design, construction, commissioning and operation of a position calibration system with X-rays}. X-rays from \Xe~ (30 keV) and \KR ~(42 keV) are point-like sources in NEXT. They will be used to correct the energy bias introduced by geometrical effects and to estimate the light collection at the SiPMs from a point-like source (PSF function). The geometrical correction is crucial to obtain the target energy resolution $<1$ \% at \Qbb. And the PSF is a fundamental element for the tracking reconstruction algorithms. A setup to introduce the \KR ~will be installed at LSC.
R+D will be needed to understand the operation with \KR.

\item {\bf Design, implementation and validation of the tracking reconstruction algorithms of NEXT}. USC will coordinate, in collaboration with Dr. F. Ferrario (IFIC), the efforts to provide the reconstruction algorithms. The algorithms will be validated with simulation samples and with the data from the \NA, \CS~  and \Tl~  calibration sources.

\item {\bf Take a leading roll in analysis}. Design and perform the measurement of the \bb~ spectrum with NEW data, and design and prepare the search for \bbonu~  for NEXT-100.  Calibration data from the radioactive sources will be crucial to estimate the efficiencies of these analysis.

\item {\bf Contribute to the R+D} for \BATA~ and NEXT 1 ton. NEXT detector performance could be improved with mixtures of xenon and specific additives (i.e. TMA). That can result in a smaller electron transverse diffusion and a sharped, clear track. Applied to the \BATA ~project, the presence of additives could facilitate the transition from $Ba^{++} \to Ba^+$, triggering the tagging process. USC will contribute to the study of what additives and in which concentrations can trigger that transition. This R+D program will reuse previous prototypes and will be done in collaboration with the IFIC and CLPU groups.
%This will be a modest by sizable contribution to the R+D, and to minimize cost it will be done in collaboration with IFIC reusing previous prototypes.
% and in IFIC installations.
%The second, is the possible great synergy bettween \bb ~and Dark Matter (DM) direct search using gas xenon detectors, in particular the possibility to measure the directionality of a xenon recoiling nucleus from a dark matter candidate, following ideas and preliminary results on ref.\footcite{NEXT-DM}. Finally, the ultimate proof   of a \bb ~decay could be tagging the daughter $Ba^{++}$ ion. This is the goal of \BATA ~subproject, the possibility to tag ``in situ" ~the presence of $Ba^{++}$ via continues emission of red-light of $Ba^+$ using a pumping system with lasers. This is a challenging proposal with great possible impact on the 1 ton scale detector search. The USC plans to collaborate, with a modest but sizable effort, to the three lines of R+D. To minimize cost and maximice the existing installations and prototypes, this work will be done in collaboration and in the laboratories of IFIC and UZ (Zaragoza).
\end{enumerate}
