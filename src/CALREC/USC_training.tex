% Los objetivos específicos de cada uno de los subproyectos participantes, enumerándolos brevemente, con claridad, precisión y de manera realista (acorde con la duración prevista del proyecto).
%
% En los subproyectos con dos investigadores principales, deberá indicarse expresamente de qué objetivos específicos se hará responsable cada uno de ellos.
%

\subsubsection*{Capacidad Formativa del equipo solicitante}

The CALREC sub-project requests one Ph.D student (FPI grant)

{\bf Training program}

The student will follow the Particle Physics doctorate program of the USC (``programa de doctorado en F\'isica de Part\'iculas"). 
He/She will attend in addition at least one international school in HEP (CERN HEP school or the Gran Sasso Sciente Institute) and an national one (Taller de Altas Energ\'ias) 
The student will learn instrumentation in HEP, and high level programming and the use of mathematical method common in HEP. He will profit from the excellent internal NEXT collaboration.
It is expected that he will attend and present his results in the NEXT meetings.
He will attend (at least two) international conferences or workshops, and will present this thesis results in a high level profile conference.

{\bf Ph.D. thesis}

The IP of the sub-project has directed the following Ph.D. thesis:
\begin{enumerate}

\item Title: ``Search for the rare decays $B_s \to \mu^+\mu^-$
and $K_S \to \mu^+\mu^-$ in the LHCb with 1 fb$^-1$ integrated luminosity" \\
Institution: Universidade de Santiago de Compostela \\
Student: Xabier Cid Vidal \\
Date: 26/10/2012 \\
ID: CERN-THESIS-2012-154  

\item Title: ``Search for the very rare decay $B_s \to \mu^+\mu^-$ in the LHCb experiment" \\
Student: Diego Martinez Santos \\
Institution: Universidade de Santiago de Compostela \\
Date: 04/04/2010 \\
ID: CERN-THESIS-2010-068 

\end{enumerate}

{\bf Scientific career of the graduated Ph.D.s}

Diego Mart\'inez and X. Cid have an excellent career after finishing their Ph.D.s. They are young promises of the Spanish Physics. 

D. Mar\'inez Santos was research fellow at CERN, Geneva, after he defended his Ph.D. thesis at USC. He is now postdoct at NIKHEF institute, Amsterdam, and CERN Corresponding Associate. He is the coordinator of the Bs mixing phase, $\phi_s$, (the second main LHCb result after the $B_s \to \mu^+\mu^-$ search). He was awarded in 2013 with the Young Experimental Physicist Prize by the European Physical Society (EPS) for his work at the LHCb trigger and the search of the $B_s \to \mu^+ \mu^-$ decay. He has currently applied for a ERC (European Research Council) Starting Grant with a project based on reusing the LHCb detector to improve strange meson physics. He has successfully pass the first step of the selection process. 

X. Cid Vidal is now 3-years research fellow at CERN, Geneva, after he defended his Ph.D thesis at the USC. He is currently working on the identification of the Higgs boson to b,b-bar jets at LHCb and leading the study of strange meson physics at LHCb. He has recently presented a Marie Curie ITN  (International Training Network) proposal based on the search of kaon physics and the use of multivariate methods in other fields outside HEP, for example to financial markets. 


