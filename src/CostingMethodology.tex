\subsection{Costing Methodology}
The methodology to estimate the construction costs of the NEW and NEXT-100 detectors and their infrastructures at the LSC is as follows.

For the apparatus:

\begin{enumerate}
\item The detectors are divided into their functional systems: Pressure vessel, energy plane, tracking plane, field cage.
\item The cost of each subsystem is estimated in terms of their components: For example, the energy plane is composed of: a) support plate, b) PMT cans, c) PMT sensors, d) PMT bases, e) feedthroughs fto extract the signal. The cost of each functional component is estimated, if necessary dividing again into components: for example the PMT cans are composed of: i) body, end-cup, sapphire window, brazing, and gasket.
\item The costs of electronics and DAQ is added: the costs of electronics is estimated using the same methodology described above. 
\end{enumerate}

Analogously each infrastructure (gas system, platform, shielding, computing) is divided into functional parts. For example, the gas system includes a recirculation loop, valves, gaskets, a compressor, getters, recovery system and emergency system. The cost of each part is estimated. 

Finally one needs to add the operation costs: gas, supplies, gloves, masks, tools, cleaning equipment, radio purity measurements, etc. 

The NEW detector has been fully payed by the AdG and the CUP grants. Since most of the parts have already been acquired, we have a rather accurate estimation of the costs. The NEXT-100 detector is build as a 1:2 scale in size (1:4 scale in area, 1:8 scale in mass) of NEW. Many of the costs benefit from economy of scale and from the previous R\&D (e.g., the pressure vessel for NEXT-100 costs only 20\% more than the pressure vessel for NEW). The costs of the sensors (PMTs, SiPMs) and the electronics scales with the area. The infrastructures are, of course, common to both systems.

\subsection{Costing of the infrastructures}
The infrastructures include:
\begin{enumerate}
\item The gas: the LSC owns 100 kg of enriched xenon and 100 kg of natural xenon. The gas was purchased in 2010, for a total cost of around 1 M\euro (the same amount of gas would cost about 3 M\euro\ today).
\item The operating platform, seismic pedestal and lead castle structure. This was designed 

\end{enumerate}


