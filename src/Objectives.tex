\subsection*{Objectives}

The PI presented proposals to this committee requesting support for the development of the NEXT program in previous calls. The referees made a number of comments that are fully addressed in this proposal.  Specifically, the claimed good energy resolution of HPXe detectors has now been demonstrated in a large system by the NEXT-DEMO/DBDM prototypes; the full detector design has been detailed in the TDR; and the radiopurity measurements are almost complete validating the background model assumed in the TDR.  The physics case of NEXT has therefore been clearly established, demonstrating that this technology has the potential to be one of the very few that can be extrapolated to the ton scale.\footcite{GomezCadenas:2012jv} 

The successful operation of the NEXT-100 detector is a necessary milestone in the way towards the ton-scale. The construction of the detector is underway. A number of infrastructures are already operational at the LSC (working platform and gas system), and other key components are being manufactured (lead shielding castle, pressure vessel). The project has received strong support from the LSC Scientific Committee, but at this moment it is not fully funded. 

This research proposal has five major objectives, during a period of 5 years (60 months), from 2014 to 2019:
%%%
\begin{enumerate}
\item Complete the construction of NEXT-100 and demonstrate in the full detector, during its commissioning run, its unique combination of excellent resolution ($<1$\% FWHM at \Qbb) and low background rate ($<8 \times 10^{-4}$ \ckky).
%
\item Search for \bbonu\ events with the NEXT-100 detector. We aim to discover the Majorana nature of the neutrino or to reach the best sensitivity of the current generation of xenon experiments, improving that of EXO and KamLAND-Zen.
%
\item In parallel with the construction and initial operation of the NEXT-100 detector, we will carry out an R\&D targeted to improve its performance, specifically aimed at improving further the energy resolution and reducing the background rate by a factor of at least 4. 
%
\item Upgrade the NEXT-100 detector implementing the solutions developed by the previous R\&D. Our final goals for the upgraded detector are a resolution of 0.5\% FWHM and a background rate of $2 \times 10^{-4}$\ckky. 
%
\item Carry out, in parallel with the physics run of the improved NEXT detector, an R\&D targeted to address the scalability of the detector to one ton.
\end{enumerate}
%%%

%The proposed research has a large potential of making a discovery, since the improved NEXT detector with 150 kg will reach in our proposed running period a sensitivity 3--4 times better than currently achieved by EXO-200. In addition, the proposed research will demonstrate the feasibility of scaling a HPXe to masses in the ton range, capable of fully exploring the parameter space in the case of an inverse hierarchical spectrum of neutrinos. 

In summary,  NEXT offers an innovative technology to search for \bbonu. It is the only experiment of its kind in the world and  is very well placed to compete for the best ultimate sensitivity. NEXT is an international collaboration led by the PI of this proposal. The NEXT group at IFIC plays a central role in the project and coordinates the development of the experiment. The two major deliverables, should this proposal be successful, are: a) the discovery of the Majorana nature of the neutrino, if its mass is in the range 50--100 meV (uncertainties are related with the values of NME elements); and b) the demonstration of a technology that can be extrapolated to the ton scale, capable of covering the full parameter space if the neutrino mass spectrum has an inverse hierarchy.

