%Si la memoria se presenta a la convocatoria de RETOS INVESTIGACIÓN, deberá identificarse el reto cuyo estudio se pretende abordar y la relevancia social o económica prevista.
%
This research project is presented within the program of ``Challenges of society'', specifically, challenge number 6: {\bf Change and social innovation.}

We argue that this project represents a major innovation in the way that the so-called ``big-science" is performed in Spain.

``Big science'' is characterised by the need for large budgets, big machines (such as particle accelerators) and large staffs (for example, the number of physicists participating in the ATLAS and CMS experiments at CERN is of the order of 5,000). The discovery of the Higss boson at CERN is a quintessential of such big science, and clearly exemplifies its pros and cons. The obvious pro is the major scientific achievement that the discovery represents. Such a discovery has required the construction and operation of the LHC, one the most impressive scientific machines ever built by humankind. The gargantuan scale of the effort could only be met by a collective effort centralised in the largest particle physics laboratory in the World, CERN.  

Among the cons of big science are the large budgets that it involves, often invested in purchasing equipment to be installed at CERN (or other laboratories) and in paying scientific staff whose activity also develops at CERN. Such large budgets are often justified in terms of industrial and scientific returns. While those returns certainly exist, it is often not easy to quantify their impact in the countries that finance big science. Scientific authorship is one example. It is difficult to assign credit, in particular to students and young post-docs, when the detector is built and operated by thousands of physicists, all of them signing, normally in alphabetic order, the scientific papers. Furthermore, returns ten to be larger for countries who are already very developed scientifically. Specifically, the positions of leadership in the large CERN experiments, and in the CERN scientific and technical divisions, are dominated by countries like Germany, Switzerland, U.K., France and Italy. Industrial returns also tend to be larger for those countries. Instead, the scientific and industrial returns for Spain is very modest. 

Remarkably, the countries leading the big science at CERN and other laboratories\footnote{We refer here only to big science in physics, although the term also applies to biology and other disciplines} have also developed ``national science'' physics programs. A case of great interest is Italy, a country closer to Spain, in terms of GDP and social habits, than, say, Germany, U.K., or USA. However, the international impact and the returns of physics in Italy is much larger than in Spain. For example, the number of staff members at CERN is 115, to be compared with 275 corresponding to Italy (which has the second largest staff population, after France, with 1031 and followed by UK, with 223). Adding fellows and associates (that is temporary CERN contracts), the figures for Spain are 363, to be compared with 1726 for Italy\footnote{http://council.web.cern.ch/council/en/Governance/TREF-PersonnelStatistics2012.pdf}. Several italians have served as CERN general directors, and have lead or are leading the major experiments, such as ATLAS and CMS. The next CERN general director (and perhaps the first woman to occupy such position in the history of the lab) may be the ex-spokesperson of ATLAS, the italian physicist Fabiola Giannoti. And Italy has four Nobel prices in physics (Marconi, 1909, Fermi, 1938, Segrè 1959, Rubbia 1984), while Spain has none. 

Italy has also the best underground laboratory of Europe, and one of the best of the world, the LNGS. The lab hosts 20 experiments including three experiments searching for \bbonu\ processes (GERDA, CUORE and COBRA) and two experiments searching for Dark Matter (WARP and EXO). 

Through these experiments, the italian physics attracts external talent (some of the best physicist from Europe and USA participate in experiments at LNGS) and external funding, complements the big science at CERN with physics of a smaller scale concerning human resources and budgets (the \bbonu\ experiments typically include about 50-100 physicists, including Ph.D. students, to be compared with $\sim 3,000$~of ATLAS or CMS, and the budgets are one order of magnitude smaller). However, such ``local'' physics results in discoveries of great scientific impact (such as the discovery of neutrino oscillations, which has been the result of a world-wide effort involving underground laboratories in Italy, USA, Canada, Russia and Japan). It also allows the training of students and post-docs in experiments where young physicists can made a major impact at all levels, ranging from the construction of the detector to the analysis of the data (this is to be contrasted with the large and hyper-specialised efforts at CERN, where students and post-doc are often restricted to very specific areas of the experiment). Last, but not least, such local science has an important impact in the italian industry and in the appreciation of science by the public in general. 

We argue that, in order to balance and optimise the current big-science effort in Spain, it is necessary to develop the physics of the LSC, in analogy to the italian case. NEXT is the flagship experiment of our national laboratory, and has achieved intentional recognition, as demonstrated by the fact that is a recognised CERN experiment and has obtained an AdG/ERC, the first granted in the field of particle physics. 

We, therefore, consider that the NEXT project is a clear example of social innovation, as it proposes and has the potential of implement, profound changes in spanish physics. As described in this project, NEXT, through its various stages, has the potential to hit a major discovery. It will bring international credit and visibility to our science and to the LSC. And it has an important impact both in local industry (through contracts to many national firms, and development of high technology) and in the public perception of science (thorough a very intense activity in public forums including large-circulation cultural magazines such as JotDown, where the PI of this project directs the science section). 

Furthermore, the on-going collaboration with the CLPU further reinforces the above arguments, since the effort involves now a second national scientific installation. In addition the \BATA\ program implies a major example of inter disciplinarily, and can result in a number of important technological returns (development of micron laser technology, which can be applied to molecular fluorescence, among other examples).

It is important to remark, that, while the usual operation of big-science in Spain implies to finance the participation of our groups in big labs like CERN (including the annual CERN quota, the common-fund of the experiments and the contributions to construction and operation of the CERn experiments), the local science that NEXT represents obtains external funding (through ERC projects, including the AdG and several Actions now in progress involving LSC, as well as the contributions of the international collaboration to detector construction and operation and common fund) and attracts external talent to our country. The NEXT group is very international, and several of our post-docs are of have been financed by EC grants (such as the Marie Curie). 

Last but not least, the NEXT experiment, and in particular the collaboration with the CLPU involves the extensive development of photonics listed as one of the  "Facilitating Essential Technologies".