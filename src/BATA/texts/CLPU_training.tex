
\subsubsection*{Training capabilities of the team}

The BaTa sub-project requests a Ph. D student through a FPI grant.

{\bf Training program}
This project is a highly multi-disciplinary task. The development of this novel barium tagging technique involves both technical and fundamental paradigm shifts that will be accomplished in the participant institutions. For this reason there is a wide scope for student training. In addition to this, the project duration is four years, making it feasible for a  full PhD period. The CLPU plans to train a PhD student financially associated to this project with an FPI grant. 

The University of Salamanca offers a suitable PhD program for this purpose which is called \emph{Fundamental Physics and Mathematics} (http://www.usal.es/webusal/node/30172). The student will be as well fully involved in the NEXT experiment, attending the meetings and presenting results on high level conferences.

{\bf Ph.D. thesis}
The following thesis (relevant to this project) are being directed at the CLPU:

\begin{enumerate}

\item Title: ``Design and construction of an extreme flux proton source "\\
Institution: University of Salamanca\\
Student: Francisco Valle Brozas\\


\item Title: ``Investigations into particle acceleration towards hadron therapy "\\
Institution: University of Salamanca\\
Student: Luca C. Stockhaussen\\


\item Title: ``Femtosecond X-ray sources from laser-driven electron accelerators "\\
Institution: University of Salamanca\\
Student: Andreas D{\oe}pp\\

\end{enumerate}

Of course, all thesis have to be related to a Ph. D. program  of a university. Due to the agreement between the University of Salamanca (USAL) and the CLPU, and because several memebers of the CLPU staff are associated to the USAL, it is expected that most Ph. D.'s, and particularly the one within this project, are going to be done at the USAL.

