
\subsubsection*{Methodology of the BATA subproject}
The BaTa subproject is clearly divided in four progressive experiments to study the detection of Barium ions starting with the simplest configuration and ending with an experiment with conditions close to NEXT. Following the description of the main objectives in the \emph{Objectives of the BaTa subproject} section, we can develop the methodology to reach the goals. This project is mainly experimentla. Of course, some previous theory work for design and simulation is needed. However, the central effort is towards experimental new developments that eventually will result in new Spanish technology. The project takes profit of the existing know how at CLPU.

Next experiment relies on the detection of Ba$^{++}$, but the first excited state of Ba$^{++}$ lies at 132770.79\,cm$^{-1}$ which corresponds to an energy of 16.46\,eV, or equivalently to a radiation of around 75\,nm wavelength. Any radiation below 200\,nm, i.e., vacuum ultraviolet (VUV) or extreme ultraviolet (XUV), is not easily accessible, and its generation and handling involves major experimental difficulties. As a consequence, the tagging process must take place in singly ionized Ba, and accordingly, the recombination process Ba$^{++}\rightarrow$Ba$^{+}$ must be somehow stimulated. The first set of experiments explained in the \emph{Objectives of the BaTa subproject} will work with this assumption and by the end of the project the experiment conditions will try to replicate the ones of NEXT in order to demonstrate the validity of it or to study a way to provoke the process Ba$^{++}\rightarrow$Ba$^{+}$.

Lets go through all the different stages of the experiments we plan to realize.

\begin{itemize}
 \item \textbf{Proof of principle experiment with Ba ions generated by an ion source to be developed.}    
We will generate Ba ions using an electrical discharge inside a vacuum chamber designed specifically for this experiment. A laser designed and built entirely at CLPU will be used in order to pump the S$\rightarrow$P transition. A detector will be used to collect the fluorescence of the P$\rightarrow$D transition. We expect  very few photons at this stage, which makes it important to use a good set of filters that will remove all the possible scattered light and all the photos of other wavelength coming from the experiment of from the background. This experiment will be carried out at the CLPU.

The laser source we need to resonantly excite the $Ba^{+}$ will work at 493.5\,nm; this wavelength is not available in commercial lasers. We will produce it by nonlinear optical processes. In particular, we will optimize a Tunable Ti:sapphire laser at 987\,nm to obtain second harmonic generation (SHG) at 493.5\,nm. We need to design a 4-folded ring cavity and control of the wavelength by temperature of nonlinear crystal. This setup provides several advantages, as the etalon devices allow to tune the wavelength and control the bandwidth of the laser which is necessary to precisely tune it to the transition frequency.

This experiment will provide information on the population dynamics of the  $Ba^{+}$. 

\item \textbf{Proof of principle experiment with Ba ions generated by means of a developed ion source and with a magneto trap.}   
This experiment will rely on the previous development by CLPU of an ion source capable of providing the chosen ion specie. The development and construction of this ion source will be done with the support of a future EXPLORA project along 2015. Once we have a controllable source of ions we will reproduce the previous explained experiment  but knowing that there is not parasitic contributions of other compounds. The experiment will again be carried out at the CLPU.

 \item \textbf{Proof of principle experiment with Ba ions generated by means of a developed ion source and with a magneto trap.}
 In order to better control the experimental conditions, we will repeat the experiment but with the ions now confined in a magnetic trap. The design and construction of the magnetic trap will also be included in the EXPLORA project and is associated with the ion source. WIth this setup we will be able to study the recombination process Ba$^{++}\rightarrow$Ba$^{+}$. We will also study the collected fluorescence dependent on the surrounding pressure of Xe in order to \emph{copy} the conditions of NEXT.
 
  \item \textbf{Proof of principle experiment with an additional laser for deshelving the D state.}
  One of the main possible drawbacks of these experiments is the deshelving rate of the D state. If the rate is small, the collected fluorescence will be small. In order to increase the collected fluorescence by our detector, and thus increase the signal to noise ratio of NEXT, we can accelerate the deshelving of the D state. This can be done by shining the ions with a 649.87\,nm laser to pump the S$\rightarrow$P transition, or by a double photon process with a 4.1\,$\mu$m laser. The first option is simpler, but it means that we are introducing in the experiment the same light we want to detect. The second option presents a bigger challenge, but will assure us that the detected  649.87\,nm photons come from the fluorescence of the $Ba^{+}$. The main challenge of this experiment is the design and construction of the infrared laser as it is a starting technology and its final needs for NEXT are very specific. It is for this reason that we will purchase a commercial available unit that can reach the specifications for the preparatory experiments that we will carry out at the CLPU, but at the same time we will develop, in parallel with the first experiments, a new infrared laser. 
  
There is only a few laser systems that generate laser emission in the mid-Infrared (MIR, from 2 to 10\,$\mu$m). There are lasers that emit in discrete wavelengths as gas lasers ($CO_2$, Xe-HE, He-Ne), chemical lasers (Hidrogen Fluoride, Deuterium Fluoride) and Dye lasers by Raman Shift. Gas and Chemical laser systems lase out at 4.1\,$\mu$m with the drawback that it uses dangerous gases. These lasers are big and complex systems and emit in relative low power. Dye lasers have many drawbacks as low output power, they are very complex systems (in order to produce Raman shift) and that the Dye powders are carcinogenic. However, we have some alternatives to produce an emission of this wavelength with optically pumped solid state Laser (OPSSL) by a specific doping of crystals with metal transition ions. 2 and 2.9\,$\mu$m are available with $Tm^{3+}$, $Tm^{3+}$-$Ho^{3+}$ and $Er^{3+}$ active ions in crystalline matrices. However, recently we have some possibilities by doping with $Cr^{2+}$ and $Fe^{2+}$ ions properly host matrices. This should allow to develop a laser system that emits in a broad range: 2.1 to 3\,$\mu$m and 3.7 to 5\,$\mu$m, respectively. Another option is to work with quantum cascade semiconductor systems potentially available from 3.8 to 9.5\,$\mu$m in a discrete range, i. e.  not only continuously. 

Our plane is to develop a laser system around 4.1\,$\mu$m to deshelve the D state. For this purpose, we need to study and evaluate the best optical parameters of some materials (crystalline matrices or semiconductors) that can be used as active laser materials.  This will allow us to design a laser cavity with the appropriate  optical components and devices (these should work in the MIR region). It will also allow us to maximize the efficiency of the laser system. The cavities are different for systems optically pumped (the case of crystals doped with $Cr^{2+}$ o $Fe^{2+}$) or electrically pumped (as the quantum cascade semiconductors). These cavities can be 2-, 3- or 4- folded mirror configurations depending on the advantages observed during the design using optical and numerical software. The characterization of the laser emission and other related parameters will allow us  to improve the laser system to use in the NEXT experiment. This final stage will have to be iterative to select the best option among the studied designs. 

\end{itemize}



