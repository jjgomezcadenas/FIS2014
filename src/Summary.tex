%%

NEXT (Neutrino Experiment with a Xenon TPC) is an experiment to search neutrino less double beta decay processes (\bbonu). The detection of such processes would demonstrate that neutrinos are Majorana particles (that is their own antiparticles) and would have deep consequences in physics and cosmology.  

The isotope chosen by NEXT is  \XE. The collaboration has access to hundred kilograms of xenon enriched at 90\% in \XE, owned by the Underground Laboratory of Canfranc (LSC). The NEXT technology is based in the use of time projection chambers operating at a typical pressure of 15 bar and using electroluminescence to amplify the signal (\HPXE). The main advantages of the experimental technique are: a) excellent energy resolution; b) the ability to reconstruct the trajectory of the two electrons emitted in the decays, a unique feature of the \HPXE\ which further contributes to the suppression of backgrounds; c) scalability to large masses; and d) the possibility to reduce the background to negligible levels thanks to the barium tagging technology (\BATA).

The NEXT roadmap was designed in four stages: i) Demonstration of the \HPXE\ technology with prototypes deploying a mass of natural xenon in the range of 1 kg; ii) Characterisation of the backgrounds to the \bbonu\ signal and measurement of the \bbtnu\ signal with the NEW detector, deploying 12 kg of enriched xenon and operating at the LSC; iii) Search for \bbonu\ decays with the NEXT-100 detector, which escales up the NEW detector by a factor 2:1 in size (8:1 in mass) and deploys, thus, 100 kg of enriched xenon. iv) Search for \bbonu\ decays with the BEXT detector (Barium-tagging Experiment with a Xenon TPC), which will deploy a mass in the ton scale and will introduce the technology of \BATA\ in order to reduce backgrounds to negligible levels.  

The first stage of NEXT has been successfully completed during the period 2009-2013. The prototypes NEXT-DEMO (IFIC) and NEXT-DBDM (Berkeley) were built and operated for more than two years. These apparatus have demonstrated the main features of the technology. The experiment is currently developing its second phase. The NEW detector is being constructed during 2014 and will operate in the LSC during 2015. The funding for the construction and operation of NEW comes from an Advanced Grant (AdG/ERC) granted to the PI of this project in 2013. The NEXT-100 detector will be built and commissioned during 2016 and 2017 and will start data taking in 2018. NEXT-100 could discover \bbonu\ processes if the period of the decay is equal or less than $6 \times 10^{25}$~year. The fourth phase of the experiment (BEXT) could start in 2020. 

NEXT is an international collaboration, lead by spanish groups (the PI of this proposal is the spokesperson of the collaboration) and with a very significant contribution of US groups. The laser technology needed for the BEXT phase is being developed in collaboration with the Spanish Center for Pulsed Lasers (CLPU). 

 This proposal requires {\em co-funding} to complete the phase three of the experiment. Specifically we request: a) funds to co-finance the construction of the NEXT-100 detector (which is being partially payed by the AdG as well as by the international collaboration, primarily US groups); b) funds to co-finance personnel; and c) a modest contribution of the R\&D to develop the \BATA\ technology.   

