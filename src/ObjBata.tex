% Los objetivos específicos de cada uno de los subproyectos participantes, enumerándolos brevemente, con claridad, precisión y de manera realista (acorde con la duración prevista del proyecto).
%
% En los subproyectos con dos investigadores principales, deberá indicarse expresamente de qué objetivos específicos se hará responsable cada uno de ellos.

\subsubsection*{Objectives of the \BATA\ subproject}
The \BATA\ subproject will focus in the R\&D program needed to clearly establish the feasibility of tagging (e.g., detecting) the barium (Ba$^{++}$) ion produced in the double beta decay of xenon. Demonstrating that an efficient detection of barium is possible in an \HPXE\ would imply that the NEXT technology could be upgraded to the ton scale while at the same time reducing the background by two or more orders of magnitude, resulting in a virtually background-free experiment with enormous possibilities of hitting a discovery. 

The \BATA\ program involves a set of proof-of-concept experiments, and requires the developments of a 4.1 $\mu$m laser, needed to deshelf the metastable state. An attractive feature of such laser is its wide range of scientific and technological applications. CLPU is specially well suited to develop the technology.

The different objectives of this subproject are:

\begin{itemize}
	\item \textbf{Proof of principle experiment with Ba ions generated by means of an electrical discharge.}
In a first round of experiments we will excite resonantly the S$\leftrightarrow$P transition of Ba$^+$ ions generated by an electrical discharge between two barium electrodes and will collect the fluorescence signal of the P$\rightarrow$D transition. Although this generation method is not ideal because several different species different from Ba ions will be generated (e.g., molecules like BaO or clusters), it does not need a major technological development. It is expected that these initial set of experiments will provide valuable information about the population dynamics in Ba$^+$ ions, and the influence of the different homogenous and in-homogenous broadening mechanisms. It is important to mention that the laser system required for this objective will be provided by the CLPU.
	
	\item \textbf{Proof of principle experiment with Ba ions generated by an ion source.}	
In this objective, in order to get a better approximation of the final conditions of NEXT experiment a source of ions will be designed and constructed. This ion source will be based on selective ionisation and mass spectrometry techniques, and it will allow a perfect selection of a target species. Once the source is ready we will repeat the set of experiments of the previous objective but without any parasitic contribution of unwanted compounds. 
	
	\item \textbf{Proof of principle experiment with Ba ions generated by an ion source and with a magneto trap.}	
Once the ion source is in operation, we will develop a magneto trap for Ba$^+$  ions. This trap will allow us to have an excellent degree of control over the experimental conditions and to approach the conditions that can be expected in the NEXT experiment. For instance we will carry out different measurements comparing the collected fluorescence signal as a function of the pressure of the Ba$^+$ ions and the pressure of the surrounding environment. These measurements are mandatory because the population dynamics is very sensitive to pressure, i.e., to collisions. 
	
	\item \textbf{Proof of principle experiment with an additional laser for deshelving the D state.}
A possible scenario is that the collisional induced decay between the metastable state D and the ground state S is either not effective or too slow for obtaining an appreciable fluorescence signal. In this situation the population is trapped in the metastable state D  and the fluorescence cycle can not be closed. To avoid this difficulty our approach will be to use a second laser to induce a two photon transition (one photon is forbidden by selection rules, between the states D and S, see Fig.\,\ref{fig.BATA}). 

\item \textbf{Development of a state-of-the-art 4.1\,$\mu$m laser}. The laser needed for
deshelving the D state must have a wavelength of around 4.1\,$\mu$m. While small commercial laser system exists in this range, (we will use one of them for the experiment described in the previous paragraph), no commercial system will satisfy the conditions of power and stability needed for a real \BATA\ experiment. Furthermore, such a laser, already well in the infrared region, has many potential applications.	
	
\end{itemize}

\subsubsection*{\BATA\ subproject: Resources}

For the successful development of this subproject, CLPU will provide the required human and technological resources. CLPU is the centre of reference in Spain regarding laser technology, and takes active part in several international and national projects. The leader of this subproject will be Alicia V. Carpentier who has a well recognised international trajectory in laser-matter interaction. Moreover, {\bf CLPU considers this project of high priority and consequently will offer the collaboration of all the scientific department} consisting of a multidisciplinar team with broad experience in laser technology and development, and laser-matter interaction. 

Furthermore, CLPU will support this project with some of the already operating laser systems in its installation. This is extremely important because such systems usually cost of the order of several hundreds of thousand euros. The human resources needed to operate the laser systems will be provided by CLPU as well. For the construction of the ion source, and taking into consideration the specific requirements of this development, we will apply for an \emph{EXPLORA tecnología} in the 2014 call. 

The budget of this subproject will be dedicated to: a) purchase small equipment for the proof-of-principle experiments; b) purchase a small commercial infrared laser for the initial deshelving experiment; c) develop a state-of-the-art, high power, very stable infrared laser to be used in a large system. It is important to insist that such a laser has many possible applications given the fact that its wavelength is not absorbed by the atmosphere as it lies in what is called the infrared atmospheric window.


\subsubsection*{ENG subproject: schedule}



